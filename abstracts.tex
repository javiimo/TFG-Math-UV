\begin{poliabstract}
    In complex multi-criteria decision making (MCDM), a key challenge is the formal integration of criteria that are inherently vague, subjective, or expressed linguistically, a domain where conventional quantitative models often fall short. This work addresses this gap by developing and applying a comprehensive framework based on fuzzy logic to systematically model and resolve MCDM problems characterized by such imprecise information. The methodology first establishes the theoretical foundations of fuzzy set theory, focusing on the algebraic properties of T-norms, the principles of fuzzy arithmetic, and the formal structures of fuzzy logics and possibility distributions. It then details the core practical processes of fuzzification, translating expert knowledge and empirical data into fuzzy sets, and aggregation, detailing how operators like the Ordered Weighted Average (OWA) can reflect a decision-maker's attitudinal preferences, such as optimism or risk aversion. The framework's efficacy is demonstrated through an extensive case study on a real-world maintenance scheduling problem from the ROADEF/EURO 2020 Challenge, where a two-level aggregation model is applied to obtain the final ranking of 29 competing solutions. Ultimately, this research project showcases the capacity of fuzzy logic to create a flexible and transparent decision-support system, providing a structured yet adaptable approach that leads to more robust and justifiable outcomes in complex scenarios.
    \end{poliabstract}
\vspace{3em}
\begin{poliabstract}[Resumen]
    En la compleja toma de decisiones multi-criterio (MCDM), un desafío clave es la integración formal de criterios que son inherentemente vagos, subjetivos o expresados lingüísticamente, un dominio donde los modelos cuantitativos convencionales a menudo resultan insuficientes. Este trabajo aborda esta carencia mediante el desarrollo y la aplicación de un marco integral basado en la lógica difusa para modelar y resolver sistemáticamente problemas de MCDM caracterizados por dicha información imprecisa. La metodología establece primero los fundamentos teóricos de la teoría de conjuntos difusos, centrándose en las propiedades algebraicas de las T-normas, los principios de la aritmética difusa y las estructuras formales de las lógicas difusas y las distribuciones de posibilidad. A continuación, detalla los procesos prácticos clave de \textit{difusificación}, que consiste en traducir el conocimiento experto y los datos empíricos a conjuntos difusos, y de agregación, explicando cómo operadores como el Promedio Ponderado Ordenado (OWA) pueden reflejar las preferencias actitudinales de quien toma la decisión, como el optimismo o la aversión al riesgo. La eficacia del marco se demuestra a través de un caso de estudio sobre un problema real de planificación de mantenimiento del Desafío ROADEF/EURO 2020, donde se aplica un modelo de agregación de dos niveles para obtener la clasificación final de 29 soluciones competidoras. En última instancia, este proyecto de investigación demuestra la capacidad de la lógica difusa para crear un sistema de soporte a la decisión flexible y transparente, proporcionando un enfoque estructurado pero adaptable que conduce a resultados más robustos y justificables en escenarios complejos.
\end{poliabstract}
    
\newpage
\begin{poliabstract}[Resum]
    En la complexa presa de decisions multi-criteri (MCDM), un repte clau és la integració formal de criteris que són inherentment vagues, subjectius o expressats lingüísticament, un domini on els models quantitatius convencionals sovint resulten insuficients. Aquest treball aborda aquesta mancança mitjançant el desenvolupament i l'aplicació d'un marc integral basat en la lògica difusa per modelar i resoldre sistemàticament problemes de MCDM caracteritzats per aquesta informació imprecisa. La metodologia estableix primer els fonaments teòrics de la teoria de conjunts difusos, centrant-se en les propietats algebraiques de les T-normes, els principis de l'aritmètica difusa i les estructures formals de les lògiques difuses i les distribucions de possibilitat. Seguidament, detalla els processos pràctics clau de \textit{difusificació}, que tradueix el coneixement expert i les dades empíriques en conjunts difusos, i d'agregació, explicant com operadors com la Mitjana Ponderada Ordenada (OWA) poden reflectir les preferències actitudinals del decisor, com ara l'optimisme o l'aversió al risc. L'eficàcia del marc es demostra a través d'un estudi de cas exhaustiu sobre un problema real de planificació de manteniment del Repte ROADEF/EURO 2020, on s'aplica un model d'agregació de dos nivells per obtenir la classificació final de 29 solucions competidores. En darrera instància, aquest projecte de recerca demostra la capacitat de la lògica difusa per crear un sistema de suport a la decisió flexible i transparent, proporcionant un enfocament estructurat però adaptable que condueix a resultats més robustos i justificables en escenaris complexos.
\end{poliabstract}
