% \begin{notation}[label={not:OpsFS}]{Notation for variables in the current section}
%     In this section, we use the variable \( x \) to represent time and \( y \) to represent distance.
%   \end{notation}

The notions of union and intersection in fuzzy sets were first introduced by Zadeh \cite{Zadeh1965} using the operations $\max\{A(x),B(x)\}$ and $\min\{A(x),B(x)\}$ respectively. These can be intuitively interpreted as follows: the union is the \textit{smallest} fuzzy set (having lowest membership values) that contains both sets, while the intersection is the \textit{biggest} fuzzy set (having highest membership values) that is contained by both sets.\\

However, these operations can be generalized by two broader classes of operators: triangular norms (for intersection) and triangular conorms (for union).

Triangular norms were first introduced by Karl Menger in 1942 \cite{OriginTNorms} in the context of probabilistic metric spaces. When generalizing distances between points to probability distributions (representing the probability that the distance is less than or equal to a given value), Menger defined an operation $T:\,[0,1]\times [0,1]\to [0,1]$ to preserve the triangular inequality. For points $x,y,z$ in a metric space with distance function $d(\cdot,\cdot)$, this operation satisfies:

\begin{equation}\label{eq:Ftriangle_inequality}
d(x, z) \leq d(x, y) + d(y, z) \quad \longrightarrow \quad F_{xz}(t + s) \geq T(F_{xy}(t), F_{yz}(s)) \quad \forall t,s \geq 0
\end{equation}

This inequality means that the probability of $d(x,z)$ being less than $t+s$ must be at least the t-norm of the probabilities that $d(x,y)<t$ and $d(y,z)<s$. Note the change from $\leq$ to $\geq$ in the inequality. This is consistent with \textit{larger} probabilities indicating \textit{smaller} distances are more likely.\\

Since this originated in the context of distances, the following properties were required for an operator to be a t-norm\footnote{Associativity and one identity were not originally proposed by Menger but were later added by Sklar and Schweizer \cite{Sklar1983} in their refinement of triangular norms}:

\begin{itemize}
  \item \textbf{Symmetry:} The order of combining probabilities shouldn't matter, just as intersection of sets is commutative. That is, combining probabilities for distances $(x,y)$ and $(y,z)$ should give the same result regardless of order.
  
  \item \textbf{Associativity:} When combining multiple probabilities (e.g., for paths through points $x,y,z,w$), the grouping shouldn't affect the result. This extends the triangular norm to be consistent with polygonal inequalities, similar to how nested intersections satisfy $(A \cap B) \cap C = A \cap (B \cap C)$.
  
  \item \textbf{Monotonicity:} If the probability $F_{xy}(t)$ increases, then the lower bound for $F_{xz}(t+s)$ given by $T(F_{xy}(t), F_{yz}(s))$ should not decrease. This is analogous to how adding elements in crisp sets (or increasing membership degrees in fuzzy sets) cannot reduce the intersection set.
  
  \item \textbf{One Identity:} If $F_{yz}(s) = 1$ (meaning $d(y,z) < s$ with certainty), then $F_{xz}(t+s)$ depends only on $F_{xy}(t)$. This is analogous to how intersecting with the universal set preserves the original set.
\end{itemize}


% \say{The name \textit{triangular norm} refers to the fact that in the framework of probabilistic metric spaces, t-norms and t-conorms are used to generalize triangle inequality of ordinary metric spaces.}\cite{NGAN2018}\\

Therefore, the concept of intersection (conjunction) of fuzzy sets is generally represented by a triangular norm (also called a t-norm).

\begin{definition}[Triangular Norm]
    A mapping $T:[0,1]\times [0,1] \longrightarrow [0,1]$ that satisfies:
    \begin{romanenum}
      \item \textbf{Symmetricity:} $T(x,y) = T(y,x) \quad \oldforall x,y \in [0,1]$
      \item \textbf{Associativity:} $T(x,T(y,z)) = T(T(x,y),z) \quad \oldforall x,y,z \in [0,1]$
      \item \textbf{Monotonicity:} $T(x,y) \leq T(x',y') \quad \textnormal{if }x\leq x' \textnormal{ and } y\leq y' \quad \oldforall x,y,x',y' \in [0,1]$
      \item \textbf{One Identity:} $T(x,1) = T(1,x) = x \quad \oldforall x \in [0,1]$
    \end{romanenum}
    is called a triangular norm or t-norm. Defines the \textbf{intersection} of two fuzzy sets $A$ and $B$ on $X$ by giving the membership function as $(A \cap B) (x) = T(A(x),B(x)) \forall x \in X$ 
\end{definition}

Its dual operator can also be obtained by a similar reasoning as before, but instead of considering the probability distribution of finding both points closer than a given distance, it considers the probability distribution ($G_{uv}(t) = 1 - F_{uv}(t)$) of finding them further apart than that distance. In this case, both inequalities are $\leq$ since larger probabilities indicate that greater distances are more likely.
\begin{equation}\label{eq:Gtriangle_inequality}
d(x, z) \leq d(x, y) + d(y, z) \quad \longrightarrow \quad G_{xz}(t + s) \leq S(G_{xy}(t), G_{yz}(s))
\end{equation}

The reasoning regarding the properties is entirely analogous to the previous case, with the only difference being that $F_{uv}(t) = 1 \Leftrightarrow  G_{uv}(t) = 0$, and here the identity element is zero (union with the empty set).



\begin{definition}[Triangular Conorm]
  A mapping $S:[0,1]\times [0,1] \longrightarrow [0,1]$ that satisfies:
  \begin{enumerate}[(i)]\setlength{\itemindent}{2em}
    \item \textbf{Symmetricity:} $S(x,y) = S(y,x) \quad \oldforall x,y \in [0,1]$
    \item \textbf{Associativity:} $S(x,S(y,z)) = S(S(x,y),z) \quad \oldforall x,y,z \in [0,1]$
    \item \textbf{Monotonicity:} $S(x,y) \leq S(x',y') \quad \textnormal{if }x\leq x' \textnormal{ and } y\leq y' \quad \oldforall x,y,x',y' \in [0,1]$
    \item \textbf{Zero Identity:} $S(x,0) = S(0,x) = x \quad \oldforall x \in [0,1]$
  \end{enumerate}
  is called a triangular conorm or t-conorm. Defines the \textbf{union} of two fuzzy sets $A$ and $B$ on $X$ by giving the membership function as $(A \cup  B) (x) = S(A(x),B(x)) \forall x \in X$ 
    
\end{definition}

Complement was defined by Zadeh \cite{Zadeh1965} as\footnote{This is not the only definition that satisfies the axioms of a complement and is compatible with the classical limit, but it is the simplest one and will be used in this text. Other alternatives and their axioms can be found in \cite{Sladoje2007}. In \cite{Klement2000}, this is called the standard negation $N_s$.}:

\begin{definition}[Complement]
  The complement of a fuzzy set $A\in \fuzzy{X}$ is another fuzzy set with membership function given by $^\lnot A(x) \coleq 1 - A(x) \forall x\in X$
\end{definition}

Notice that this definition of complement is consistent with the classical definition of complement but implies that an element might have \textbf{non-zero partial membership} to both a fuzzy set and its complement: Let $A$ be a fuzzy set on $X$ and $x \in X / A(x)\notin \{0,1\}$ then $\lnot A(x)= 1 - A(x) \notin \{0,1\}$.\\

One consequence of this fact is that the union of a fuzzy set and its complement is not the total set in general. Analogously, the intersection will not the empty set in general. Those two properties that hold in classical sets but might not be true in fuzzy sets, are often called the \textbf{laws of excluded middle and of non-contradiction}, respectively.\\

However, there are t-norms and t-conorms such as the ones named after \luka that do satisfy both laws\footnote{Indeed, all continuous t-norms in agreement with non-contradiction are isomorphic to \luka \cite[p.~7]{LukasiewiczNonContrad}.}. Another example is the drastic t-norm which also satisfies the law of non-contradiction, but not the law of excluded middle. See example \ref{ex:basic_tnorms} for their formal expressions. \signal{All of this, has implications for the derived logic that will be explained in section \ref{sec:fuzzy_logic}. }\\

Another important property that classical union and intersection satisfy is De Morgan's Laws. For an arbitrary pair of t-norm and t-conorm, these laws are not automatically satisfied. However, there are specific pairs that do fulfill them. To illustrate the relationship between t-norms and t-conorms that satisfy De Morgan's Laws, we can use our probabilistic metric space analogy. Reconsidering the probabilistic metrics $F,G$ introduced earlier and substituting $F = 1 - G$ into equation \ref{eq:Ftriangle_inequality}, we obtain:

\[ G_{xz}(t + s) \leq 1 - T(1 - G_{xy}(t), 1 - G_{yz}(s))\]

Comparing this with equation \ref{eq:Gtriangle_inequality}, we can derive a relationship between t-norms and t-conorms, which is formalized in the following proposition:

\begin{proposition}[Relationship between t-norm and t-conorm]
  Given a t-norm $T$, the t-conorm is $S(a,b)\coleq 1 - T(1-a, 1-b)$ if and only if the union and intersection defined by that pair satisfy the De Morgan's Laws.
\end{proposition}
\begin{remark}
  It is easy to see that the previous relation is equivalent to $T(a,b) = 1-S(1-a, 1-b)$ which can be obtained simply by substituting $a'=1-a$ and $b'=1-b$, i.e., working with the complementary fuzzy sets.
\end{remark}

\begin{proof}
  Let $x\in X$, $A$, $B$ be fuzzy sets over $X$ with $a \coleq A(x)$ and $b \coleq B(x)$\\

  $\quad \boxed{\text{not}(A \text{ or } B) = (\text{not } A) \text{ and } (\text{not } B)}$\\
  [0.5em]
  $\lnot S(a,b) = T(\lnot a, \lnot b) \iff 1 - S(a,b) = T(1-a, 1-b) \iff S(a,b) = 1 - T(1-a, 1-b)$\\

  $\quad \boxed{\text{not}(A \text{ and } B) = (\text{not } A) \text{ or } (\text{not } B)}$\\
  [0.5em]
  $\lnot T(a,b) = S(\lnot a, \lnot b) \iff 1 - T(a,b) = S(1-a, 1-b) \iff T(a,b) = 1 - S(1-a, 1-b)$

\end{proof}

\signal{Creo que además de esto de las leyes de de Morgan, también nos da que el modus ponen funciona si se cumple la propiedad esa. Además no sé si se llama residual property.}

\signal{
  Lo de Archimedean sirve para el teorema 1.8.1 de \cite{FULLER2}. Y tb con la law of large numbers con LR-fuzzy numbers.}

  \begin{example}[Basic T-norms {\cite[Ex.~1.2]{Klement2000}}]\label{ex:basic_tnorms}
    The following are the four fundamental (basic) t-norms, which serve as canonical examples because they represent the extremal and most important cases with respect to the order relation and algebraic properties defined in Section~\ref{sec:class_tnorms}. From weaker to stronger they are: $T_D \leq T_L \leq T_P \leq T_M$.
      \begin{itemize}
        \item \textbf{Minimum ($T_M$):} 
            This is the strongest t-norm (\cite[Rem.~1.5(i)]{Klement2000}). Every element is idempotent (indeed it is the only t-norm satisfying that \cite[Lem.~1.2.3]{FULLER2}), therefore it is not Archimedean. It has no zero divisors and no nilpotent elements other than 0.
    
        \[T_M(x, y) = \min(x, y) \quad S_M(x, y) = \max(x, y)
    \]
        \item \textbf{Product ($T_P$):} 
        This t-norm is strict Archimedean (\cite[Ex.~2.14(i)]{Klement2000}). It has only $0$ and $1$ as idempotent elements, no nilpotent elements (other than $0$), and no zero divisors (\cite[Ex.~2.2(i)]{Klement2000}).
        \[T_P(x, y) = x \cdot y \quad S_P(x, y) = x + y - x \cdot y\]
        \item \textbf{Łukasiewicz ($T_L$):} 
        This t-norm is nilpotent Archimedean (\cite[Ex.~2.14(i)]{Klement2000}). It has only $0$ and $1$ as idempotent elements. Every $a \in ]0,1[$ is a nilpotent element and also a zero divisor (\cite[Ex.~2.2(i)]{Klement2000}).
        \[T_L(x, y) = \max(0, x + y - 1) \quad S_L(x, y) = \min(1, x + y)\]
        \item \textbf{Drastic Product ($T_D$):} This is the weakest t-norm (\cite[Rem.~1.5(i)]{Klement2000}). It is Archimedean since $T_D(x,x)=0 < x$ for $x \in ]0,1[$. It has only $0$ and $1$ as idempotent elements. Every $a \in ]0,1[$ is a zero divisor, and also nilpotent (since $a_D^{(2)} = T_D(a, T_D(a,a)) = T_D(a,0) = 0$ for $a<1$). It is not continuous.
        \[T_D(x, y) = \begin{cases} \min(x,y) & \text{if } \max(x,y)=1 \\ 0 & \text{otherwise} \end{cases} \quad S_D(x, y) = \begin{cases} \max(x,y) & \text{if } \min(x,y)=0 \\ 1 & \text{otherwise} \end{cases}\]
      \end{itemize}
    \end{example}

  \subsection{Classification of T-norms}

\signal{Añadir lo de strict Archimedean y divisores de zero y eso.}
\begin{definition}[Archimedean t-norm]
  A continuous t-norm that satisfies $T(x,x)<x \forall x\in ]0,1[$ is called an Archimedean t-norm.
\end{definition}

\begin{proposition}[Characterization of Archimedean t-norms]
  For all Archimedean t-norms there exists a continuous decreasing function $f:[0,1] \longrightarrow [0,\infty[$ with $f(1)=0$ such that: 
  \[ 
  T(x,y)= f^{-1}(\min\{f(x)+f(y), f(0)\}) \text{ where } f^{-1} =
  \begin{cases}
    f^{-1}(y) & \text{if } y\in [0,f(0) ]\\
    0 & \text{otherwise}
  \end{cases}
  \text{ is a pseudo-inverse}.
  \]
\end{proposition}


\signal{
\begin{definition}[Nilpotent t-norm]
  
\end{definition}}

\begin{definition}[Weaker t-norm]
  Given $T_1, T_2$ t-norms, then $T_1$ is weaker than $T_2$ $(T_1 \leq T_2)$ if $T_1(x,y)\leq T_2(x,y)\forall x,y\in [0,1]$.\\
  In that case, it is equivalent to say $T_2$ is stronger than $T_1$ $(T_2 \geq T_1)$

\end{definition}
\begin{remark}
  This defines a partial order relation in the set of t-norms.
\end{remark}
\signal{The weaker the t-norm, the stronger the associated s-norm. Lo pongo como un remark igual.}

\signal{
There are many results like all t-norms are between the weak and the min, all t-conorms are between max and strong, or that min is the only t-norm that satisfies $T(a,a)=a$ (igual es por esto ultimo q se usa tanto. Qué implicaciones tiene que lukasiewicz no cumpla eso?)}

\signal{Tambien lo de que la T-norm e distributiva con max/sup sirve para justificar la definicion del producto cartesiano, así que esa propiedad la tendré que meter por aquí igual.}


\noindent\rule{\textwidth}{2pt}


\subsection{Classification of T-norms}
The first and most straight forward way to classify them is by defining the following partial order on the set of all t-norms. 
\begin{definition}[Weaker/Stronger t-norm {\cite[Def.~1.4]{Klement2000}}]
  Given two t-norms $T_1$ and $T_2$, $T_1$ is said to be \emph{weaker} than $T_2$ (denoted $T_1 \leq T_2$) if $T_1(x,y) \leq T_2(x,y) \forall x,y \in [0,1]$.
  Equivalently, $T_2$ is said to be \emph{stronger} than $T_1$.
\end{definition}

\begin{remark}
  It's a fundamental result that for any t-norm $T$, we have $T_D \leq T \leq T_M$, where $T_D$ and $T_M$ are the drastic and the minimum t-norms respectively (\cite[Rem.~1.5]{Klement2000}).
\end{remark}

The interpretation is that a weaker t-norm can be seen as a stricter and more pessimistic intersection (conjunction in the fuzzy logic derived), returning lower membership values than a stronger one. Notice that it is not a total order: there are pairs of t-norms where none of them is weaker than the other. Some examples, such as the product $T_P$ and Yager $T_2^Y$ t-norms, are shown in \cite[Fig.~6.1]{Klement2000}.\\


In order to further classify t-norms, the following algebraic properties will be needed \cite[Def.~2.1]{Klement2000}:

\begin{definition}[Nilpotent Element]
Let $T$ be a t-norm. An element $a \in ]0,1[$ is a \emph{nilpotent element} of $T$ if there exists $n \in \mathbb{N}$ such that $a_T^{(n)} = 0$, where $a_T^{(n)} = T(a, a_T^{(n-1)})$ with $a_T^{(1)}=a$.
\end{definition}

\begin{definition}[Zero Divisor]
Let $T$ be a t-norm. An element $a \in ]0,1[$ is a \emph{zero divisor} of $T$ if there exists $b \in ]0,1[$ such that $T(a,b)=0$.
\end{definition}

\begin{definition}[Idempotent Element]
  Let $T$ be a t-norm. An element $a \in [0,1]$ is an \emph{idempotent element} of $T$ if $T(a,a)=a$. The elements $0$ and $1$ are always trivial idempotent elements.
  \end{definition}
  

\begin{definition}[Archimedean T-norm]
A t-norm $T$ is \emph{Archimedean} if for each $(x,y) \in ]0,1[^2$ there is an $n \in \mathbb{N}$ with $x_T^{(n)} < y$. \cite[Def.~2.9]{Klement2000}  

In particular, a \textit{continuous} t-norm $T$ is Archimedean if and only if $T(x,x) < x \forall x \in ]0,1[$, i.e. it doesn't have non-trivial idempotent elements. \cite[Thm.~2.12]{Klement2000}
\end{definition}

This Archimedean property is the key idea that is used to classify continuous t-norms since continuous Archimedean t-norms can be splitted into two classes (closed under isomorphisms): strict (isomorphic to $T_P$) and nilpotent (isomorphic to $T_L$). \signal{Then, it is possible to express any continuous (Archimedean or not) t-norm as an ordinal sum of continuous archimedean t-norms.}\\

Before defining t-norms



\subsection{Classification of Continuous T-norms}

Continuous t-norms form a particularly well-structured class, admitting elegant representation theorems. Their study often revolves around whether they are Archimedean.

\subsubsection{Continuous Archimedean T-norms and Generators}
The structure of continuous Archimedean t-norms is intimately linked to certain functions called generators.
\begin{definition}[Additive Generator and Pseudo-inverse]
  An \emph{additive generator} of a t-norm $T$ is a strictly decreasing function $t: [0,1] \to [0,\infty]$ which is right-continuous in $0$ and satisfies $t(1)=0$, such that $T(x,y) = t^{(-1)}(t(x) + t(y))$ for all $(x,y) \in [0,1]^2$, provided $t(x)+t(y) \in \mathrm{Ran}(t) \cup [t(0),\infty]$ (\cite[Def.~3.25, p.~70]{Klement2000}).
  The function $t^{(-1)}: [0,\infty] \to [0,1]$ is the \emph{pseudo-inverse} of $t$, defined as $t^{(-1)}(y) = \sup \{ x \in [0,1] \mid t(x) > y \}$ (adapted from \cite[Def.~3.2, p.~68 and Cor.~3.3]{Klement2000} for strictly decreasing $t$). \signal{`Ran(t)` denotes the range of $t$, i.e., the set of all values $t(x)$ for $x \in [0,1]$ \cite[p. xvii]{Klement2000}.}
\end{definition}

\begin{theorem}[Representation of Continuous Archimedean T-norms {\cite[Thm.~5.1, p.~122]{Klement2000}}]
  A t-norm $T$ is a continuous Archimedean t-norm if and only if it possesses a continuous additive generator $t: [0,1] \to [0,\infty]$. This generator is unique up to a positive multiplicative constant.
\end{theorem}
Continuous Archimedean t-norms are further categorized based on the behavior of their generator at $0$:
\begin{itemize}
    \item $T$ is \textbf{strict} if its continuous additive generator $t$ satisfies $t(0)=\infty$. This means $T(x,y)>0$ whenever $x,y > 0$. Strict t-norms are strictly monotone on $]0,1]^2$. (\cite[Cor.~3.30(i), p.~88; Def.~2.13(i), p.~42]{Klement2000}).
    \item $T$ is \textbf{nilpotent} if its continuous additive generator $t$ satisfies $t(0)<\infty$. This implies that for any $x,y \in ]0,1[$, there exists $n$ such that $T(x, \dots, x)$ ($n$ times) is $0$. Every $a \in ]0,1[$ is a nilpotent element. (\cite[Cor.~3.30(ii), p.~88; Def.~2.13(ii), p.~42]{Klement2000}).
\end{itemize}

\subsubsection{Isomorphism of Continuous Archimedean T-norms}
The concept of isomorphism reveals a deep structural similarity among t-norms within these classes.
\begin{definition}[Isomorphic T-norms {\cite[Def.~2.27, p.~51; Prop.~2.28(iv), p.~52]{Klement2000}}]
  Two t-norms $T_1$ and $T_2$ are \emph{isomorphic} if there exists a strictly increasing bijection $\varphi: [0,1] \to [0,1]$ (an automorphism of the unit interval) such that $T_2(x,y) = \varphi^{-1}(T_1(\varphi(x), \varphi(y)))$ for all $x,y \in [0,1]$.
\end{definition}
Isomorphic t-norms share the same algebraic structure, merely operating on rescaled inputs and outputs via $\varphi$. A fundamental result is:
\begin{proposition}[{\cite[Cor.~5.7, p.~125, referring to Prop.~5.9 and Prop.~5.10]{Klement2000}}]
  \begin{enumerate}
      \item Every strict t-norm is isomorphic to the Product t-norm $T_P$.
      \item Every nilpotent t-norm is isomorphic to the Łukasiewicz t-norm $T_L$.
  \end{enumerate}
\end{proposition}
This implies that, up to isomorphism, there are only two distinct types of continuous Archimedean t-norms: the product type and the Łukasiewicz type.

\subsubsection{General Continuous T-norms and Ordinal Sums}
Continuous t-norms that are not Archimedean must have non-trivial idempotent elements. These are constructed using ordinal sums.
\begin{definition}[Ordinal Sum of T-norms {\cite[Def.~3.44, p.~97]{Klement2000}}]
Let $(T_\alpha)_{\alpha \in A}$ be a family of t-norms and $(]a_\alpha, e_\alpha[)_{\alpha \in A}$ be a family of non-empty, pairwise disjoint open subintervals of $[0,1]$. The t-norm $T$ defined by
\[
T(x,y) =
\begin{cases}
  a_\alpha + (e_\alpha - a_\alpha) \cdot T_\alpha \left( \frac{x-a_\alpha}{e_\alpha - a_\alpha}, \frac{y-a_\alpha}{e_\alpha - a_\alpha} \right) & \text{if } (x,y) \in [a_\alpha, e_\alpha]^2 \text{ for some } \alpha \in A \\
  \min(x,y) & \text{otherwise}
\end{cases}
\]
is called the \emph{ordinal sum} of the summands $(a_\alpha, e_\alpha, T_\alpha)$, $\alpha \in A$.
\end{definition}
Intuitively, an ordinal sum "glues" copies of the t-norms $T_\alpha$ (scaled to fit the intervals $[a_\alpha, e_\alpha]$) onto the diagonal of the unit square. Outside these "active" regions, the t-norm behaves like the minimum $T_M$. The endpoints $a_\alpha, e_\alpha$ become idempotent elements of $T$.

\begin{theorem}[Representation of Continuous T-norms {\cite[Thm.~5.11, p.~140]{Klement2000}}]
  A function $T: [0,1]^2 \to [0,1]$ is a continuous t-norm if and only if $T$ is uniquely representable as an ordinal sum of continuous Archimedean t-norms.
\end{theorem}
\begin{remark}
  A continuous t-norm is \emph{ordinally irreducible} if its only ordinal sum representation is $T = ((0,1,T))$. For continuous t-norms, being ordinally irreducible is equivalent to being Archimedean (\cite[Prop.~3.53, p.~99 and context]{Klement2000}). Thus, the "building blocks" in the ordinal sum representation are precisely the continuous Archimedean t-norms (isomorphic to $T_P$ or $T_L$). If the family of subintervals is empty, the ordinal sum is defined as $T_M$.
\end{remark}

\subsection{Further Examples: Parametric Families of T-norms}
Beyond the four basic t-norms, several parametric families offer a spectrum of behaviors and are widely used. Their construction often relies on generators.
\begin{definition}[Multiplicative Generator {\cite[Def.~3.36, p.~91]{Klement2000}}]
  A \emph{multiplicative generator} $\theta: [0,1] \to [0,1]$ of a t-norm $T$ is a strictly increasing function, right-continuous in $0$, with $\theta(1)=1$, such that $T(x,y) = \theta^{(-1)}(\theta(x) \cdot \theta(y))$, assuming $\theta(x)\cdot\theta(y)$ is in a suitable range. \signal{This is a simplified statement; the book's definition includes range conditions.}
\end{definition}
\begin{remark}[Duality of Generators {\cite[Rem.~3.34, p.~90]{Klement2000}}]
  If $t(x)$ is an additive generator, then $\theta(x) = e^{-c \cdot t(x)}$ (for some $c>0$) is a multiplicative generator, and if $\theta(x)$ is a multiplicative generator, then $t(x) = -c \cdot \log(\theta(x))$ is an additive generator.
\end{remark}

\begin{example}[Parametric Families of T-norms (selected from {\cite[Chapter 4]{Klement2000}})]
  \begin{itemize}
    \item \textbf{Schweizer-Sklar T-norms ($T_\lambda^{SS}$), $\lambda \in [-\infty, \infty]$:}
    $T_\lambda^{SS}(x,y) = (\max(0, x^\lambda + y^\lambda - 1))^{1/\lambda}$.
    Includes $T_M (\lambda=-\infty)$, $T_P (\lambda \to 0)$, $T_L (\lambda=1)$, $T_D (\lambda=\infty)$. Additive generator $t_\lambda^{SS}(x) = \frac{1-x^\lambda}{\lambda}$ for $\lambda \neq 0$.
    \item \textbf{Frank T-norms ($T_\lambda^F$), $\lambda \in [0, \infty]$:}
    $T_\lambda^F(x,y) = \log_\lambda \left(1 + \frac{(\lambda^x-1)(\lambda^y-1)}{\lambda-1}\right)$ for $\lambda \in ]0,\infty[, \lambda \neq 1$.
    Includes $T_M (\lambda=0)$, $T_P (\lambda=1)$, $T_L (\lambda=\infty)$. All $T_\lambda^F$ for $\lambda \in ]0,\infty]$ are continuous Archimedean. Additive generator $t_\lambda^F(x) = -\log \frac{\lambda^x-1}{\lambda-1}$.
    \item \textbf{Yager T-norms ($T_p^Y$), $p \in [0, \infty]$:} \signal{The book uses $\lambda$ for Yager, but $p$ is more common in literature to avoid clash with Frank's parameter.}
    $T_p^Y(x,y) = \max(0, 1 - ((1-x)^p + (1-y)^p)^{1/p})$ for $p \in ]0,\infty[$.
    Includes $T_D (p \to 0)$, $T_L (p=1)$, $T_M (p=\infty)$. All $T_p^Y$ for $p \in ]0,\infty[$ are continuous nilpotent Archimedean. Additive generator $t_p^Y(x) = (1-x)^p$.
  \end{itemize}
  \signal{Many other families exist, like Hamacher, Dombi, Aczél-Alsina, Sugeno-Weber, Mayor-Torrens. Refer to \cite[Table 4.1, p.~119]{Klement2000} and Appendix A therein for a comprehensive list and properties.}
\end{example}

\subsection{Non-Continuous T-norms and Left-Continuity}
While continuous t-norms are well-classified, not all t-norms are continuous.
The \textbf{drastic product $T_D$} is a key example of a non-continuous t-norm. It is Archimedean. It is upper semicontinuous, implying right-continuity in each variable when the other is fixed (\cite[Rem.~1.21(i), Prop.~1.22]{Klement2000}). However, it is not left-continuous (e.g., at $(1,y)$ for $y<1$).

The \textbf{nilpotent minimum $T^{nM}$} (\cite[Rem.~1.21(i), p.~16]{Klement2000}) is defined as:
  \[
  T^{nM}(x,y) =
  \begin{cases}
    0 & \text{if } x+y \leq 1 \\
    \min(x,y) & \text{otherwise.}
  \end{cases}
  \]
This t-norm is lower semicontinuous, which for monotone functions implies it is left-continuous in each variable (\cite[Prop.~1.22, p.~17]{Klement2000}). It is not continuous (specifically, not right-continuous at points on the line $x+y=1$ when approached from $x+y>1$).

The \textbf{Krause t-norm $T^K$} (\cite[App.~B.1, Thm.~B.1]{Klement2000}) is a more complex example. It is constructed using the Cantor set and Farey series. It is stated to be "neither left- nor right-continuous, but has a continuous diagonal section." This highlights that t-norms can exhibit quite irregular continuity behavior.

\begin{remark}[Importance of Left-Continuity]
For non-continuous t-norms, the property of \emph{left-continuity} (in each variable) is often a desirable, or even required, condition in certain applications, particularly in fuzzy logic. For instance, in residuum-based logics, if a t-norm $T$ is left-continuous, its corresponding residuated implication $I(x,y) = \sup\{z \in [0,1] \mid T(x,z) \le y\}$ exhibits well-behaved properties. Specifically, a commutative, integral lattice-ordered monoid based on $T$ is residuated if and only if $T$ is left-continuous (\cite[Prop.~2.47, p.~63]{Klement2000}). This ensures that the implication adequately captures deductive reasoning. While the intuitive notion that "a microscopic decrease of the truth degree of a conjunct should not macroscopically decrease the truth degree of the conjunction" points towards continuity, left-continuity is a weaker but often sufficient condition for preserving logical coherence in such frameworks.
\end{remark}


\noindent\rule{\textwidth}{2pt}


\subsection{Classification of T-norms}
\signal{Hablar de los generadores de las t-norms (familias) y propiedades. Diagrama de qué t-norms son weaker que otras. Cuáles son continuas y cuales no (drastic por ejemplo)}

\signal{Normalmente se pide que sean left continuous (it is sufficient in either argument \cite{Esteva2001MonoidalTB}), which expresses the assumption that a microscopic decrease of the truth degree of a conjunct should not macroscopically decrease the truth degree of conjunction.}

The set of all t-norms can be partially ordered, and understanding their properties allows for a useful classification, particularly for continuous t-norms.

\begin{definition}[Weaker/Stronger t-norm {\citep[Definition 1.4, p.~21]{Klement2000}}]
  Given two t-norms $T_1$ and $T_2$, $T_1$ is said to be \emph{weaker} than $T_2$ (denoted $T_1 \leq T_2$) if $T_1(x,y) \leq T_2(x,y)$ for all $x,y \in [0,1]$.
  Equivalently, $T_2$ is said to be \emph{stronger} than $T_1$.
\end{definition}

\begin{remark}
  The relation $\leq$ defines a partial order on the set of all t-norms.
  It's a fundamental result that for any t-norm $T$, we have $T_D \leq T \leq T_M$, where $T_D$ is the drastic product and $T_M$ is the minimum t-norm (\citep[Remark 1.5(i), p.~21]{Klement2000}).
\end{remark}

\begin{remark}
  Duality with t-conorms: If $T_1 \leq T_2$ are t-norms, and $S_1, S_2$ are their respective dual t-conorms (obtained via a strong negation $N$, typically $N(x)=1-x$), then $S_1 \geq S_2$. Thus, a weaker t-norm corresponds to a stronger dual t-conorm.
\end{remark}

To classify t-norms further, several algebraic properties are essential:

\begin{definition}[Algebraic Properties of T-norms {\citep[Definition 2.1, Definition 2.9]{Klement2000}}]
Let $T$ be a t-norm.
\begin{enumerate}
    \item An element $a \in [0,1]$ is an \emph{idempotent element} of $T$ if $T(a,a)=a$. The elements $0$ and $1$ are always trivial idempotent elements.
    \item An element $a \in ]0,1[$ is a \emph{nilpotent element} of $T$ if there exists $n \in \mathbb{N}$ such that $a_T^{(n)} = 0$, where $a_T^{(n)}$ denotes the $n$-th power of $a$ with respect to $T$.
    \item An element $a \in ]0,1[$ is a \emph{zero divisor} of $T$ if there exists $b \in ]0,1[$ such that $T(a,b)=0$.
    \item $T$ is \emph{Archimedean} if for each $(x,y) \in ]0,1[^2$ there is an $n \in \mathbb{N}$ with $x_T^{(n)} < y$. Equivalently, a t-norm $T$ is Archimedean if and only if it has only trivial idempotent elements (0 and 1) and satisfies a further condition related to limits, or more simply for continuous t-norms, if $T(x,x) < x$ for all $x \in ]0,1[$ (\citep[Theorem 2.12 and Theorem 5.1]{Klement2000}).
\end{enumerate}
\end{definition}

\begin{proposition}[{\citep[Proposition 1.9(i), p.~24]{Klement2000}}]
  The minimum t-norm $T_M(x,y) = \min(x,y)$ is the only t-norm $T$ satisfying $T(x,x)=x$ for all $x \in [0,1]$.
\end{proposition}

\paragraph{Continuous T-norms}
Continuous t-norms admit a very structured classification.

\begin{theorem}[Generator Theorem for Continuous Archimedean T-norms {\citep[Theorem 5.1, p.~122]{Klement2000}}]
  A function $T: [0,1]^2 \to [0,1]$ is a continuous Archimedean t-norm if and only if it has a \emph{continuous additive generator}. That is, there exists a continuous, strictly decreasing function $t: [0,1] \to [0,\infty]$ with $t(1)=0$, such that for all $(x,y) \in [0,1]^2$:
  \[
  T(x,y) = t^{(-1)}(t(x) + t(y))
  \]
  where $t^{(-1)}: [0,\infty] \to [0,1]$ is the pseudo-inverse of $t$, defined as
  \[
  t^{(-1)}(z) =
  \begin{cases}
    t^{-1}(z) & \text{if } z \in [0, t(0)] \\
    0 & \text{if } z \in ]t(0), \infty]
  \end{cases}
  \]
  (If $t(0)=\infty$, then $t^{(-1)} = t^{-1}$). The generator $t$ is unique up to a positive multiplicative constant.
\end{theorem}

Continuous Archimedean t-norms are further divided into two main classes:

\begin{definition}[Strict and Nilpotent T-norms {\citep[Definition 2.13, p.~42]{Klement2000}}]
  \begin{enumerate}
      \item A t-norm $T$ is called \emph{strict} if it is continuous and strictly monotone (i.e., $T(x,y) < T(x,z)$ whenever $x>0$ and $y<z$).
      \item A t-norm $T$ is called \emph{nilpotent} if it is continuous and if each $a \in ]0,1[$ is a nilpotent element of $T$.
  \end{enumerate}
\end{definition}

\begin{corollary}[{\citep[Corollary 3.30, p.~88]{Klement2000}}]
  Let $T$ be a continuous Archimedean t-norm with a continuous additive generator $t$.
  \begin{enumerate}
      \item $T$ is strict if and only if $t(0) = \infty$.
      \item $T$ is nilpotent if and only if $t(0) < \infty$.
  \end{enumerate}
\end{corollary}

\begin{proposition}[Isomorphism of Continuous Archimedean T-norms {\citep[Corollary 5.7, p.~125; Proposition 5.9, p.~126; Proposition 5.10, p.~127]{Klement2000}}]
  \begin{enumerate}
      \item Every strict t-norm is isomorphic to the product t-norm $T_P(x,y) = xy$.
      \item Every nilpotent t-norm is isomorphic to the Łukasiewicz t-norm $T_L(x,y) = \max(0, x+y-1)$.
  \end{enumerate}
  This means that any continuous Archimedean t-norm is isomorphic to either $T_P$ or $T_L$.
\end{proposition}

\paragraph{Ordinal Sum Representation}
Not all continuous t-norms are Archimedean. Those that are not (i.e., possess non-trivial idempotents) can be constructed from Archimedean t-norms using the concept of an ordinal sum.

\begin{definition}[Ordinal Sum of T-norms {\citep[Definition 3.44, p.~97]{Klement2000}}]
Let $(T_\alpha)_{\alpha \in A}$ be a family of t-norms and $(]a_\alpha, e_\alpha[)_{\alpha \in A}$ be a family of non-empty, pairwise disjoint open subintervals of $[0,1]$. The t-norm $T$ defined by
\[
T(x,y) =
\begin{cases}
  a_\alpha + (e_\alpha - a_\alpha) \cdot T_\alpha \left( \frac{x-a_\alpha}{e_\alpha - a_\alpha}, \frac{y-a_\alpha}{e_\alpha - a_\alpha} \right) & \text{if } (x,y) \in [a_\alpha, e_\alpha]^2 \text{ for some } \alpha \in A \\
  \min(x,y) & \text{otherwise}
\end{cases}
\]
is called the \emph{ordinal sum} of the summands $(a_\alpha, e_\alpha, T_\alpha)$, $\alpha \in A$.
We denote this as $T = ((a_\alpha, e_\alpha, T_\alpha))_{\alpha \in A}$.
\end{definition}

\begin{theorem}[Representation of Continuous T-norms {\citep[Theorem 5.11, p.~140]{Klement2000}}]
  A function $T: [0,1]^2 \to [0,1]$ is a continuous t-norm if and only if $T$ is uniquely representable as an ordinal sum of continuous Archimedean t-norms.
\end{theorem}

\begin{remark}
  A t-norm is \emph{ordinally irreducible} if it only has a trivial ordinal sum representation (i.e., $T = ((0,1,T))$). For continuous t-norms, being ordinally irreducible is equivalent to being Archimedean (\citep[Proposition 3.53, p.~99]{Klement2000}).
\end{remark}

In summary, continuous t-norms are either Archimedean (and thus isomorphic to $T_P$ or $T_L$) or they are ordinal sums of such Archimedean t-norms (and $T_M$ as a limiting case).





\begin{example}
  \signal{Some examples of t-norms and t-conorms.}
  \begin{itemize}
    \item \textbf{Minimum/Maximum:} 
    The strongest t-norm (minimum) and weakest t-conorm (maximum). They are the standard interpretations of conjunction and disjunction in fuzzy logic and are idempotent ($T(x,x)=x, S(x,x)=x$).\\
    $T_M(x, y) = \min(x, y) \quad S_M(x, y) = \max(x, y)$

    \item \textbf{Łukasiewicz:}
    Represents a bounded arithmetic sum/difference logic. Corresponds to the logic introduced by Łukasiewicz. It is Archimedean but not strict (has zero divisors). Boundary condition for t-norms satisfying $T(x,y)+S(x,y) = x+y$. \\
    $T_L(x, y) = \max(0, x + y - 1) \quad S_L(x, y) = \min(1, x + y)$

    \item \textbf{Product:}
    The standard algebraic product t-norm and the probabilistic sum t-conorm. Represents the intersection of independent events in probability theory. It is a strict Archimedean t-norm. \\
    $T_P(x, y) = x \cdot y \quad S_P(x, y) = x + y - x \cdot y$

    \item \textbf{Drastic:}
    The weakest t-norm and the strongest t-conorm. They represent the most extreme intersection and union possible. \signal{Discontinuous except at boundary points (0,0), (1,1), etc.} \\
    $T_D(x, y) = \begin{cases} \min(x,y) & \text{if } \max(x,y)=1 \\ 0 & \text{otherwise} \end{cases} \quad S_D(x, y) = \begin{cases} \max(x,y) & \text{if } \min(x,y)=0 \\ 1 & \text{otherwise} \end{cases}$
    % Simpler equivalent definitions:
    % T(x, y) = y if x=1, x if y=1, 0 otherwise
    % S(x, y) = y if x=0, x if y=0, 1 otherwise

    \item \textbf{Hamacher Family ($\gamma \ge 0$):}
    A parametric family of t-norms and t-conorms. Includes the Product t-norm as a special case when $\gamma = 1$. It is strictly Archimedean for $\gamma > 0$. \\
    $T_\gamma(x, y) = \frac{xy}{\gamma + (1-\gamma)(x+y-xy)} \quad S_\gamma(x, y) = \frac{x+y-xy-(1-\gamma)xy}{1-(1-\gamma)xy}$
    % Note: For gamma=0 this is related to Lukasiewicz via generator, for gamma -> infinity to Drastic.

    \item \textbf{Dubois and Prade Family ($\alpha \in [0, 1]$):}
    A parametric family that includes the Minimum t-norm ($\alpha=0$) and the Product t-norm ($\alpha=1$). Provides flexibility between these two important t-norms. \\
    $T_\alpha(x, y) = \frac{xy}{\max(x, y, \alpha)} \quad S_\alpha(x, y) = 1 - \frac{(1-x)(1-y)}{\max(1-x, 1-y, \alpha)}$ 
    % Note: The S-conorm is expressed directly via duality relation S(x,y) = 1 - T(1-x, 1-y).

    \item \textbf{Yager Family ($p > 0$):}
    A parametric family characterized by the parameter $p$. It approaches the Drastic t-norm as $p \to 0^+$ and the Minimum t-norm as $p \to \infty$. The Łukasiewicz t-norm is obtained for $p=1$. \\
    $T_p(x, y) = \max(0, 1 - ((1-x)^p + (1-y)^p)^{1/p}) \quad S_p(x, y) = \min(1, (x^p + y^p)^{1/p})$

    \item \textbf{Frank Family ($p > 0, p \neq 1$):}
    The only family of t-norms (besides min-max) that are Archimedean and satisfy the functional equation $T(x, y) + S(x, y) = x + y$. It includes Łukasiewicz ($p \to 0^+$), Product ($p \to 1$), and Minimum/Maximum ($p \to \infty$) as limiting cases. \\
    $T_p(x, y) = \log_p \left( 1 + \frac{(p^x - 1)(p^y - 1)}{p - 1} \right) \quad S_p(x, y) = 1 - \log_p \left( 1 + \frac{(p^{1-x} - 1)(p^{1-y} - 1)}{p - 1} \right)$

    \item \textbf{Schweizer-Sklar Family ($p \in [-\infty, \infty]$):}
    A very general parametric family encompassing several others. Includes Drastic ($p \to -\infty$), Łukasiewicz ($p=1$ in a different parameterization, not this T form), Product ($p=0$ in a related form), and Minimum ($p \to \infty$). The Yager family is generated differently but related. \\
    $T_p(x, y) = (\max(0, x^p + y^p - 1))^{1/p} \quad S_p(x, y) = 1 - (\max(0, (1-x)^p + (1-y)^p - 1))^{1/p}$ 
    % Note: S-conorm shown is the dual S_p(x,y) = 1 - T_p(1-x, 1-y). The commonly cited S_p(x,y) = (min(1, x^p + y^p))^{1/p} is NOT dual to this T_p w.r.t standard negation.
  \end{itemize}
\end{example}





\subsection{Construction and Properties of T-norms} % Example section

\subsubsection{Generators for T-norms}

One of the most powerful methods for constructing and characterizing t-norms, particularly continuous Archimedean ones, involves the use of generator functions.

\begin{definition}[Additive Generator {\cite[Definition 3.25, p.~70]{Klement2000}}]
  An \emph{additive generator} $t: [0,1] \to [0,\infty]$ of a t-norm $T$ is a strictly decreasing function which is also right-continuous in $0$ and satisfies $t(1)=0$, such that for all $(x,y) \in [0,1]^2$:
  \begin{enumerate}
      \item $t(x) + t(y) \in \mathrm{Ran}(t) \cup [t(0), \infty]$, and
      \item $T(x,y) = t^{(-1)}(t(x) + t(y))$,
  \end{enumerate}
  where $t^{(-1)}$ is the pseudo-inverse of $t$.
\end{definition}

\begin{remark}
  As established in \cite[Theorem 5.1, p.~122]{Klement2000}, every continuous Archimedean t-norm possesses a continuous additive generator, unique up to a positive multiplicative constant. Conversely, any such generator defines a continuous Archimedean t-norm.
\end{remark}

\begin{definition}[Multiplicative Generator {\cite[Definition 3.36, p.~91]{Klement2000}}]
  A \emph{multiplicative generator} $\theta: [0,1] \to [0,1]$ of a t-norm $T$ is a strictly increasing function which is right-continuous in $0$ and satisfies $\theta(1)=1$, such that for all $(x,y) \in [0,1]^2$:
  \begin{enumerate}
      \item $\theta(x) \cdot \theta(y) \in \mathrm{Ran}(\theta) \cup [0, \theta(0)]$, and
      \item $T(x,y) = \theta^{(-1)}(\theta(x) \cdot \theta(y))$,
  \end{enumerate}
  where $\theta^{(-1)}$ is the pseudo-inverse of $\theta$.
\end{definition}

\begin{remark}
  There is a direct duality between additive and multiplicative generators. If $t$ is an additive generator, then $\theta(x) = e^{-t(x)}$ (or $e^{-c \cdot t(x)}$ for $c>0$) can serve as a multiplicative generator, and vice-versa with $t(x) = -\log(\theta(x))$ (\cite[Remark 3.34, p.~90]{Klement2000}). Continuous Archimedean t-norms also possess continuous multiplicative generators (\cite[Corollary 5.4, p.~124]{Klement2000}).
\end{remark}

\subsubsection{Families of T-norms}

Chapter 4 of \cite{Klement2000} presents various parameterized families of t-norms. These families often interpolate between or include the basic t-norms ($T_M, T_P, T_L, T_D$) as special or limiting cases. Some prominent examples include:

\begin{itemize}
    \item \textbf{Schweizer-Sklar T-norms} ($T_\lambda^{SS}$) for $\lambda \in [-\infty, \infty]$ (\cite[Example 4.3, p.~104]{Klement2000}). This family includes $T_M$ ($\lambda=-\infty$), $T_P$ ($\lambda=0$), and $T_L$ ($\lambda=1$), and $T_D$ ($\lambda=\infty$).
    \item \textbf{Hamacher T-norms} ($T_\lambda^H$) for $\lambda \in [0, \infty]$ (\cite[Example 4.5, p.~106]{Klement2000}). Includes $T_P$ ($\lambda=1$) and $T_D$ ($\lambda=\infty$). $T_0^H$ is a notable member.
    \item \textbf{Frank T-norms} ($T_\lambda^F$) for $\lambda \in [0, \infty]$ (\cite[Example 4.7, p.~108]{Klement2000}). This family includes $T_M$ ($\lambda=0$), $T_P$ ($\lambda=1$), and $T_L$ ($\lambda=\infty$).
    \item \textbf{Yager T-norms} ($T_\lambda^Y$) for $\lambda \in [0, \infty]$ (\cite[Example 4.9, p.~110]{Klement2000}). Includes $T_D$ ($\lambda=0$) and $T_L$ ($\lambda=1$), and $T_M$ ($\lambda=\infty$).
    \item \textbf{Dombi T-norms} ($T_\lambda^D$) for $\lambda \in [0, \infty]$ (\cite[Example 4.11, p.~112]{Klement2000}). Includes $T_D$ ($\lambda=0$) and $T_M$ ($\lambda=\infty$).
    \item \textbf{Aczél-Alsina T-norms} ($T_\lambda^{AA}$) for $\lambda \in [0, \infty]$ (\cite[Example 4.15, p.~116]{Klement2000}). Includes $T_D$ ($\lambda=0$) and $T_M$ ($\lambda=\infty$).
    \item \textbf{Sugeno-Weber T-norms} ($T_\lambda^{SW}$) for $\lambda \in [-1, \infty]$ (\cite[Example 4.13, p.~114]{Klement2000}). Includes $T_L$ ($\lambda=0$) and $T_P$ ($\lambda=\infty$). $T_{-1}^{SW}$ is $T_D$.
    \item \textbf{Mayor-Torrens T-norms} ($T_\lambda^{MT}$) for $\lambda \in [0, 1]$ (\cite[Example 4.17, p.~118]{Klement2000}). This family consists of ordinal sums and includes $T_M$ ($\lambda=0$) and $T_L$ ($\lambda=1$).
\end{itemize}
A summary table of properties for these families, including their continuity and Archimedean nature for different parameter values, can be found in \cite[Table 4.1, p.~119]{Klement2000}. Further details and visualizations are in Appendix A of the book.

\subsection{Continuity Properties of T-norms}

T-norms can exhibit various continuity behaviors.

\begin{example}[Continuous T-norms]
  The basic t-norms $T_M(x,y) = \min(x,y)$, $T_P(x,y) = xy$, and $T_L(x,y) = \max(0, x+y-1)$ are all continuous on $[0,1]^2$ (\cite[p.~15]{Klement2000}). All strict and nilpotent t-norms are, by definition, continuous (\cite[Definition 2.13, p.~42]{Klement2000}).
\end{example}

\begin{remark}[Left-continuous and Right-continuous T-norms]
  The book primarily focuses on continuity in both variables simultaneously or discusses left/right continuity of generator functions. However, some t-norms constructed via ordinal sums using non-continuous components or certain constructions in Chapter 3.4 can exhibit more nuanced continuity.
  For example, any t-norm $T$ that is an ordinal sum (Definition 3.44) will be continuous if and only if all its summands $T_\alpha$ are continuous and certain conditions at the join points $a_\alpha, e_\alpha$ are met (related to limits, see \cite[Proposition 3.49, p.~100]{Klement2000}).
\end{remark}

\begin{example}[Right-continuous, but not Left-continuous T-norm]
  The drastic product $T_D(x,y)$ is not continuous. For instance, consider a sequence $(x_n, y_n) = (1-\frac{1}{n}, 1-\frac{1}{n})$ converging to $(1,1)$.
  $T_D(x_n, y_n) = 0$ for all $n$, so $\lim_{n\to\infty} T_D(x_n, y_n) = 0$.
  However, $T_D(1,1) = 1$.
  It is upper semicontinuous, which implies it is right-continuous in each variable when the other is fixed (see \cite[Remark 1.21, Proposition 1.22]{Klement2000} which link upper/lower semicontinuity to right/left continuity of monotone functions).
  Specifically, $T_D$ is right-continuous in each variable but not left-continuous. For example, fixing $y=1$, $T_D(x,1)=x$ is continuous. Fixing $y \in [0,1[$, $T_D(x,y)$ is $0$ for $x \in [0,1[$ and $y$ for $x=1$. This is right-continuous at $x=0$ (for $y<1$) but not left-continuous at $x=1$.
\end{example}

\begin{example}[Left-continuous, but not Right-continuous T-norm]
  The book mentions the nilpotent minimum $T^{nM}$ (\cite[Remark 1.21, p.~16 and Fig 1.5, p.~17]{Klement2000}) defined as:
  \[
  T^{nM}(x,y) =
  \begin{cases}
    0 & \text{if } x+y \leq 1 \\
    \min(x,y) & \text{otherwise}
  \end{cases}
  \]
  This t-norm is lower semicontinuous, which for monotone functions implies it is left-continuous in each variable (\cite[Proposition 1.22, p.~17]{Klement2000}). It is not upper semicontinuous (and thus not right-continuous in general). For example, let $x_n = 0.5 + \frac{1}{n}$ and $y_n = 0.5 + \frac{1}{n}$. Then $x_n+y_n > 1$, so $T^{nM}(x_n, y_n) = 0.5 + \frac{1}{n} \to 0.5$. However, $T^{nM}(0.5, 0.5) = 0$.
\end{example}

\begin{example}[Discontinuous T-norm (neither left nor right continuous in general)]
  The Krause t-norm $T^K$ (\cite[Appendix B.1, p.~341 and Theorem B.1, p.~344]{Klement2000}) is explicitly stated to be "neither left- nor right-continuous, but has a continuous diagonal section."
  Another example of a t-norm that is generally discontinuous (unless specific parameters are chosen) is the t-norm from \cite[Example 3.21, p.~80]{Klement2000}, constructed using a discontinuous generator $f$. The resulting t-norm $(T_P)_{[f(-1)]}$ would generally be discontinuous.
\end{example}

\begin{remark}
  The book also mentions t-norms that are "border continuous" (\cite[Definition 1.23, p.~17]{Klement2000}), meaning they are continuous on the boundary of $[0,1]^2$ but not necessarily in the interior. \cite[Example 1.24(i), p.~18]{Klement2000} gives such an example which is border continuous but not left-continuous (and hence not fully continuous).
\end{remark}




















































