
\subsection{Fuzzy Conjunctions, Implications, and Foundational Logics}

In classical logic, conjunction corresponds to set intersection. When generalizing to fuzzy sets, the truth function for conjunction should capture the intuition that \say{a large truth degree of $\phi \text{ \& } \psi$ should indicate that both the truth degree of $\phi$ and the truth degree of $\psi$ is large}\cite[p.27]{Hajek1998}. This leads to the adoption of \textbf{t-norms} (triangular norms) as the truth functions for fuzzy conjunction.

A t-norm $T(x, y)$ is a binary operation on $[0,1]$ which is associative, commutative, non-decreasing in both arguments, and uses $1$ as a unit element ($T(1, x) = x$). Selection of a specific t-norm for conjunction is fundamental in specifying a system of fuzzy logic, as it determines the behavior of "and" at the truth-functional level.

Once a t-norm is chosen, the implication connective (denoted as $x \Rightarrow y$) is usually derived through the principle of \textbf{residuation}, that is, as the largest $z$ such that $T(x, z) \le y$; formally, $x \Rightarrow y = \sup\{z \in [0,1] \mid T(x, z) \le y\}$. This ensures the property of adjointness, guaranteeing the deduction rule modus ponens. For this residuum to be well defined and to satisfy adjointness, the t-norm $T$ must be at least \textbf{left-continuous}~\cite[p.272]{GodoMonoidal}.

This insight underpins \textbf{Monoidal T-norm Logic (MTL)}, which captures exactly the common tautologies of all fuzzy systems based on left-continuous t-norms and their residua~\cite{GodoMonoidal}. MTL is formulated with two primitive conjunctions: strong conjunction ($\&$, interpreted by a left-continuous t-norm) and weak conjunction ($\wedge$, interpreted by min). In general, for logics based on merely left-continuous t-norms, weak conjunction is not definable from strong conjunction and implication, which is why it must be included as a primitive. The algebraic semantics of MTL are given by linearly ordered prelinear residuated lattices, or MTL-algebras.

If continuity of the t-norm is imposed (rather than just left-continuity), the so-called \emph{divisibility property} holds: $x * (x \Rightarrow y) = \min(x, y)$. In this case, the min-conjunction becomes definable in terms of strong conjunction and implication, and the class of logics is known as \textbf{Basic Logic (BL)}~\cite[Sec. 2.2]{Hajek1998}. BL-algebras provide the algebraic semantics for these logics.

Significantly, all continuous t-norms can be decomposed (as an ordinal sum) from three basic shapes: the \textbf{Łukasiewicz t-norm} ($T_L(x, y) = \max(0, x+y-1)$), the \textbf{Gödel t-norm} ($T_G(x, y) = \min(x, y)$), and the \textbf{Product t-norm} ($T_P(x, y) = x y$)~\cite[Thm. 2.1.16]{Hajek1998}. These yield the three major systems of fuzzy logic, obtained by extending BL with axioms reflecting these t-norms:
\begin{itemize}
    \item \textbf{Łukasiewicz Logic ($\L$)} is characterized by involutive negation ($1-x$) and an axiom equivalent to $\lnot\lnot\phi \rightarrow \phi$. Its algebraic structures are MV-algebras~\cite[Ch. 3]{Hajek1998}.
    \item \textbf{Gödel Logic (G)} is based on the minimum t-norm and is characterized by the idempotence of its conjunction: $\phi \rightarrow (\phi \& \phi)$; its semantics are given by G-algebras, a subclass of linearly ordered Heyting algebras~\cite[Sec. 4.2]{Hajek1998}.
    \item \textbf{Product Logic ($\Pi$)} is based on the product t-norm and incorporates additional properties characteristic of multiplication. Its standard negation is the Gödel-style negation: $\lnot x$ is $1$ if $x=0$, $0$ if $x>0$; algebraically, its models are product algebras~\cite[Sec. 4.1]{Hajek1998}.
\end{itemize}

Fuzzy \textbf{disjunction (OR)} and \textbf{negation (NOT)} in these logics are typically generalized from classical set union and complement. Negation is defined in MTL, BL, and their extensions as $\lnot \phi := \phi \rightarrow 0$. Its specific properties depend on the logic (involutive in $\L$, not so in G and $\Pi$). Disjunction is definable, e.g., in MTL, by $\phi \vee \psi := \big((\phi \rightarrow \psi) \rightarrow \psi\big) \wedge \big((\psi \rightarrow \phi) \rightarrow \phi\big)$~\cite[Def. 1]{GodoMonoidal}~\cite[Def. 2.2.1]{Hajek1998}; additional forms and their interpretations are described in Section~\ref{sec:set_operations}.

\subsection{Fuzzy Implications}

The implication connective ($P \rightarrow Q$, "IF $P$ THEN $Q$") is arguably the most critical for logical reasoning and building rule-based systems, such as those used to model expert opinion in MCDM, a topic that will be explored further in the next chapter. Its generalization to fuzzy logic presents a rich variety of approaches. A fuzzy implication $I(x,y)$ (where $x$ is the truth value of the antecedent and $y$ that of the consequent) aims to quantify "to what extent $q$ is at least as true as $p$" (\cite[p.57]{FULLER2}). Different formalizations capture this intuition in different ways. The choice among them depends heavily on the specific application and the desired properties.

\paragraph{R-Implications (Residuated Implications)}
This class of implications arises directly from the residuation principle and forms the "native" implication for the logics MTL, BL, Ł, G, and $\Pi$. Given a left-continuous t-norm $T$ (representing the strong conjunction $\&$ of the logic), its corresponding R-implication $I_R$ is defined as its residuum:
\[
I_R(x, y) = \sup\{z \in [0,1] \mid T(x, z) \le y\}
\]
This property (adjointness: $T(x,z) \le y$ iff $z \le I_R(x, y)$) is essential for the soundness of modus ponens in such logics, and underlies the algebraic semantics of MTL-algebras, BL-algebras, MV-algebras, G-algebras, and product algebras~\cite[Sec. 2.3, Ch. 3, Ch. 4]{Hajek1998}~\cite{GodoMonoidal}.

\begin{example}[R-Implications from Continuous T-norms~\cite{Hajek1998}]
The main continuous t-norms lead to these R-implications (standard in the respective logics):
\begin{itemize}
    \item \textbf{Łukasiewicz implication}:
        \[
        I_L(x, y) = \min(1, 1-x+y)
        \]
    \item \textbf{Gödel implication}:
        \[
        I_G(x, y) = 
        \begin{cases}
            1 & \text{if } x \le y \\
            y & \text{if } x > y
        \end{cases}
        \]
    \item \textbf{Goguen (Product) implication}:
        \[
        I_P(x, y) = 
        \begin{cases}
            1 & \text{if } x \le y \\
            y/x & \text{if } x > y
        \end{cases}
        \]
\end{itemize}
Left-continuous but non-continuous t-norms yield further R-implications at the MTL level~\cite{GodoMonoidal}.
\end{example}

\paragraph{S-Implications (Standard Implications)}
S-implications generalize the classical equivalence $(P \rightarrow Q) \equiv (\lnot P \vee Q)$, and are defined using a fuzzy negation and a t-conorm $S$:
\[
I_S(x, y) = S(\overline{x}, y)
\]
For example, with standard negation and maximum t-conorm, the Kleene-Dienes implication is recovered:
\[
I_{KD}(x, y) = \max(1-x, y)
\]
If the Łukasiewicz t-conorm is used ($S_L(a, b) = \min(1, a+b)$), the resulting implication coincides with the Łukasiewicz R-implication.

\paragraph{Implications Defined Directly by T-norms}
In many practical settings, particularly fuzzy control and systems modeling, implications are simply defined as $I(x, y) = T(x, y)$, e.g., Mamdani's and Larsen's implications. Although practical, these fail to satisfy the full properties of material implication (such as $I(0,0) = 1$) and generally do not supply adjointness, so they are best viewed as separate from the formal logics above.

\paragraph{Standard Strict Implication}
A different fuzzy implication is $I_{SS}(x,y)$, which is $1$ when $x\le y$ and $0$ otherwise. While highly direct, it is too brittle for practical purposes and not used in the formal fuzzy logics discussed here.

\subsection{Characteristics and Intuitions of Foundational Fuzzy Logics}

The foundational logics—MTL, BL, $\L$, G, and $\Pi$—form a hierarchy: MTL is most general, based on all left-continuous t-norms and their residua; BL restricts to continuous t-norms; and each of $\L$, G, $\Pi$ is a further axiomatic specialization of BL connected to a distinguished continuous t-norm.

\begin{itemize}
    \item \textbf{Generality}: MTL omits the divisibility property ($x*(x \Rightarrow y)=\min(x,y)$), so min-conjunction is not definable from strong conjunction and implication; BL recovers it for continuous t-norms.
    \item \textbf{Negation}: $\L$ has involutive (double) negation, G and $\Pi$ use a Gödel-type negation, while MTL and BL only ensure weaker forms (involutive negation may be added as an axiom for special subclasses like IMTL~\cite{GodoMonoidal}).
    \item \textbf{Nature of conjunction (t-norm)}:
        \begin{itemize}
            \item $\L$: Additive-like, compensatory.
            \item G: Minimum, conservative (weakest link/idempotent).
            \item $\Pi$: Product, multiplicative.
        \end{itemize}
    \item \textbf{Deduction theorem}: G has a stronger deduction theorem than the others; in $\L$, $\Pi$, BL, and MTL, only a weaker, "n-fold" deduction theorem holds.
    \item \textbf{Completeness}: BL, $\L$, G, and $\Pi$ are complete with respect to their standard [0,1] semantics as based on continuous t-norms. MTL is complete w.r.t.\ linearly ordered MTL-algebras and, non-trivially, for the left-continuous t-norms~\cite{Jenei2001MTLCompl}. G has full (not only 1-tautology) completeness for arbitrary theories~\cite[Thm. 4.2.17]{Hajek1998}; completeness of BL and $\L$ is for 1-tautologies, and extensions vary.
    \item \textbf{Relationship to classical logic}: Pairwise combinations of $\L$, G, and $\Pi$ generate classical logic~\cite[Thm. 4.3.9]{Hajek1998}.
\end{itemize}
Each logic is therefore tailored for different reasoning needs. MTL, as the most general, accommodates any left-continuous t-norm and provides the universal backbone for fuzzy deduction systems.

\subsection{Approximate Reasoning and Pavelka-Style Completeness}

Classical deductive systems typically deal with absolute truth: premises are assumed to be fully true (truth value 1), and sound inference rules guarantee that conclusions are also fully true. However, in many real-world scenarios and in the spirit of fuzzy logic, we often reason with information that is only partially true or true to a certain degree. This leads to the field of **approximate reasoning**, where we are interested in the degree to which a conclusion follows from premises that themselves have degrees of truth.

A significant contribution to formalizing reasoning with degrees of truth was made by Jan Pavelka in the 1970s, particularly for logics related to Łukasiewicz logic. His work was later refined and integrated into the broader t-norm based framework by Hájek. The core idea is to extend a fuzzy logical system (like Łukasiewicz logic, denoted L) with rational truth constants $\bar{r}$ for every rational $r \in [0,1] \cap \mathbb{Q}$. A formula $(\bar{r} \rightarrow \phi)$ can then be interpreted as stating "$\phi$ is true to a degree at least $r$".

This framework allows for the definition of two key concepts for a given theory $T$ (a set of axioms, each potentially associated with a truth degree or assumed to be fully true):

\begin{itemize}
    \item \textbf{Truth Degree of $\phi$ over $T$}: $||\phi||_T = \inf\{e(\phi) \mid e \text{ is a model of } T\}$. This is the lowest degree to which $\phi$ is true in all models that satisfy the theory $T$.
    \item \textbf{Provability Degree of $\phi$ over $T$}: $|\phi|_T = \sup\{r \mid T \vdash (\bar{r} \rightarrow \phi)\}$. This is the highest degree $r$ such that it is provable from $T$ that $\phi$ is true to degree at least $r$.
\end{itemize}

Pavelka's completeness theorem for Rational Pavelka Logic (RPL), which is Łukasiewicz logic extended with rational truth-value constants and appropriate "bookkeeping" axioms for these constants, establishes a fundamental connection between these semantic and syntactic degrees.

\begin{theorem}[Pavelka-style Completeness for RPL]
For any theory $T$ in Rational Pavelka Logic and any formula $\phi$:
\[
 |\phi|_T = ||\phi||_T 
\]
\end{theorem}

This remarkable result states that the highest degree to which a formula is provable from a theory $T$ is precisely the lowest degree to which it is true in all models of $T$. It provides a formal calculus for "approximate reasoning" where conclusions can be derived with associated degrees of truth based on the degrees of truth of the premises. For instance, if premises $A_1, \ldots, A_k$ have associated truth degrees $r_1, \ldots, r_k$, and a conclusion $C$ can be derived (possibly using fuzzy deductions) with provability degree $s$ that depends on $r_1, \ldots, r_k$, Pavelka's theorem ensures that $s$ is the best possible semantic guarantee for $C$. While the original RPL is specific to Łukasiewicz logic due to the properties of the Łukasiewicz t-norm and its residuum, the general concept carries over to truth-functional logics where analogous positional properties hold.

This highlights a specific strength of Łukasiewicz logic for fine-grained approximate reasoning with degrees of truth. In contrast, for MTL and its extensions (unless they collapse to Łukasiewicz logic), Pavelka-style completeness in this exact sense is, in general, unattainable because of differences in the interplay between syntax and semantics, and the greater diversity of their algebraic models~\cite[Rem. 4.1.22, Rem. 4.2.22]{Hajek1998}.
