
Considering partial memberships and working with fuzzy sets has many implications for the logical framework that is derived from it. The field is vast and is best understood from the perspective of algebraic logic (see appendix \ref{app:alg_log}). However, this section only aims to very briefly introduce without proof some of the main results and concepts related to fuzzy logics: fuzzy implications, different fuzzy logics derived from t-norms and approximate reasoning.\\

In classical propositional logic\footnote{See appendix \ref{app:form_log} for a brief introduction to the concepts of formal logic}, there is a very straightforward relation between set-membership and truth values. It is the same to state \say{$x$ belongs to $A$} or \say{(it is true that) $x$ is $A$}. This direct correspondence extends to the fundamental operations of sets and logical connectives: stating that \say{$x$ belongs to the intersection of $A$ and $B$} (denoted $x \in A \cap B$) is equivalent to asserting that \say{$x$ is $A$ AND $x$ is $B$} (the logical conjunction $P_A(x) \land P_B(x)$ is true). The case for union $\cup$ and disjunction $\lor$ is analogous. Finally, the concept of an element $x$ belonging to the negation of set $A$ ($x \in \overline{A}$), meaning $x$ does not belong to $A$, directly mirrors logical negation ($\neg$). When working with fuzzy sets, we consider partial truth (degrees of truth as degrees of membership).\\

The implication connective is rooted in the residuation (or adjointness) property, which views implication $I$ ($A \Rightarrow B$, or "if $A$, then $B$") as a form of logical division: $A \land I \models B$ in logic (or $A \cap I \subseteq B$ in set theory). The goal is to find the weakest proposition (or largest set)\footnote{Intersection with a larger set $I$ is less restrictive respect to inclusion of sets. If intersection with the largest set is contained, intersection with any smaller set will also be contained.} $I$ such that combining $A$ with $I$ through conjunction (intersection) yields something at least as strong as (contained in) $B$. This largest $I$ satisfying the condition defines the implication $A \Rightarrow B$, internalizing the notion of logical consequence. Logical equivalence ($A \iff B$, or "$A$ if and only if $B$") occurs when both $A \Rightarrow B$ and $B \Rightarrow A$ hold. In set theory, this corresponds to mutual subset relations ($A \subseteq B$ and $B \subseteq A$), which can only be true when sets $A$ and $B$ are identical. Thus, logical equivalence between propositions directly corresponds to set equality.



\subsection{Fuzzy implications}
S-implications
R-implications
t-norm implications
\signal{$(\fuzzy{X},\cup , \cap , \lnot)$ is a complete, completely distributive, lattice with an involution. This extends the boolean algebra.}

\signal{The notion of equality is replaced by a graded relation (often measured via the biresiduum $\leftrightarrow$)}

\signal{Fuzzy description logics explained %https://www.umbertostraccia.it/cs/download/papers/KES09/KES09.pdf

The Lukasiewicz t-conorm is closely related to the basic binary operation of multi-valued
algebras. Additionally, t-norms and t-conorms form examples of aggregation operators. They
play a significant role in the axiomatic definition of the concept of triangular norm-based measure
and, in particular, of the concept of probability of fuzzy events; the Frank family of t-norms and
t-conorms plays a particular role [6]. 

%https://cake.fiu.edu/Publications/Ngan+al-18-LC.Logic_Connectives_of_Complex_Fuzzy_Sets_ROMJIST_downloaded.pdf 

It should be mentioned that t-norms overlap with copulas [3, 24]: commutative associative
copulas are t-norms; t-norms which satisfy the 1-Lipschitz condition are copulas. Some families
of t-norms are known as families of copulas under different names}

\signal{
    Lukasiewicz logic algebraic structure and formal implications thanks to the residual property.\\

    Gödel Logic uses 
    $a \otimes b = \min(a, b)$, which tends to produce conservative scores dominated by the weakest criterion. While useful for risk-averse scenarios, it fails to differentiate alternatives when all criteria are partially satisfied.

    Product Logic employs 
    $a \otimes b = a \cdot b$, amplifying the impact of low-scoring criteria. This can lead to premature elimination of alternatives with one poor attribute.

    Łukasiewicz Logic's additive t-norm balances compensation between criteria, allowing alternatives to offset weaknesses in one dimension with strengths in others—a critical feature for complex trade-off analysis.
}








\subsection{Approximate Reasoning and Pavelka-Style Completeness}

Classical deductive systems typically deal with absolute truth: premises are assumed to be fully true (truth value 1), and sound inference rules guarantee that conclusions are also fully true. However, in many real-world scenarios and in the spirit of fuzzy logic, we often reason with information that is only partially true or true to a certain degree. This leads to the field of **approximate reasoning**, where we are interested in the degree to which a conclusion follows from premises that themselves have degrees of truth.

A significant contribution to formalizing reasoning with degrees of truth was made by Jan Pavelka in the 1970s, particularly for logics related to Łukasiewicz logic. His work was later refined and integrated into the broader t-norm based framework by Hájek. The core idea is to extend a fuzzy logical system (like Łukasiewicz logic, denoted L) with rational truth constants $\bar{r}$ for every rational $r \in [0,1] \cap \mathbb{Q}$. A formula $(\bar{r} \rightarrow \phi)$ can then be interpreted as stating "$\phi$ is true to a degree at least $r$".

This framework allows for the definition of two key concepts for a given theory $T$ (a set of axioms, each potentially associated with a truth degree or assumed to be fully true):

\begin{itemize}
    \item \textbf{Truth Degree of $\phi$ over $T$}: $||\phi||_T = \inf\{e(\phi) \mid e \text{ is a model of } T\}$. This is the lowest degree to which $\phi$ is true in all models that satisfy the theory $T$.
    \item \textbf{Provability Degree of $\phi$ over $T$}: $|\phi|_T = \sup\{r \mid T \vdash (\bar{r} \rightarrow \phi)\}$. This is the highest degree $r$ such that it is provable from $T$ that $\phi$ is true to degree at least $r$.
\end{itemize}

Pavelka's completeness theorem for Rational Pavelka Logic (RPL), which is Łukasiewicz logic extended with rational truth-value constants and appropriate "bookkeeping" axioms for these constants, establishes a fundamental connection between these semantic and syntactic degrees.

\begin{theorem}[Pavelka-style Completeness for RPL]
For any theory $T$ in Rational Pavelka Logic and any formula $\phi$:
\[ |\phi|_T = ||\phi||_T \]
\end{theorem}

This remarkable result states that the highest degree to which a formula is provable from a theory $T$ is precisely the lowest degree to which it is true in all models of $T$. It provides a formal calculus for "approximate reasoning" where conclusions can be derived with associated degrees of truth based on the degrees of truth of the premises. For instance, if we have premises $A_1, \dots, A_k$ with associated truth degrees $r_1, \dots, r_k$, and we can syntactically derive a conclusion $C$ with an associated provability degree $s$ that depends on $r_1, \dots, r_k$, Pavelka's theorem ensures that this syntactically derived degree $s$ is the "best possible" semantic guarantee for the truth of $C$. While the original RPL is specific to Łukasiewicz logic due to its reliance on the properties of the Łukasiewicz t-norm and its residuum, the general concept of graded provability and its relation to graded truth is a cornerstone of advanced fuzzy logic.