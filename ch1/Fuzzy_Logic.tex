\subsection{Fuzzy implications}
S-implications
R-implications
t-norm implications
\signal{$(\fuzzy{X},\cup , \cap , \lnot)$ is a complete, completely distributive, lattice with an involution. This extends the boolean algebra.}



% \signal{
%     Lukasiewicz logic algebraic structure and formal implications thanks to the residual property.\\

%     Gödel Logic uses 
% a
% ⊗
% b
% =
% min

% (
% a
% ,
% b
% )
% a⊗b=min(a,b), which tends to produce conservative scores dominated by the weakest criterion. While useful for risk-averse scenarios, it fails to differentiate alternatives when all criteria are partially satisfied.

% Product Logic employs 
% a
% ⊗
% b
% =
% a
% ⋅
% b
% a⊗b=a⋅b, amplifying the impact of low-scoring criteria. This can lead to premature elimination of alternatives with one poor attribute.

% Łukasiewicz Logic's additive t-norm balances compensation between criteria, allowing alternatives to offset weaknesses in one dimension with strengths in others—a critical feature for complex trade-off analysis
% }