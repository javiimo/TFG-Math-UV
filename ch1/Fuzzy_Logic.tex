\subsection{Fuzzy implications}
S-implications
R-implications
t-norm implications
\signal{$(\fuzzy{X},\cup , \cap , \lnot)$ is a complete, completely distributive, lattice with an involution. This extends the boolean algebra.}

\signal{Fuzzy description logics explained %https://www.umbertostraccia.it/cs/download/papers/KES09/KES09.pdf

The Lukasiewicz t-conorm is closely related to the basic binary operation of multi-valued
algebras. Additionally, t-norms and t-conorms form examples of aggregation operators. They
play a significant role in the axiomatic definition of the concept of triangular norm-based measure
and, in particular, of the concept of probability of fuzzy events; the Frank family of t-norms and
t-conorms plays a particular role [6]. 

%https://cake.fiu.edu/Publications/Ngan+al-18-LC.Logic_Connectives_of_Complex_Fuzzy_Sets_ROMJIST_downloaded.pdf 

It should be mentioned that t-norms overlap with copulas [3, 24]: commutative associative
copulas are t-norms; t-norms which satisfy the 1-Lipschitz condition are copulas. Some families
of t-norms are known as families of copulas under different names

A new t-norm was introduced and applied in Active Learning Method (ALM) by Kiaei et
al. [20]. The original operators of ALM were presented, along with the Ink Drop Spread and
the Center of Gravity operators, and two basic morphological operators. The obtained results
show that new operators have overcome several of the disadvantages of the original operators.
The new operators were applied well in ALM. In that paper, aggregations of fuzzy relations
using aggregation functions have been considered. This has been performed by determining
certain conditions expressed in computational formulas and deploying t-norms and t-conorms
properties. Recently, T-operators have been used to combine criteria in MCDM.}

\signal{
    Lukasiewicz logic algebraic structure and formal implications thanks to the residual property.\\

    Gödel Logic uses 
    $a \otimes b = \min(a, b)$, which tends to produce conservative scores dominated by the weakest criterion. While useful for risk-averse scenarios, it fails to differentiate alternatives when all criteria are partially satisfied.

    Product Logic employs 
    $a \otimes b = a \cdot b$, amplifying the impact of low-scoring criteria. This can lead to premature elimination of alternatives with one poor attribute.

    Łukasiewicz Logic's additive t-norm balances compensation between criteria, allowing alternatives to offset weaknesses in one dimension with strengths in others—a critical feature for complex trade-off analysis.
}