\begin{notation}[label={not:OpsFS}]{Notation for variables in the current section}
    In this section, we use the variable \( x \) to represent time and \( y \) to represent distance.
  \end{notation}

\subsection{Union, Intersection and Complement of Fuzzy Sets}
\begin{definition}[Triangular Norm]
    A mapping $T:[0,1]\times [0,1] \longrightarrow [0,1]$ that satisfies:
    \begin{enumerate}[(i)]\setlength{\itemindent}{2em}
      \item \textbf{Symmetricity:} $T(x,y) = T(y,x) \quad \oldforall x,y \in [0,1]$
      \item \textbf{Associativity:} $T(x,T(y,z)) = T(T(x,y),z) \quad \oldforall x,y,z \in [0,1]$
      \item \textbf{Monotonicity:} $T(x,y) \leq T(x',y') \quad \textnormal{if }x\leq x' \textnormal{ and } y\leq y' \quad \oldforall x,y,x',y' \in [0,1]$
      \item \textbf{One Identity:} $T(x,1) = T(1,x) = x \quad \oldforall x \in [0,1]$
    \end{enumerate}
    is called a triangular norm or t-norm.
\end{definition}

\begin{definition}[Triangular Conorm]
  A mapping $S:[0,1]\times [0,1] \longrightarrow [0,1]$ that satisfies:
  \begin{enumerate}[(i)]\setlength{\itemindent}{2em}
    \item \textbf{Symmetricity:} $S(x,y) = S(y,x) \quad \oldforall x,y \in [0,1]$
    \item \textbf{Associativity:} $S(x,T(y,z)) = S(S(x,y),z) \quad \oldforall x,y,z \in [0,1]$
    \item \textbf{Monotonicity:} $S(x,y) \leq S(x',y') \quad \textnormal{if }x\leq x' \textnormal{ and } y\leq y' \quad \oldforall x,y,x',y' \in [0,1]$
    \item \textbf{Zero Identity:} $S(x,0) = S(0,x) = x \quad \oldforall x \in [0,1]$
  \end{enumerate}
  is called a triangular conorm or t-conorm.
    
\end{definition}

\begin{definition}[Complement]
    The complement of a fuzzy set $A$ is another fuzzy set with membership function given by $^\lnot A(x) \coleq 1 - A(x)$
\end{definition}
\signal{Menciona lo de que no se cumplen las leyes de De Morgan en general.}

\begin{definition}[t-norm based union]
  
\end{definition}

\begin{definition}[t-conorm based intersection]
  
\end{definition}

\begin{proposition}[Relationship between t-norm and t-conorm]
  Given a t-norm $T$, the t-conorm $S(a,b)\coleq 1 - T(1-a, 1-b)$, then the union and intersection defined by that pair satisfy the De Morgan's Laws.
\end{proposition}
\begin{remark}
  It is easy to see that the previous relation is equivalent to $T(a,b) = 1-S(1-a, 1-b)$ which can be obtained simply by substituting $a'=1-a$ and $b'=1-b$, i.e., working with the complementary fuzzy sets.
\end{remark}

\begin{proof}
  Let $x\in X$, $A$, $B$ be fuzzy sets over $X$ with $a \coleq A(x)$ and $b \coleq B(x)$\\

  $\quad \boxed{\text{not}(A \text{ or } B) = (\text{not } A) \text{ and } (\text{not } B)}$\\
  $$\lnot S(a,b) = T(\lnot a, \lnot b)$$
  $$1 - S(a,b) = T(1-a, 1-b)$$
  $$S(a,b) = 1 - T(1-a, 1-b)$$
  $\quad \boxed{\text{not}(A \text{ and } B) = (\text{not } A) \text{ or } (\text{not } B)}$\\
  $$\lnot T(a,b) = S(\lnot a, \lnot b)$$
  $$1 - T(a,b) = S(1-a, 1-b)$$
  $$T(a,b) = 1 - S(1-a, 1-b)$$

\end{proof}

\begin{definition}[Archimedean t-norm]
  
\end{definition}
\begin{proposition}[Characterization of archimedean t-norms]
  
\end{proposition}

\signal{
\begin{definition}[Nilpotent t-norm]
  
\end{definition}}

\begin{definition}[Weaker t-norm]
  Partial order relation
\end{definition}
\signal{The weaker the t-norm, the stronger the associated s-norm?}

\subsection{Fuzzy Relations}
\begin{definition*}[Classical n-ary relation]
    
\end{definition*}

\begin{definition}[Fuzzy Binary Relation]
    
\end{definition}


\begin{definition*}[Classical Projection]

\end{definition*}

\begin{definition}[Projection of Fuzzy Binary Relations]

\end{definition}

\begin{definition}[Cartesian Product]

\end{definition}

\subsubsection*{Composition of Fuzzy Sets}
\begin{notation}[label={not:compositionFS}]{Notation}
    Although Fuzzy Relations are Fuzzy Sets as well, we will use the name distinction to denote whether the domain is a cartesian product or not.
  \end{notation}
\begin{definition}[Composition of 2 Fuzzy Relations]
    
\end{definition}

\begin{definition}[Composition of a Fuzzy Set and a Fuzzy Relation]
    
\end{definition}