\begin{notation}[label={not:OpsFS}]{Notation for variables in the current section}
    In this section, we use the variable \( x \) to represent time and \( y \) to represent distance.
  \end{notation}

\subsection{Union, Intersection and Complement of Fuzzy Sets}
\begin{definition}[Triangular Norm]
    
\end{definition}

\begin{definition}[Triangular Conorm]
    
\end{definition}


\begin{definition}[Complement]
    
\end{definition}
\signal{Menciona lo de que no se cumplen las leyes de De Morgan en general.}

\subsection{Fuzzy Relations}
\begin{definition*}[Classical n-ary relation]
    
\end{definition*}

\begin{definition}[Fuzzy Binary Relation]
    
\end{definition}


\begin{definition*}[Classical Projection]

\end{definition*}

\begin{definition}[Projection of Fuzzy Binary Relations]

\end{definition}

\begin{definition}[Cartesian Product]

\end{definition}

\subsubsection*{Composition of Fuzzy Sets}
\begin{notation}[label={not:compositionFS}]{Notation}
    Although Fuzzy Relations are Fuzzy Sets as well, we will use the name distinction to denote whether the domain is a cartesian product or not.
  \end{notation}
\begin{definition}[Composition of 2 Fuzzy Relations]
    
\end{definition}

\begin{definition}[Composition of a Fuzzy Set and a Fuzzy Relation]
    
\end{definition}