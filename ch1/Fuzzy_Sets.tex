\signal{Decir cómo a partir de ese concepto de partial membership llegamos a esto y cómo es una generalización de boolean membership de classical sets.}

\begin{definition}[Fuzzy Set]
    Let $X\neq\emptyset$ a set. Then we define a \textbf{fuzzy set A in X}, i.e., $A \in \fuzzy{X}$ as:
    \[A=\{(x,\mu_A(x))\mid x\in X\}\]
    Where $\mu_A:X\longrightarrow [0,1]$ is the \textbf{membership function} and $X$ is the \textbf{domain} of the fuzzy set.
\end{definition}

\begin{remark}
    We may use the notation \( A(x) \equiv \mu_A(x) \) interchangeably since both the fuzzy set and the membership function uniquely identify each other.
\end{remark}

\begin{definition}[Support]
    Let $A \in \fuzzy{X}$. The set of non zero membership value elements is called the support:
    \[\textnormal{Supp}(A)=\{x\in X \mid A(x)>0\}\]
\end{definition}

\signal{
\begin{definition}[Normal Fuzzy Set]
    A fuzzy set is called \textbf{normal} if there exists $x\in X$ such that $A(x)=1$. Otherwise it is called \textbf{subnormal}.
\end{definition}
}
\begin{definition}[$\alpha$-cut]
    Let $\alpha \in [0,1]$, the $\alpha$-cut (also called $\alpha$-cut) of a fuzzy set \( A \in \fuzzy{X}\) is:
    \[
    [A]^\alpha =
    \begin{cases}
    \{x \in X \mid A(x)\geq \alpha\} & \text{if } \alpha > 0, \\
    \textnormal{cl}(\textnormal{Supp}(A)) & \text{if } \alpha = 0.
    \end{cases}
    \]
    where \textit{cl} denotes the closure.
\end{definition}

\signal{
    \begin{definition}[Extremes of the support]
        
    \end{definition}
}


\begin{definition}[Fuzzy Subset]
    Given the fuzzy sets $A, B \in \fuzzy{X}$ we say $A$ is a fuzzy subset of $B$ (and write $A \subseteq B$) if and only if $A(x)
    \leq B(x) \forall x \in X$.

    Analogously, $A$ and $B$ are equal if and only if $A(x)=B(x) \forall x \in X$, i.e., each of them is a subset of the other.
\end{definition}

\begin{example}
    Here are some examples of common fuzzy sets:
    \begin{itemize}
        \item \textbf{Empty Fuzzy Set in $X$:} such that $\emptyset(x)=0 \forall x \in X$.
        \item \textbf{Universal Fuzzy Set in $X$:} such that $X(x)=1  \forall x \in X$.
        \item \textbf{Fuzzy Point in $X$:} such that $P(x_0)=1 \land A(x)=0 \forall x \in X-\{x_0\}$
        \item \textbf{Fuzzy Number:} Usually defined as a fuzzy set in $\mathbb{R}$ with some desirable properties. Will be covered in Section \ref{sec:fuzzy_numbers}.
    \end{itemize}
\end{example}
