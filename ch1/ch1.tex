\chapter{Fuzzy Set Theory}
This chapter draws upon foundational concepts and theoretical frameworks presented in \cite{FULLER1} and \cite{FULLER2}. 
% \section*{Motivation}
% Fuzzy sets were introduced by Zadeh in 1965 \cite{Zadeh1965}. 
% \signal{the example of tall people, we want to model imprecision and why probability theory is not well suited for this. También pensar si voy a diferenciar entre incertidumbre epistemológica o aleatoria, o si eso lo meto cuando hable de possibility.}

\section{Fuzzy Sets}
Fuzzy sets were introduced by Zadeh in 1965, following the idea of generalizing sets that was presented in section \ref{sec:sorites}:

\say{More often than not, the classes of objects encountered in the real physical world do not have precisely defined criteria of membership. [...]Clearly, the ''class of all real numbers which are much greater than 1," or ''the class of beautiful women," or ''the class of tall men," do not constitute classes or sets in the usual mathematical sense of these terms. [...]Yet, the fact remains that such imprecisely defined ''classes" play an important role in human thinking, particularly in the domains of pattern recognition, communication of information, and abstraction.}\cite{Zadeh1965}\\

The idea he proposed for representing those classes is using a continuum of grades of membership. While classical (also called crisp or boolean) sets use a boolean membership function $\chi_A:X\rightarrow\{0,1\}$ that assigns either 0 or 1 to each element, fuzzy sets generalize this by using a membership function $\mu_A:X\rightarrow[0,1]$ that can assign any value between 0 and 1. 

\begin{remark}
    The membership degrees represent how compatible an object is with a set. Since membership degrees are real numbers in $[0,1]$, they are totally ordered. However, this does not imply a total ordering of objects, since two different objects may have the same membership degree. The set of objects can be totally ordered by considering equivalence classes of objects with the same membership degree.
\end{remark}

Let us take a closer look at Zadeh's example of the set of ''tall men" to answer a fundamental question: where do these membership values come from?

Most people would agree that a person with a height of 1.50m is not tall, so we can anchor the function with $\mu_{tall}(1.50) = 0$. Conversely, a person 2.00m tall is generally considered tall, so $\mu_{tall}(2.00) = 1$. But how should we define the values in between? The choice is not unique, and this non-uniqueness is not a flaw, but a feature.

However, not all choices are valid. For a membership function to be a reasonable model of ''tallness", it must be consistent with our understanding of the concept. For example, it would be illogical to define a 1.70m person as ''more tall" than a 1.80m person. This means the function must be non-decreasing. Furthermore, a classical boolean definition fails to capture the gradual nature of ''tallness", even if it is not \emph{wrong}, it lacks the expressive power for more a nuanced definition. Figure \ref{fig:tall_definitions} illustrates these points.

\begin{figure}[!ht]
    \centering
    \includegraphics[width=\textwidth]{ch1/figures/Fuzzy_tall.png}
    \caption{Illustrations of different membership functions for the fuzzy set ''tall". The left plot shows incorrect definitions: a crisp set is too rigid to capture the gradual nature of the concept, and a non-monotonic function is logically inconsistent since it implies a shorter person could be considered ''more tall" than a taller one. The right plot shows several valid, monotonic functions that can represent ''tall". Each reflects a different modeling choice, but all are anchored to the clear cases: membership is 0 for heights below 1.5m and 1 for heights above 2.0m.}
    \label{fig:tall_definitions}
\end{figure}

As shown in Figure \ref{fig:tall_correct}, there are infinitely many valid ways to define the fuzzy set ''tall". This flexibility might seem like a problem of subjectivity or a lack of rigor, but it is better understood as a \textbf{matter of definition}. In classical set theory, we don't ask whether the set of ''prime numbers" is more correct than the set of ''even numbers"; they are simply different sets defined for different purposes. 

Similarly, choosing a membership function \emph{is} the act of defining the fuzzy set. The goal is not to discover a single, universal truth for ''tallness," but to \textbf{design a fuzzy set that is useful for a specific problem}. This allows us to encode context-dependent or domain-specific expert knowledge into a precise mathematical model. The various methods for systematically creating these membership functions, a process known as \textit{fuzzification}, will be discussed in Section \ref{sec:fuzzification}.

For the remainder of this chapter, we will assume that appropriate fuzzy sets are given, and we will focus on the mathematical properties and operations that can be derived from them.


Therefore, fuzzy sets can be seen as an extension of crisp sets, since every crisp set can be modeled as a fuzzy set but not the other way around. The formal definition is as follows:

\begin{definition}[Fuzzy Set]
    Let $X\neq\emptyset$ a set. Then we define a \textbf{fuzzy set A in X}, i.e., $A \in \fuzzy{X}$ as:
    \[A=\{(x,\mu_A(x))\mid x\in X\}\]
    Where $\mu_A:X\longrightarrow [0,1]$ is the \textbf{membership function} and $X$ is the \textbf{domain} of the fuzzy set.
\end{definition}

\begin{remark}
     Both the fuzzy set and the membership function uniquely identify each other.
\end{remark}

\begin{notation}{Notation}
    We may use \( A(x) \equiv \mu_A(x) \) interchangeably.

    In general, $\chi$ will denote boolean membership functions and $\mu$, fuzzy membership functions.
\end{notation}



\begin{definition}[Support]
    Let $A \in \fuzzy{X}$. The crisp set of non zero membership value elements is called the support:
    \[\textnormal{Supp}(A)=\{x\in X \mid A(x)>0\}\]
\end{definition}

\begin{definition}[Fuzzy Subset]
    Given the fuzzy sets $A, B \in \fuzzy{X}$ we say $A$ is a fuzzy subset of $B$ (and write $A \subseteq B$) if and only if $A(x)
    \leq B(x) \forall x \in X$.

    Analogously, $A$ and $B$ are equal if and only if $A(x)=B(x) \forall x \in X$, i.e., each of them is a subset of the other.
\end{definition}

\begin{example}
    Here are some examples of common fuzzy sets:
    \begin{itemize}
        \item \textbf{Empty Fuzzy Set in $X$:} such that $\emptyset(x)=0 \forall x \in X$.
        \item \textbf{Universal Fuzzy Set in $X$:} such that $X(x)=1  \forall x \in X$.
        \item \textbf{Fuzzy Point in $X$:} such that $P(x_0)=1 \land A(x)=0 \forall x \in X-\{x_0\}$
        \item \textbf{Fuzzy Number:} Usually defined as a fuzzy set in $\mathbb{R}$ with some desirable properties. Will be covered in Section \ref{sec:fuzzy_numbers}.
    \end{itemize}
\end{example}

\section{Union, Intersection and Complement of Fuzzy Sets}
% \begin{notation}[label={not:OpsFS}]{Notation for variables in the current section}
%     In this section, we use the variable \( x \) to represent time and \( y \) to represent distance.
%   \end{notation}

The notions of union and intersection in fuzzy sets were first introduced by Zadeh \cite{Zadeh1965} using the operations $\max\{A(x),B(x)\}$ and $\min\{A(x),B(x)\}$ respectively. These can be intuitively interpreted as follows: the union is the \textit{smallest} fuzzy set (having lowest membership values) that contains both sets, while the intersection is the \textit{biggest} fuzzy set (having highest membership values) that is contained by both sets.\\

However, these operations can be generalized by two broader classes of operators: triangular norms (for intersection) and triangular conorms (for union).

Triangular norms were first introduced by Karl Menger in 1942 \cite{OriginTNorms} in the context of probabilistic metric spaces. When generalizing distances between points to probability distributions (representing the probability that the distance is less than or equal to a given value), Menger defined an operation $T:\,[0,1]\times [0,1]\to [0,1]$ to preserve the triangular inequality. For points $x,y,z$ in a metric space with distance function $d(\cdot,\cdot)$, this operation satisfies:

\begin{equation}\label{eq:Ftriangle_inequality}
d(x, z) \leq d(x, y) + d(y, z) \quad \longrightarrow \quad F_{xz}(t + s) \geq T(F_{xy}(t), F_{yz}(s)) \quad \forall t,s \geq 0
\end{equation}

This inequality means that the probability of $d(x,z)$ being less than $t+s$ must be at least the t-norm of the probabilities that $d(x,y)<t$ and $d(y,z)<s$. Note the change from $\leq$ to $\geq$ in the inequality. This is consistent with \textit{larger} probabilities indicating \textit{smaller} distances are more likely.\\

Since this originated in the context of distances, the following properties were required for an operator to be a t-norm\footnote{Associativity and one identity were not originally proposed by Menger but were later added by Sklar and Schweizer \cite{Sklar1983} in their refinement of triangular norms}:

\begin{itemize}
  \item \textbf{Symmetry:} The order of combining probabilities shouldn't matter, just as intersection of sets is commutative. That is, combining probabilities for distances $(x,y)$ and $(y,z)$ should give the same result regardless of order.
  
  \item \textbf{Associativity:} When combining multiple probabilities (e.g., for paths through points $x,y,z,w$), the grouping shouldn't affect the result. This extends the triangular norm to be consistent with polygonal inequalities, similar to how nested intersections satisfy $(A \cap B) \cap C = A \cap (B \cap C)$.
  
  \item \textbf{Monotonicity:} If the probability $F_{xy}(t)$ increases, then the lower bound for $F_{xz}(t+s)$ given by $T(F_{xy}(t), F_{yz}(s))$ should not decrease. This is analogous to how adding elements in crisp sets (or increasing membership degrees in fuzzy sets) cannot reduce the intersection set.
  
  \item \textbf{One Identity:} If $F_{yz}(s) = 1$ (meaning $d(y,z) < s$ with certainty), then $F_{xz}(t+s)$ depends only on $F_{xy}(t)$. This is analogous to how intersecting with the universal set preserves the original set.
\end{itemize}


% \say{The name \textit{triangular norm} refers to the fact that in the framework of probabilistic metric spaces, t-norms and t-conorms are used to generalize triangle inequality of ordinary metric spaces.}\cite{NGAN2018}\\

Therefore, the concept of intersection (conjunction) of fuzzy sets is generally represented by a triangular norm (also called a t-norm).

\begin{definition}[Triangular Norm]
    A mapping $T:[0,1]\times [0,1] \longrightarrow [0,1]$ that satisfies:
    \begin{romanenum}
      \item \textbf{Symmetricity:} $T(x,y) = T(y,x) \quad \oldforall x,y \in [0,1]$
      \item \textbf{Associativity:} $T(x,T(y,z)) = T(T(x,y),z) \quad \oldforall x,y,z \in [0,1]$
      \item \textbf{Monotonicity:} $T(x,y) \leq T(x',y') \quad \textnormal{if }x\leq x' \textnormal{ and } y\leq y' \quad \oldforall x,y,x',y' \in [0,1]$
      \item \textbf{One Identity:} $T(x,1) = T(1,x) = x \quad \oldforall x \in [0,1]$
    \end{romanenum}
    is called a triangular norm or t-norm. Defines the \textbf{intersection} of two fuzzy sets $A$ and $B$ on $X$ by giving the membership function as $(A \cap B) (x) = T(A(x),B(x)) \forall x \in X$ 
\end{definition}

Its dual operator can also be obtained by a similar reasoning as before, but instead of considering the probability distribution of finding both points closer than a given distance, it considers the probability distribution ($G_{uv}(t) = 1 - F_{uv}(t)$) of finding them further apart than that distance. In this case, both inequalities are $\leq$ since larger probabilities indicate that greater distances are more likely.
\begin{equation}\label{eq:Gtriangle_inequality}
d(x, z) \leq d(x, y) + d(y, z) \quad \longrightarrow \quad G_{xz}(t + s) \leq S(G_{xy}(t), G_{yz}(s))
\end{equation}

The reasoning regarding the properties is entirely analogous to the previous case, with the only difference being that $F_{uv}(t) = 1 \Leftrightarrow  G_{uv}(t) = 0$, and here the identity element is zero (union with the empty set).



\begin{definition}[Triangular Conorm]
  A mapping $S:[0,1]\times [0,1] \longrightarrow [0,1]$ that satisfies:
  \begin{enumerate}[(i)]\setlength{\itemindent}{2em}
    \item \textbf{Symmetricity:} $S(x,y) = S(y,x) \quad \oldforall x,y \in [0,1]$
    \item \textbf{Associativity:} $S(x,S(y,z)) = S(S(x,y),z) \quad \oldforall x,y,z \in [0,1]$
    \item \textbf{Monotonicity:} $S(x,y) \leq S(x',y') \quad \textnormal{if }x\leq x' \textnormal{ and } y\leq y' \quad \oldforall x,y,x',y' \in [0,1]$
    \item \textbf{Zero Identity:} $S(x,0) = S(0,x) = x \quad \oldforall x \in [0,1]$
  \end{enumerate}
  is called a triangular conorm or t-conorm. Defines the \textbf{union} of two fuzzy sets $A$ and $B$ on $X$ by giving the membership function as $(A \cup  B) (x) = S(A(x),B(x)) \forall x \in X$ 
    
\end{definition}

Complement was defined by Zadeh \cite{Zadeh1965} as\footnote{This is not the only definition that satisfies the axioms of a complement and is compatible with the classical limit, but it is the simplest one and will be used in this text. Other alternatives and their axioms can be found in \cite{Sladoje2007}. In \cite{Klement2000}, this is called the standard negation $N_s$.}:

\begin{definition}[Complement]
  The complement of a fuzzy set $A\in \fuzzy{X}$ is another fuzzy set with membership function given by $^\lnot A(x) \coleq 1 - A(x) \forall x\in X$
\end{definition}

Notice that this definition of complement is consistent with the classical definition of complement but implies that an element might have \textbf{non-zero partial membership} to both a fuzzy set and its complement: Let $A$ be a fuzzy set on $X$ and $x \in X / A(x)\notin \{0,1\}$ then $\lnot A(x)= 1 - A(x) \notin \{0,1\}$.\\

One consequence of this fact is that the union of a fuzzy set and its complement is not the total set in general. Analogously, the intersection will not the empty set in general. Those two properties that hold in classical sets but might not be true in fuzzy sets, are often called the \textbf{laws of excluded middle and of non-contradiction}, respectively.\\

However, there are t-norms and t-conorms such as the ones named after \luka that do satisfy both laws\footnote{Indeed, all continuous t-norms in agreement with non-contradiction are isomorphic to \luka \cite[p.~7]{LukasiewiczNonContrad}.}. Another example is the drastic t-norm which also satisfies the law of non-contradiction, but not the law of excluded middle. See example \ref{ex:basic_tnorms} for their formal expressions. \signal{All of this, has implications for the derived logic that will be explained in section \ref{sec:fuzzy_logic}. }\\

Another important property that classical union and intersection satisfy is De Morgan's Laws. For an arbitrary pair of t-norm and t-conorm, these laws are not automatically satisfied. However, there are specific pairs that do fulfill them. To illustrate the relationship between t-norms and t-conorms that satisfy De Morgan's Laws, we can use our probabilistic metric space analogy. Reconsidering the probabilistic metrics $F,G$ introduced earlier and substituting $F = 1 - G$ into equation \ref{eq:Ftriangle_inequality}, we obtain:

\[ G_{xz}(t + s) \leq 1 - T(1 - G_{xy}(t), 1 - G_{yz}(s))\]

Comparing this with equation \ref{eq:Gtriangle_inequality}, we can derive a relationship between t-norms and t-conorms, which is formalized in the following proposition:

\begin{proposition}[Relationship between t-norm and t-conorm]
  Given a t-norm $T$, the t-conorm is $S(a,b)\coleq 1 - T(1-a, 1-b)$ if and only if the union and intersection defined by that pair satisfy the De Morgan's Laws.
\end{proposition}
\begin{remark}
  It is easy to see that the previous relation is equivalent to $T(a,b) = 1-S(1-a, 1-b)$ which can be obtained simply by substituting $a'=1-a$ and $b'=1-b$, i.e., working with the complementary fuzzy sets.
\end{remark}

\begin{proof}
  Let $x\in X$, $A$, $B$ be fuzzy sets over $X$ with $a \coleq A(x)$ and $b \coleq B(x)$\\

  $\quad \boxed{\text{not}(A \text{ or } B) = (\text{not } A) \text{ and } (\text{not } B)}$\\
  [0.5em]
  $\lnot S(a,b) = T(\lnot a, \lnot b) \iff 1 - S(a,b) = T(1-a, 1-b) \iff S(a,b) = 1 - T(1-a, 1-b)$\\

  $\quad \boxed{\text{not}(A \text{ and } B) = (\text{not } A) \text{ or } (\text{not } B)}$\\
  [0.5em]
  $\lnot T(a,b) = S(\lnot a, \lnot b) \iff 1 - T(a,b) = S(1-a, 1-b) \iff T(a,b) = 1 - S(1-a, 1-b)$

\end{proof}

\signal{Creo que además de esto de las leyes de de Morgan, también nos da que el modus ponen funciona si se cumple la propiedad esa. Además no sé si se llama residual property.}

\signal{
  Lo de archimedean sirve para el teorema 1.8.1 de \cite{FULLER2}. Y tb con la law of large numbers con LR-fuzzy numbers.}



  \subsection{Classification of T-norms}

\signal{Añadir lo de strict archimedean y divisores de zero y eso.}
\begin{definition}[Archimedean t-norm]
  A continuous t-norm that satisfies $T(x,x)<x \forall x\in ]0,1[$ is called an archimedean t-norm.
\end{definition}

\begin{proposition}[Characterization of archimedean t-norms]
  For all archimedean t-norms there exists a continuous decreasing function $f:[0,1] \longrightarrow [0,\infty[$ with $f(1)=0$ such that: 
  \[ 
  T(x,y)= f^{-1}(\min\{f(x)+f(y), f(0)\}) \text{ where } f^{-1} =
  \begin{cases}
    f^{-1}(y) & \text{if } y\in [0,f(0) ]\\
    0 & \text{otherwise}
  \end{cases}
  \text{ is a pseudo-inverse}.
  \]
\end{proposition}


\signal{
\begin{definition}[Nilpotent t-norm]
  
\end{definition}}

\begin{definition}[Weaker t-norm]
  Given $T_1, T_2$ t-norms, then $T_1$ is weaker than $T_2$ $(T_1 \leq T_2)$ if $T_1(x,y)\leq T_2(x,y)\forall x,y\in [0,1]$.\\
  In that case, it is equivalent to say $T_2$ is stronger than $T_1$ $(T_2 \geq T_1)$

\end{definition}
\begin{remark}
  This defines a partial order relation in the set of t-norms.
\end{remark}
\signal{The weaker the t-norm, the stronger the associated s-norm. Lo pongo como un remark igual.}

\signal{
There are many results like all t-norms are between the weak and the min, all t-conorms are between max and strong, or that min is the only t-norm that satisfies $T(a,a)=a$ (igual es por esto ultimo q se usa tanto. Qué implicaciones tiene que lukasiewicz no cumpla eso?)}

\signal{Tambien lo de que la T-norm e distributiva con max/sup sirve para justificar la definicion del producto cartesiano, así que esa propiedad la tendré que meter por aquí igual.}


\noindent\rule{\textwidth}{2pt}


\subsection{Basic Properties and Examples of T-norms}
The set of all t-norms can be partially ordered, which helps in their classification.
\begin{definition}[Weaker/Stronger t-norm {\cite[Def.~1.4]{Klement2000}}]
  Given two t-norms $T_1$ and $T_2$, $T_1$ is said to be \emph{weaker} than $T_2$ (denoted $T_1 \leq T_2$) if $T_1(x,y) \leq T_2(x,y) \forall x,y \in [0,1]$.
  Equivalently, $T_2$ is said to be \emph{stronger} than $T_1$.
\end{definition}
\begin{remark}
  The relation $\leq$ defines a partial order on the set of all t-norms. It's a fundamental result that for any t-norm $T$, we have $T_D \leq T \leq T_M$, where $T_D$ is the drastic product and $T_M$ is the minimum t-norm (\cite[Rem.~1.5]{Klement2000}).
\end{remark}
In order to classify t-norms further, the following algebraic properties will be needed \cite[Def.~2.1]{Klement2000}:
\begin{definition}[Idempotent Element]
Let $T$ be a t-norm. An element $a \in [0,1]$ is an \emph{idempotent element} of $T$ if $T(a,a)=a$. The elements $0$ and $1$ are always trivial idempotent elements.
\end{definition}

\begin{definition}[Nilpotent Element]
Let $T$ be a t-norm. An element $a \in ]0,1[$ is a \emph{nilpotent element} of $T$ if there exists $n \in \mathbb{N}$ such that $a_T^{(n)} = 0$, where $a_T^{(n)} = T(a, a_T^{(n-1)})$ with $a_T^{(1)}=a$.
\end{definition}

\begin{definition}[Zero Divisor]
Let $T$ be a t-norm. An element $a \in ]0,1[$ is a \emph{zero divisor} of $T$ if there exists $b \in ]0,1[$ such that $T(a,b)=0$.
\end{definition}

\begin{definition}[Archimedean T-norm]
A t-norm $T$ is \emph{Archimedean} if for each $(x,y) \in ]0,1[^2$ there is an $n \in \mathbb{N}$ with $x_T^{(n)} < y$ (\cite[Def.~2.9]{Klement2000}). 

Equivalently, a continuous t-norm $T$ is Archimedean if and only if $T(x,x) < x \forall x \in ]0,1[$ (\cite[Thm.~2.12]{Klement2000}). 
\end{definition}

\begin{example}[Basic T-norms {\cite[Ex.~1.2]{Klement2000}}]
  
  \begin{itemize}
    \item \textbf{Minimum ($T_M$):} $T_M(x, y) = \min(x, y)$.
    This is the strongest t-norm (\cite[Rem.~1.5(i)]{Klement2000}). Every element $x \in [0,1]$ is an idempotent element ($T_M(x,x)=x$). It is not Archimedean (unless interpreted on a trivial interval, as it has non-trivial idempotents). It has no zero divisors and no nilpotent elements other than 0.
    \item \textbf{Product ($T_P$):} $T_P(x, y) = x \cdot y$.
    This t-norm is strict Archimedean (\cite[Ex.~2.14(i)]{Klement2000}). It has only $0$ and $1$ as idempotent elements, no nilpotent elements (other than $0$), and no zero divisors (\cite[Ex.~2.2(i)]{Klement2000}).
    \item \textbf{Łukasiewicz ($T_L$):} $T_L(x, y) = \max(0, x + y - 1)$.
    This t-norm is nilpotent Archimedean (\cite[Ex.~2.14(i)]{Klement2000}). It has only $0$ and $1$ as idempotent elements. Every $a \in ]0,1[$ is a nilpotent element and also a zero divisor (\cite[Ex.~2.2(i)]{Klement2000}).
    \item \textbf{Drastic Product ($T_D$):} $T_D(x, y) = \begin{cases} \min(x,y) & \text{if } \max(x,y)=1 \\ 0 & \text{otherwise} \end{cases}$.
    This is the weakest t-norm (\cite[Rem.~1.5(i)]{Klement2000}). It is Archimedean since $T_D(x,x)=0 < x$ for $x \in ]0,1[$. It has only $0$ and $1$ as idempotent elements. Every $a \in ]0,1[$ is a zero divisor, and also nilpotent (since $a_D^{(2)} = T_D(a, T_D(a,a)) = T_D(a,0) = 0$ for $a<1$). It is not continuous.
  \end{itemize}
\end{example}

\subsection{Classification of Continuous T-norms}
\signal{Add an appendix with upper, lower, left, right continuity.}

Continuous t-norms form a particularly well-structured class, admitting elegant representation theorems. Their study often revolves around whether they are Archimedean.

\subsubsection{Continuous Archimedean T-norms and Generators}
The structure of continuous Archimedean t-norms is intimately linked to certain functions called generators.
\begin{definition}[Additive Generator and Pseudo-inverse]
  An \emph{additive generator} of a t-norm $T$ is a strictly decreasing function $t: [0,1] \to [0,\infty]$ which is right-continuous in $0$ and satisfies $t(1)=0$, such that $T(x,y) = t^{(-1)}(t(x) + t(y))$ for all $(x,y) \in [0,1]^2$, provided $t(x)+t(y) \in \mathrm{Ran}(t) \cup [t(0),\infty]$ (\cite[Def.~3.25, p.~70]{Klement2000}).
  The function $t^{(-1)}: [0,\infty] \to [0,1]$ is the \emph{pseudo-inverse} of $t$, defined as $t^{(-1)}(y) = \sup \{ x \in [0,1] \mid t(x) > y \}$ (adapted from \cite[Def.~3.2, p.~68 and Cor.~3.3]{Klement2000} for strictly decreasing $t$). \signal{`Ran(t)` denotes the range of $t$, i.e., the set of all values $t(x)$ for $x \in [0,1]$ \cite[p. xvii]{Klement2000}.}
\end{definition}

\begin{theorem}[Representation of Continuous Archimedean T-norms {\cite[Thm.~5.1, p.~122]{Klement2000}}]
  A t-norm $T$ is a continuous Archimedean t-norm if and only if it possesses a continuous additive generator $t: [0,1] \to [0,\infty]$. This generator is unique up to a positive multiplicative constant.
\end{theorem}
Continuous Archimedean t-norms are further categorized based on the behavior of their generator at $0$:
\begin{itemize}
    \item $T$ is \textbf{strict} if its continuous additive generator $t$ satisfies $t(0)=\infty$. This means $T(x,y)>0$ whenever $x,y > 0$. Strict t-norms are strictly monotone on $]0,1]^2$. (\cite[Cor.~3.30(i), p.~88; Def.~2.13(i), p.~42]{Klement2000}).
    \item $T$ is \textbf{nilpotent} if its continuous additive generator $t$ satisfies $t(0)<\infty$. This implies that for any $x,y \in ]0,1[$, there exists $n$ such that $T(x, \dots, x)$ ($n$ times) is $0$. Every $a \in ]0,1[$ is a nilpotent element. (\cite[Cor.~3.30(ii), p.~88; Def.~2.13(ii), p.~42]{Klement2000}).
\end{itemize}

\subsubsection{Isomorphism of Continuous Archimedean T-norms}
The concept of isomorphism reveals a deep structural similarity among t-norms within these classes.
\begin{definition}[Isomorphic T-norms {\cite[Def.~2.27, p.~51; Prop.~2.28(iv), p.~52]{Klement2000}}]
  Two t-norms $T_1$ and $T_2$ are \emph{isomorphic} if there exists a strictly increasing bijection $\varphi: [0,1] \to [0,1]$ (an automorphism of the unit interval) such that $T_2(x,y) = \varphi^{-1}(T_1(\varphi(x), \varphi(y)))$ for all $x,y \in [0,1]$.
\end{definition}
Isomorphic t-norms share the same algebraic structure, merely operating on rescaled inputs and outputs via $\varphi$. A fundamental result is:
\begin{proposition}[{\cite[Cor.~5.7, p.~125, referring to Prop.~5.9 and Prop.~5.10]{Klement2000}}]
  \begin{enumerate}
      \item Every strict t-norm is isomorphic to the Product t-norm $T_P$.
      \item Every nilpotent t-norm is isomorphic to the Łukasiewicz t-norm $T_L$.
  \end{enumerate}
\end{proposition}
This implies that, up to isomorphism, there are only two distinct types of continuous Archimedean t-norms: the product type and the Łukasiewicz type.

\subsubsection{General Continuous T-norms and Ordinal Sums}
Continuous t-norms that are not Archimedean must have non-trivial idempotent elements. These are constructed using ordinal sums.
\begin{definition}[Ordinal Sum of T-norms {\cite[Def.~3.44, p.~97]{Klement2000}}]
Let $(T_\alpha)_{\alpha \in A}$ be a family of t-norms and $(]a_\alpha, e_\alpha[)_{\alpha \in A}$ be a family of non-empty, pairwise disjoint open subintervals of $[0,1]$. The t-norm $T$ defined by
\[
T(x,y) =
\begin{cases}
  a_\alpha + (e_\alpha - a_\alpha) \cdot T_\alpha \left( \frac{x-a_\alpha}{e_\alpha - a_\alpha}, \frac{y-a_\alpha}{e_\alpha - a_\alpha} \right) & \text{if } (x,y) \in [a_\alpha, e_\alpha]^2 \text{ for some } \alpha \in A \\
  \min(x,y) & \text{otherwise}
\end{cases}
\]
is called the \emph{ordinal sum} of the summands $(a_\alpha, e_\alpha, T_\alpha)$, $\alpha \in A$.
\end{definition}
Intuitively, an ordinal sum "glues" copies of the t-norms $T_\alpha$ (scaled to fit the intervals $[a_\alpha, e_\alpha]$) onto the diagonal of the unit square. Outside these "active" regions, the t-norm behaves like the minimum $T_M$. The endpoints $a_\alpha, e_\alpha$ become idempotent elements of $T$.

\begin{theorem}[Representation of Continuous T-norms {\cite[Thm.~5.11, p.~140]{Klement2000}}]
  A function $T: [0,1]^2 \to [0,1]$ is a continuous t-norm if and only if $T$ is uniquely representable as an ordinal sum of continuous Archimedean t-norms.
\end{theorem}
\begin{remark}
  A continuous t-norm is \emph{ordinally irreducible} if its only ordinal sum representation is $T = ((0,1,T))$. For continuous t-norms, being ordinally irreducible is equivalent to being Archimedean (\cite[Prop.~3.53, p.~99 and context]{Klement2000}). Thus, the "building blocks" in the ordinal sum representation are precisely the continuous Archimedean t-norms (isomorphic to $T_P$ or $T_L$). If the family of subintervals is empty, the ordinal sum is defined as $T_M$.
\end{remark}

\subsection{Further Examples: Parametric Families of T-norms}
Beyond the four basic t-norms, several parametric families offer a spectrum of behaviors and are widely used. Their construction often relies on generators.
\begin{definition}[Multiplicative Generator {\cite[Def.~3.36, p.~91]{Klement2000}}]
  A \emph{multiplicative generator} $\theta: [0,1] \to [0,1]$ of a t-norm $T$ is a strictly increasing function, right-continuous in $0$, with $\theta(1)=1$, such that $T(x,y) = \theta^{(-1)}(\theta(x) \cdot \theta(y))$, assuming $\theta(x)\cdot\theta(y)$ is in a suitable range. \signal{This is a simplified statement; the book's definition includes range conditions.}
\end{definition}
\begin{remark}[Duality of Generators {\cite[Rem.~3.34, p.~90]{Klement2000}}]
  If $t(x)$ is an additive generator, then $\theta(x) = e^{-c \cdot t(x)}$ (for some $c>0$) is a multiplicative generator, and if $\theta(x)$ is a multiplicative generator, then $t(x) = -c \cdot \log(\theta(x))$ is an additive generator.
\end{remark}

\begin{example}[Parametric Families of T-norms (selected from {\cite[Chapter 4]{Klement2000}})]
  \begin{itemize}
    \item \textbf{Schweizer-Sklar T-norms ($T_\lambda^{SS}$), $\lambda \in [-\infty, \infty]$:}
    $T_\lambda^{SS}(x,y) = (\max(0, x^\lambda + y^\lambda - 1))^{1/\lambda}$.
    Includes $T_M (\lambda=-\infty)$, $T_P (\lambda \to 0)$, $T_L (\lambda=1)$, $T_D (\lambda=\infty)$. Additive generator $t_\lambda^{SS}(x) = \frac{1-x^\lambda}{\lambda}$ for $\lambda \neq 0$.
    \item \textbf{Frank T-norms ($T_\lambda^F$), $\lambda \in [0, \infty]$:}
    $T_\lambda^F(x,y) = \log_\lambda \left(1 + \frac{(\lambda^x-1)(\lambda^y-1)}{\lambda-1}\right)$ for $\lambda \in ]0,\infty[, \lambda \neq 1$.
    Includes $T_M (\lambda=0)$, $T_P (\lambda=1)$, $T_L (\lambda=\infty)$. All $T_\lambda^F$ for $\lambda \in ]0,\infty]$ are continuous Archimedean. Additive generator $t_\lambda^F(x) = -\log \frac{\lambda^x-1}{\lambda-1}$.
    \item \textbf{Yager T-norms ($T_p^Y$), $p \in [0, \infty]$:} \signal{The book uses $\lambda$ for Yager, but $p$ is more common in literature to avoid clash with Frank's parameter.}
    $T_p^Y(x,y) = \max(0, 1 - ((1-x)^p + (1-y)^p)^{1/p})$ for $p \in ]0,\infty[$.
    Includes $T_D (p \to 0)$, $T_L (p=1)$, $T_M (p=\infty)$. All $T_p^Y$ for $p \in ]0,\infty[$ are continuous nilpotent Archimedean. Additive generator $t_p^Y(x) = (1-x)^p$.
  \end{itemize}
  \signal{Many other families exist, like Hamacher, Dombi, Aczél-Alsina, Sugeno-Weber, Mayor-Torrens. Refer to \cite[Table 4.1, p.~119]{Klement2000} and Appendix A therein for a comprehensive list and properties.}
\end{example}

\subsection{Non-Continuous T-norms and Left-Continuity}
While continuous t-norms are well-classified, not all t-norms are continuous.
The \textbf{drastic product $T_D$} is a key example of a non-continuous t-norm. It is Archimedean. It is upper semicontinuous, implying right-continuity in each variable when the other is fixed (\cite[Rem.~1.21(i), Prop.~1.22]{Klement2000}). However, it is not left-continuous (e.g., at $(1,y)$ for $y<1$).

The \textbf{nilpotent minimum $T^{nM}$} (\cite[Rem.~1.21(i), p.~16]{Klement2000}) is defined as:
  \[
  T^{nM}(x,y) =
  \begin{cases}
    0 & \text{if } x+y \leq 1 \\
    \min(x,y) & \text{otherwise.}
  \end{cases}
  \]
This t-norm is lower semicontinuous, which for monotone functions implies it is left-continuous in each variable (\cite[Prop.~1.22, p.~17]{Klement2000}). It is not continuous (specifically, not right-continuous at points on the line $x+y=1$ when approached from $x+y>1$).

The \textbf{Krause t-norm $T^K$} (\cite[App.~B.1, Thm.~B.1]{Klement2000}) is a more complex example. It is constructed using the Cantor set and Farey series. It is stated to be "neither left- nor right-continuous, but has a continuous diagonal section." This highlights that t-norms can exhibit quite irregular continuity behavior.

\begin{remark}[Importance of Left-Continuity]
For non-continuous t-norms, the property of \emph{left-continuity} (in each variable) is often a desirable, or even required, condition in certain applications, particularly in fuzzy logic. For instance, in residuum-based logics, if a t-norm $T$ is left-continuous, its corresponding residuated implication $I(x,y) = \sup\{z \in [0,1] \mid T(x,z) \le y\}$ exhibits well-behaved properties. Specifically, a commutative, integral lattice-ordered monoid based on $T$ is residuated if and only if $T$ is left-continuous (\cite[Prop.~2.47, p.~63]{Klement2000}). This ensures that the implication adequately captures deductive reasoning. While the intuitive notion that "a microscopic decrease of the truth degree of a conjunct should not macroscopically decrease the truth degree of the conjunction" points towards continuity, left-continuity is a weaker but often sufficient condition for preserving logical coherence in such frameworks.
\end{remark}


\noindent\rule{\textwidth}{2pt}


\subsection{Classification of T-norms}
\signal{Hablar de los generadores de las t-norms (familias) y propiedades. Diagrama de qué t-norms son weaker que otras. Cuáles son continuas y cuales no (drastic por ejemplo)}

\signal{Normalmente se pide que sean left continuous (it is sufficient in either argument \cite{Esteva2001MonoidalTB}), which expresses the assumption that a microscopic decrease of the truth degree of a conjunct should not macroscopically decrease the truth degree of conjunction.}

The set of all t-norms can be partially ordered, and understanding their properties allows for a useful classification, particularly for continuous t-norms.

\begin{definition}[Weaker/Stronger t-norm {\citep[Definition 1.4, p.~21]{Klement2000}}]
  Given two t-norms $T_1$ and $T_2$, $T_1$ is said to be \emph{weaker} than $T_2$ (denoted $T_1 \leq T_2$) if $T_1(x,y) \leq T_2(x,y)$ for all $x,y \in [0,1]$.
  Equivalently, $T_2$ is said to be \emph{stronger} than $T_1$.
\end{definition}

\begin{remark}
  The relation $\leq$ defines a partial order on the set of all t-norms.
  It's a fundamental result that for any t-norm $T$, we have $T_D \leq T \leq T_M$, where $T_D$ is the drastic product and $T_M$ is the minimum t-norm (\citep[Remark 1.5(i), p.~21]{Klement2000}).
\end{remark}

\begin{remark}
  Duality with t-conorms: If $T_1 \leq T_2$ are t-norms, and $S_1, S_2$ are their respective dual t-conorms (obtained via a strong negation $N$, typically $N(x)=1-x$), then $S_1 \geq S_2$. Thus, a weaker t-norm corresponds to a stronger dual t-conorm.
\end{remark}

To classify t-norms further, several algebraic properties are essential:

\begin{definition}[Algebraic Properties of T-norms {\citep[Definition 2.1, Definition 2.9]{Klement2000}}]
Let $T$ be a t-norm.
\begin{enumerate}
    \item An element $a \in [0,1]$ is an \emph{idempotent element} of $T$ if $T(a,a)=a$. The elements $0$ and $1$ are always trivial idempotent elements.
    \item An element $a \in ]0,1[$ is a \emph{nilpotent element} of $T$ if there exists $n \in \mathbb{N}$ such that $a_T^{(n)} = 0$, where $a_T^{(n)}$ denotes the $n$-th power of $a$ with respect to $T$.
    \item An element $a \in ]0,1[$ is a \emph{zero divisor} of $T$ if there exists $b \in ]0,1[$ such that $T(a,b)=0$.
    \item $T$ is \emph{Archimedean} if for each $(x,y) \in ]0,1[^2$ there is an $n \in \mathbb{N}$ with $x_T^{(n)} < y$. Equivalently, a t-norm $T$ is Archimedean if and only if it has only trivial idempotent elements (0 and 1) and satisfies a further condition related to limits, or more simply for continuous t-norms, if $T(x,x) < x$ for all $x \in ]0,1[$ (\citep[Theorem 2.12 and Theorem 5.1]{Klement2000}).
\end{enumerate}
\end{definition}

\begin{proposition}[{\citep[Proposition 1.9(i), p.~24]{Klement2000}}]
  The minimum t-norm $T_M(x,y) = \min(x,y)$ is the only t-norm $T$ satisfying $T(x,x)=x$ for all $x \in [0,1]$.
\end{proposition}

\paragraph{Continuous T-norms}
Continuous t-norms admit a very structured classification.

\begin{theorem}[Generator Theorem for Continuous Archimedean T-norms {\citep[Theorem 5.1, p.~122]{Klement2000}}]
  A function $T: [0,1]^2 \to [0,1]$ is a continuous Archimedean t-norm if and only if it has a \emph{continuous additive generator}. That is, there exists a continuous, strictly decreasing function $t: [0,1] \to [0,\infty]$ with $t(1)=0$, such that for all $(x,y) \in [0,1]^2$:
  \[
  T(x,y) = t^{(-1)}(t(x) + t(y))
  \]
  where $t^{(-1)}: [0,\infty] \to [0,1]$ is the pseudo-inverse of $t$, defined as
  \[
  t^{(-1)}(z) =
  \begin{cases}
    t^{-1}(z) & \text{if } z \in [0, t(0)] \\
    0 & \text{if } z \in ]t(0), \infty]
  \end{cases}
  \]
  (If $t(0)=\infty$, then $t^{(-1)} = t^{-1}$). The generator $t$ is unique up to a positive multiplicative constant.
\end{theorem}

Continuous Archimedean t-norms are further divided into two main classes:

\begin{definition}[Strict and Nilpotent T-norms {\citep[Definition 2.13, p.~42]{Klement2000}}]
  \begin{enumerate}
      \item A t-norm $T$ is called \emph{strict} if it is continuous and strictly monotone (i.e., $T(x,y) < T(x,z)$ whenever $x>0$ and $y<z$).
      \item A t-norm $T$ is called \emph{nilpotent} if it is continuous and if each $a \in ]0,1[$ is a nilpotent element of $T$.
  \end{enumerate}
\end{definition}

\begin{corollary}[{\citep[Corollary 3.30, p.~88]{Klement2000}}]
  Let $T$ be a continuous Archimedean t-norm with a continuous additive generator $t$.
  \begin{enumerate}
      \item $T$ is strict if and only if $t(0) = \infty$.
      \item $T$ is nilpotent if and only if $t(0) < \infty$.
  \end{enumerate}
\end{corollary}

\begin{proposition}[Isomorphism of Continuous Archimedean T-norms {\citep[Corollary 5.7, p.~125; Proposition 5.9, p.~126; Proposition 5.10, p.~127]{Klement2000}}]
  \begin{enumerate}
      \item Every strict t-norm is isomorphic to the product t-norm $T_P(x,y) = xy$.
      \item Every nilpotent t-norm is isomorphic to the Łukasiewicz t-norm $T_L(x,y) = \max(0, x+y-1)$.
  \end{enumerate}
  This means that any continuous Archimedean t-norm is isomorphic to either $T_P$ or $T_L$.
\end{proposition}

\paragraph{Ordinal Sum Representation}
Not all continuous t-norms are Archimedean. Those that are not (i.e., possess non-trivial idempotents) can be constructed from Archimedean t-norms using the concept of an ordinal sum.

\begin{definition}[Ordinal Sum of T-norms {\citep[Definition 3.44, p.~97]{Klement2000}}]
Let $(T_\alpha)_{\alpha \in A}$ be a family of t-norms and $(]a_\alpha, e_\alpha[)_{\alpha \in A}$ be a family of non-empty, pairwise disjoint open subintervals of $[0,1]$. The t-norm $T$ defined by
\[
T(x,y) =
\begin{cases}
  a_\alpha + (e_\alpha - a_\alpha) \cdot T_\alpha \left( \frac{x-a_\alpha}{e_\alpha - a_\alpha}, \frac{y-a_\alpha}{e_\alpha - a_\alpha} \right) & \text{if } (x,y) \in [a_\alpha, e_\alpha]^2 \text{ for some } \alpha \in A \\
  \min(x,y) & \text{otherwise}
\end{cases}
\]
is called the \emph{ordinal sum} of the summands $(a_\alpha, e_\alpha, T_\alpha)$, $\alpha \in A$.
We denote this as $T = ((a_\alpha, e_\alpha, T_\alpha))_{\alpha \in A}$.
\end{definition}

\begin{theorem}[Representation of Continuous T-norms {\citep[Theorem 5.11, p.~140]{Klement2000}}]
  A function $T: [0,1]^2 \to [0,1]$ is a continuous t-norm if and only if $T$ is uniquely representable as an ordinal sum of continuous Archimedean t-norms.
\end{theorem}

\begin{remark}
  A t-norm is \emph{ordinally irreducible} if it only has a trivial ordinal sum representation (i.e., $T = ((0,1,T))$). For continuous t-norms, being ordinally irreducible is equivalent to being Archimedean (\citep[Proposition 3.53, p.~99]{Klement2000}).
\end{remark}

In summary, continuous t-norms are either Archimedean (and thus isomorphic to $T_P$ or $T_L$) or they are ordinal sums of such Archimedean t-norms (and $T_M$ as a limiting case).





\begin{example}
  \signal{Some examples of t-norms and t-conorms.}
  \begin{itemize}
    \item \textbf{Minimum/Maximum:} 
    The strongest t-norm (minimum) and weakest t-conorm (maximum). They are the standard interpretations of conjunction and disjunction in fuzzy logic and are idempotent ($T(x,x)=x, S(x,x)=x$).\\
    $T_M(x, y) = \min(x, y) \quad S_M(x, y) = \max(x, y)$

    \item \textbf{Łukasiewicz:}
    Represents a bounded arithmetic sum/difference logic. Corresponds to the logic introduced by Łukasiewicz. It is Archimedean but not strict (has zero divisors). Boundary condition for t-norms satisfying $T(x,y)+S(x,y) = x+y$. \\
    $T_L(x, y) = \max(0, x + y - 1) \quad S_L(x, y) = \min(1, x + y)$

    \item \textbf{Product:}
    The standard algebraic product t-norm and the probabilistic sum t-conorm. Represents the intersection of independent events in probability theory. It is a strict Archimedean t-norm. \\
    $T_P(x, y) = x \cdot y \quad S_P(x, y) = x + y - x \cdot y$

    \item \textbf{Drastic:}
    The weakest t-norm and the strongest t-conorm. They represent the most extreme intersection and union possible. \signal{Discontinuous except at boundary points (0,0), (1,1), etc.} \\
    $T_D(x, y) = \begin{cases} \min(x,y) & \text{if } \max(x,y)=1 \\ 0 & \text{otherwise} \end{cases} \quad S_D(x, y) = \begin{cases} \max(x,y) & \text{if } \min(x,y)=0 \\ 1 & \text{otherwise} \end{cases}$
    % Simpler equivalent definitions:
    % T(x, y) = y if x=1, x if y=1, 0 otherwise
    % S(x, y) = y if x=0, x if y=0, 1 otherwise

    \item \textbf{Hamacher Family ($\gamma \ge 0$):}
    A parametric family of t-norms and t-conorms. Includes the Product t-norm as a special case when $\gamma = 1$. It is strictly Archimedean for $\gamma > 0$. \\
    $T_\gamma(x, y) = \frac{xy}{\gamma + (1-\gamma)(x+y-xy)} \quad S_\gamma(x, y) = \frac{x+y-xy-(1-\gamma)xy}{1-(1-\gamma)xy}$
    % Note: For gamma=0 this is related to Lukasiewicz via generator, for gamma -> infinity to Drastic.

    \item \textbf{Dubois and Prade Family ($\alpha \in [0, 1]$):}
    A parametric family that includes the Minimum t-norm ($\alpha=0$) and the Product t-norm ($\alpha=1$). Provides flexibility between these two important t-norms. \\
    $T_\alpha(x, y) = \frac{xy}{\max(x, y, \alpha)} \quad S_\alpha(x, y) = 1 - \frac{(1-x)(1-y)}{\max(1-x, 1-y, \alpha)}$ 
    % Note: The S-conorm is expressed directly via duality relation S(x,y) = 1 - T(1-x, 1-y).

    \item \textbf{Yager Family ($p > 0$):}
    A parametric family characterized by the parameter $p$. It approaches the Drastic t-norm as $p \to 0^+$ and the Minimum t-norm as $p \to \infty$. The Łukasiewicz t-norm is obtained for $p=1$. \\
    $T_p(x, y) = \max(0, 1 - ((1-x)^p + (1-y)^p)^{1/p}) \quad S_p(x, y) = \min(1, (x^p + y^p)^{1/p})$

    \item \textbf{Frank Family ($p > 0, p \neq 1$):}
    The only family of t-norms (besides min-max) that are Archimedean and satisfy the functional equation $T(x, y) + S(x, y) = x + y$. It includes Łukasiewicz ($p \to 0^+$), Product ($p \to 1$), and Minimum/Maximum ($p \to \infty$) as limiting cases. \\
    $T_p(x, y) = \log_p \left( 1 + \frac{(p^x - 1)(p^y - 1)}{p - 1} \right) \quad S_p(x, y) = 1 - \log_p \left( 1 + \frac{(p^{1-x} - 1)(p^{1-y} - 1)}{p - 1} \right)$

    \item \textbf{Schweizer-Sklar Family ($p \in [-\infty, \infty]$):}
    A very general parametric family encompassing several others. Includes Drastic ($p \to -\infty$), Łukasiewicz ($p=1$ in a different parameterization, not this T form), Product ($p=0$ in a related form), and Minimum ($p \to \infty$). The Yager family is generated differently but related. \\
    $T_p(x, y) = (\max(0, x^p + y^p - 1))^{1/p} \quad S_p(x, y) = 1 - (\max(0, (1-x)^p + (1-y)^p - 1))^{1/p}$ 
    % Note: S-conorm shown is the dual S_p(x,y) = 1 - T_p(1-x, 1-y). The commonly cited S_p(x,y) = (min(1, x^p + y^p))^{1/p} is NOT dual to this T_p w.r.t standard negation.
  \end{itemize}
\end{example}





\subsection{Construction and Properties of T-norms} % Example section

\subsubsection{Generators for T-norms}

One of the most powerful methods for constructing and characterizing t-norms, particularly continuous Archimedean ones, involves the use of generator functions.

\begin{definition}[Additive Generator {\cite[Definition 3.25, p.~70]{Klement2000}}]
  An \emph{additive generator} $t: [0,1] \to [0,\infty]$ of a t-norm $T$ is a strictly decreasing function which is also right-continuous in $0$ and satisfies $t(1)=0$, such that for all $(x,y) \in [0,1]^2$:
  \begin{enumerate}
      \item $t(x) + t(y) \in \mathrm{Ran}(t) \cup [t(0), \infty]$, and
      \item $T(x,y) = t^{(-1)}(t(x) + t(y))$,
  \end{enumerate}
  where $t^{(-1)}$ is the pseudo-inverse of $t$.
\end{definition}

\begin{remark}
  As established in \cite[Theorem 5.1, p.~122]{Klement2000}, every continuous Archimedean t-norm possesses a continuous additive generator, unique up to a positive multiplicative constant. Conversely, any such generator defines a continuous Archimedean t-norm.
\end{remark}

\begin{definition}[Multiplicative Generator {\cite[Definition 3.36, p.~91]{Klement2000}}]
  A \emph{multiplicative generator} $\theta: [0,1] \to [0,1]$ of a t-norm $T$ is a strictly increasing function which is right-continuous in $0$ and satisfies $\theta(1)=1$, such that for all $(x,y) \in [0,1]^2$:
  \begin{enumerate}
      \item $\theta(x) \cdot \theta(y) \in \mathrm{Ran}(\theta) \cup [0, \theta(0)]$, and
      \item $T(x,y) = \theta^{(-1)}(\theta(x) \cdot \theta(y))$,
  \end{enumerate}
  where $\theta^{(-1)}$ is the pseudo-inverse of $\theta$.
\end{definition}

\begin{remark}
  There is a direct duality between additive and multiplicative generators. If $t$ is an additive generator, then $\theta(x) = e^{-t(x)}$ (or $e^{-c \cdot t(x)}$ for $c>0$) can serve as a multiplicative generator, and vice-versa with $t(x) = -\log(\theta(x))$ (\cite[Remark 3.34, p.~90]{Klement2000}). Continuous Archimedean t-norms also possess continuous multiplicative generators (\cite[Corollary 5.4, p.~124]{Klement2000}).
\end{remark}

\subsubsection{Families of T-norms}

Chapter 4 of \cite{Klement2000} presents various parameterized families of t-norms. These families often interpolate between or include the basic t-norms ($T_M, T_P, T_L, T_D$) as special or limiting cases. Some prominent examples include:

\begin{itemize}
    \item \textbf{Schweizer-Sklar T-norms} ($T_\lambda^{SS}$) for $\lambda \in [-\infty, \infty]$ (\cite[Example 4.3, p.~104]{Klement2000}). This family includes $T_M$ ($\lambda=-\infty$), $T_P$ ($\lambda=0$), and $T_L$ ($\lambda=1$), and $T_D$ ($\lambda=\infty$).
    \item \textbf{Hamacher T-norms} ($T_\lambda^H$) for $\lambda \in [0, \infty]$ (\cite[Example 4.5, p.~106]{Klement2000}). Includes $T_P$ ($\lambda=1$) and $T_D$ ($\lambda=\infty$). $T_0^H$ is a notable member.
    \item \textbf{Frank T-norms} ($T_\lambda^F$) for $\lambda \in [0, \infty]$ (\cite[Example 4.7, p.~108]{Klement2000}). This family includes $T_M$ ($\lambda=0$), $T_P$ ($\lambda=1$), and $T_L$ ($\lambda=\infty$).
    \item \textbf{Yager T-norms} ($T_\lambda^Y$) for $\lambda \in [0, \infty]$ (\cite[Example 4.9, p.~110]{Klement2000}). Includes $T_D$ ($\lambda=0$) and $T_L$ ($\lambda=1$), and $T_M$ ($\lambda=\infty$).
    \item \textbf{Dombi T-norms} ($T_\lambda^D$) for $\lambda \in [0, \infty]$ (\cite[Example 4.11, p.~112]{Klement2000}). Includes $T_D$ ($\lambda=0$) and $T_M$ ($\lambda=\infty$).
    \item \textbf{Aczél-Alsina T-norms} ($T_\lambda^{AA}$) for $\lambda \in [0, \infty]$ (\cite[Example 4.15, p.~116]{Klement2000}). Includes $T_D$ ($\lambda=0$) and $T_M$ ($\lambda=\infty$).
    \item \textbf{Sugeno-Weber T-norms} ($T_\lambda^{SW}$) for $\lambda \in [-1, \infty]$ (\cite[Example 4.13, p.~114]{Klement2000}). Includes $T_L$ ($\lambda=0$) and $T_P$ ($\lambda=\infty$). $T_{-1}^{SW}$ is $T_D$.
    \item \textbf{Mayor-Torrens T-norms} ($T_\lambda^{MT}$) for $\lambda \in [0, 1]$ (\cite[Example 4.17, p.~118]{Klement2000}). This family consists of ordinal sums and includes $T_M$ ($\lambda=0$) and $T_L$ ($\lambda=1$).
\end{itemize}
A summary table of properties for these families, including their continuity and Archimedean nature for different parameter values, can be found in \cite[Table 4.1, p.~119]{Klement2000}. Further details and visualizations are in Appendix A of the book.

\subsection{Continuity Properties of T-norms}

T-norms can exhibit various continuity behaviors.

\begin{example}[Continuous T-norms]
  The basic t-norms $T_M(x,y) = \min(x,y)$, $T_P(x,y) = xy$, and $T_L(x,y) = \max(0, x+y-1)$ are all continuous on $[0,1]^2$ (\cite[p.~15]{Klement2000}). All strict and nilpotent t-norms are, by definition, continuous (\cite[Definition 2.13, p.~42]{Klement2000}).
\end{example}

\begin{remark}[Left-continuous and Right-continuous T-norms]
  The book primarily focuses on continuity in both variables simultaneously or discusses left/right continuity of generator functions. However, some t-norms constructed via ordinal sums using non-continuous components or certain constructions in Chapter 3.4 can exhibit more nuanced continuity.
  For example, any t-norm $T$ that is an ordinal sum (Definition 3.44) will be continuous if and only if all its summands $T_\alpha$ are continuous and certain conditions at the join points $a_\alpha, e_\alpha$ are met (related to limits, see \cite[Proposition 3.49, p.~100]{Klement2000}).
\end{remark}

\begin{example}[Right-continuous, but not Left-continuous T-norm]
  The drastic product $T_D(x,y)$ is not continuous. For instance, consider a sequence $(x_n, y_n) = (1-\frac{1}{n}, 1-\frac{1}{n})$ converging to $(1,1)$.
  $T_D(x_n, y_n) = 0$ for all $n$, so $\lim_{n\to\infty} T_D(x_n, y_n) = 0$.
  However, $T_D(1,1) = 1$.
  It is upper semicontinuous, which implies it is right-continuous in each variable when the other is fixed (see \cite[Remark 1.21, Proposition 1.22]{Klement2000} which link upper/lower semicontinuity to right/left continuity of monotone functions).
  Specifically, $T_D$ is right-continuous in each variable but not left-continuous. For example, fixing $y=1$, $T_D(x,1)=x$ is continuous. Fixing $y \in [0,1[$, $T_D(x,y)$ is $0$ for $x \in [0,1[$ and $y$ for $x=1$. This is right-continuous at $x=0$ (for $y<1$) but not left-continuous at $x=1$.
\end{example}

\begin{example}[Left-continuous, but not Right-continuous T-norm]
  The book mentions the nilpotent minimum $T^{nM}$ (\cite[Remark 1.21, p.~16 and Fig 1.5, p.~17]{Klement2000}) defined as:
  \[
  T^{nM}(x,y) =
  \begin{cases}
    0 & \text{if } x+y \leq 1 \\
    \min(x,y) & \text{otherwise}
  \end{cases}
  \]
  This t-norm is lower semicontinuous, which for monotone functions implies it is left-continuous in each variable (\cite[Proposition 1.22, p.~17]{Klement2000}). It is not upper semicontinuous (and thus not right-continuous in general). For example, let $x_n = 0.5 + \frac{1}{n}$ and $y_n = 0.5 + \frac{1}{n}$. Then $x_n+y_n > 1$, so $T^{nM}(x_n, y_n) = 0.5 + \frac{1}{n} \to 0.5$. However, $T^{nM}(0.5, 0.5) = 0$.
\end{example}

\begin{example}[Discontinuous T-norm (neither left nor right continuous in general)]
  The Krause t-norm $T^K$ (\cite[Appendix B.1, p.~341 and Theorem B.1, p.~344]{Klement2000}) is explicitly stated to be "neither left- nor right-continuous, but has a continuous diagonal section."
  Another example of a t-norm that is generally discontinuous (unless specific parameters are chosen) is the t-norm from \cite[Example 3.21, p.~80]{Klement2000}, constructed using a discontinuous generator $f$. The resulting t-norm $(T_P)_{[f(-1)]}$ would generally be discontinuous.
\end{example}

\begin{remark}
  The book also mentions t-norms that are "border continuous" (\cite[Definition 1.23, p.~17]{Klement2000}), meaning they are continuous on the boundary of $[0,1]^2$ but not necessarily in the interior. \cite[Example 1.24(i), p.~18]{Klement2000} gives such an example which is border continuous but not left-continuous (and hence not fully continuous).
\end{remark}





















































\section{Fuzzy Relations}
  In the previous sections we have been working with a single domain denoted by $X$. Intuitively, it might be useful to think about the domain as the different values of a property (independently of the object that has that property). For example, we could say that for the property \textit{lenght} the domain is $\R^+\coleq
  \{x\in\R \mid x \geq 0\}$ and a person might have a \textit{height} defined in that domain (modeled as a fuzzy set), although not all length values will be compatible with a person's height since it is safe to assume impossible to be 10 meters tall, but we do not know where to place a sharp boundary, so the feasible region of heights is best represented by a fuzzy set. If we were to look at the arm length, then we would have another set of feasible lenghts. Treating each set independently, doesn't allow to distinguish between the people of the same height with different arm lenght or viceversa. Therefore it is needed to a way to differentiate each unique combination of attributes. Notice that in this example, height and arm length have the same domain but for example hair color would have a different domain.\\

  Mathematically, this is done with relations which are subsets of the cartesian product of the domains, so that each element is a unique possible combination of attributes, an ordered n-tuple. For simplicity, let us consider just the cartesian product of 2 sets since the general case can be obtained inductively (see remark below). Again, the ``fuzzy" part will be referred to how we generalize the membership values to the continuum.


  \begin{definition}[Fuzzy Relation]
    Let $X\neq \emptyset \neq Y$ be classical sets. Then a fuzzy relation $R$ is a fuzzy set on $X\times Y$, i.e., $R\in \fuzzy{X\times Y}$. $R(x,y)$ will denote the degree of membership of $(x,y) \in R$.
  \end{definition}

  \begin{notation}[label={not:compositionFS}]{Notation}
    Although Fuzzy Relations are Fuzzy Sets as well, the name distinction will be used to denote whether the domain is a cartesian product or not.
  \end{notation}

  \begin{remark}
    To formally extend the Cartesian product from two sets to \( n \) sets using induction, it is important to observe that it satisfies associativity up to a natural isomorphism, i.e., 
    \[
    (A\times B)\times C \cong A\times (B\times C).
    \]
    and that t-norms satisfy the associativity property.
  \end{remark}

  Before giving the definition of the fuzzy cartesian product, we need to first understand what the classical cartesian product is in terms of the membership function. When we define a cartesian product $A\times B$ as "\textit{all unique ordered pairs of elements from $A$ \textbf{and} $B$}", in terms of membership functions we are taking the intersection of membership to $A$ and membership to $B$. Therefore, generalizing that notion with a t-norm we get the following definition.

  \begin{definition}[Fuzzy Cartesian Product]
    It is a fuzzy relation $A\times B \in \fuzzy{X\times Y}$ such that the membership function is given by:
    \[ 
    (A\times B)(x,y) = T(A(x), B(y)), \quad \forall (x,y) \in X\times Y
    \]
    where $T$ is a t-norm.
  \end{definition}

  To justify the definition of the membership function of a cartesian product of two fuzzy sets, let us first recall that in classical sets, we can retrieve the orginal subsets individually by taking the projection of the cartesian product. Given $R$ a relation on $X\times Y$ the projections are:

  \[\Pi_X(R)=\{x \in X \mid \exists y \in Y \textnormal{ such that } (x,y) \in R\}\]
  \[\Pi_Y(R)=\{y \in Y \mid \exists x \in X \textnormal{ such that } (x,y) \in R\}\]

  Then, with the boolean membership, it can be expressed as well as:

  \[\Pi_X(R)(x)=\sup\{R(x,y) \mid y\in Y\}=
  \begin{cases}
    1 & \textnormal{if } \exists y \in Y \textnormal{ such that } (x,y) \in R \\
    0 & \textnormal{otherwise}
  \end{cases}
  \]
  \[
    \Pi_Y(R)(y)=\sup\{R(x,y) \mid x\in X\}=
    \begin{cases}
      1 & \textnormal{if } \exists x \in X \textnormal{ such that } (x,y) \in R \\
      0 & \textnormal{otherwise}
    \end{cases}
  \]

  It is clear that in the case of having 2 possible values for the membership function, the above expresions are identical. The use of $\sup$ in the definition of the projection can be intuitively interpreted as the \textit{shadow} of the membership function of the cartesian product, as figure \ref{fig:class_cart_prod} illustrates. It is also desirable because then the following property holds for any classical cartesian product:\\
  $$ 
  \Pi_X(A\times B)=A \quad \textnormal{ and } \quad \Pi_Y(A\times B)=B \forall A\in X, \forall B \in Y
  $$

  \begin{figure}[ht]
    \centering
    \includegraphics[width=0.65\textwidth]{ch1/figures/class_cart_prod.png}
    \caption{The blue volume represents the membership function of the cartesian product of two classical sets $X=[a,b]$ and $Y=[c,d]$ in $\R$. In the plane $y=0$ we have the projection that corresponds to the membership function of $X$ and analogously, the projection of $Y$ in the plane $x=0$.}
    \label{fig:class_cart_prod}
  \end{figure}

  Therefore, we define:

  \begin{definition}[Projection of a fuzzy relation]
    The projection from fuzzy relations on $X\times Y$ onto the fuzzy sets on $X$ is the function:
    \[
      \begin{aligned}
        \Pi_X: \fuzzy{X\times Y} &\longrightarrow \fuzzy{X} \\
        R &\longmapsto \Pi_X(R)
      \end{aligned}
    \]
    where $\Pi_X(R)(x) = \sup_{y\in Y}\{R(x,y)\}\forall x \in X$
  \end{definition}

  Applying the definition of projection with the supremum to fuzzy sets, we have a way to retrieve the membership function of each fuzzy set given the membership function of a fuzzy cartesian product (figure \ref{fig:fuzzy_cart_prod}). Therefore, any fuzzy cartesian product can be expressed as the fuzzy cartesian product of its own projections. This is justified in the following proposition: \\






  \begin{figure}[ht]
      \centering
      \includegraphics[width=0.65\textwidth]{ch1/figures/fuzzy_cart_prod.png}
      \caption{The blue volume represents the membership function of the cartesian product of two fuzzy sets $X$ and $Y$. In the plane $y=0$, we have the projection that corresponds to the membership function of $X$, and analogously, the projection of $Y$ in the plane $x=0$. The partial memberships illustrate how the fuzzy relations can vary across the domains.}
      \label{fig:fuzzy_cart_prod}
  \end{figure}



\begin{proposition}[Retrieval of Fuzzy Sets from their T-norm Cartesian Product]
  Let $A \in \fuzzy{X}$ and $B \in \fuzzy{Y}$ be fuzzy sets and $T$ a left-continuous t-norm. If the fuzzy set $B$ is normalized (i.e., $\sup\{B(y) \mid y \in Y\}  = 1$), then $\Pi_X(A \times B)(x) = A(x)$ for all $x \in X$. 
  In general, if $B$ is not normalized, we only get a lower bound $\Pi_X(A \times B)(x) \le A(x)$.
  \end{proposition}
  
  \begin{proof}
  \begin{equation*}
    \begin{split}
      \Pi_X(A \times B)(x) &= \sup_{y \in Y} T(A(x), B(y))= \\
      &= T(A(x), \sup_{y \in Y} B(y)) \leq
      \begin{cases}
        \min(A(x), 1) = A(x) & \text{if } B \text{ is normalized}\\
        \min(A(x), \sup_{y \in Y} B(y)) \leq A(x) & \text{otherwise}
      \end{cases}
    \end{split}
  \end{equation*}
  Since $A(x)$ is a constant with respect to the supremum over $y$, and t-norms are non-decreasing and left-continuous\footnote{If the function was only left-continuous but not non-decreasing, then that equality would instead be the $\geq$ inequality.} in their second argument, then the second equality holds. The inequality before the cases is justified using that the minimum t-norm is the strongest t-norm.
  \end{proof}
  
  \begin{remark}
  It is also crucial to emphasize that this retrieval is meaningful when the fuzzy relation $R$ under consideration is indeed a fuzzy Cartesian product of the form $A \times B$ using the t-norm $T$. If a given fuzzy relation $R'$ on $X \times Y$ is not $T$-decomposable (i.e., it cannot be expressed as $T(A'(x), B'(y))$ for any fuzzy sets $A'$ on $X$ and $B'$ on $Y$ with respect to the t-norm $T$), then the notion of retrieving original sets $A'$ and $B'$ from $R'$ doesn't make sense. While one can always compute the projections $A_P(x) = \Pi_X(R')(x)$ and $B_P(y) = \Pi_Y(R')(y)$, the relation $R'$ will not necessarily be equal to the $T$-Cartesian product of its projections, i.e., $R'(x,y) \neq T(A_P(x), B_P(y))$ in general for non-decomposable relations.
  \end{remark}


  Since infinitely many relations (including those that are not a fuzzy cartesian product) may have the same projections, it is not possible define the inverse operation. However, in the literature \cite[p.~61]{HistoryFL2017}, it is also defined the cylindric extension of a fuzzy set $A\in\fuzzy{X}$ as $CE_X(A)(x,y) = A(x)$. This operation is the simplest way to extend a fuzzy set to a relation ($\Pi_X(CE_X(A))=A$), and can be generalized to any n-ary relation as well.\\



\subsubsection*{Types of Fuzzy Relations on a Single Set}

When a binary fuzzy relation $R$ is defined on the Cartesian product of a single set with itself, it can characterize various ways in which elements of $X$ relate to themselves. Several properties, analogous to those in classical relations, are important for classifying these fuzzy relations \cite[p.~66]{HistoryFL2017}.

\begin{definition}[Properties of Binary Fuzzy Relations on $X^2$] Let $R \in \fuzzy{X \times X}$ (or $R \in \fuzzy{X^2}$) be a fuzzy binary relation, then:
  \begin{itemize}
    \item $R$ is \textbf{reflexive} if $R(x,x) = 1$ for all $x \in X$.
          Intuitively, every element is fully related to itself.
    \item $R$ is \textbf{symmetric} if $R(x,y) = R(y,x)$ for all $x,y \in X$.
          Intuitively, the degree of relationship from $x$ to $y$ is the same as from $y$ to $x$.
    \item $R$ is \textbf{transitive} (specifically, transitive under a t-norm $T$) if for all $x,y,z \in X$ and a given t-norm $T$,
          \[ T(R(x,y), R(y,z)) \le R(x,z). \]
          Using the definition of composition (see definition \ref{def:compos})this can be expressed as $R \supseteq R \circ R$. Often the minimum t-norm is used and it is then called sup-min transitive.
  \end{itemize}
\end{definition}

Based on these properties, two important types of fuzzy relations are:

\begin{definition}[Fuzzy equivalence relation]
  A fuzzy relation $S \in \fuzzy{X \times X}$ is A fuzzy equivalence relation (also called similarity relation) if it is reflexive, symmetric, and transitive (typically sup-min transitive).
\end{definition}
It generalizes the concept of a classical equivalence relation to the fuzzy context, indicating the degree to which elements are considered ``similar" or ``equivalent."

\begin{definition}[Fuzzy Compatibility Relation]
  A fuzzy relation $C \in \fuzzy{X \times X}$ is a \textbf{fuzzy compatibility relation} (also sometimes called a tolerance or proximity relation) if it is reflexive and symmetric.
\end{definition}
A fuzzy compatibility relation indicates that elements are compatible or close to each other, but this compatibility is not transitive. If $x$ is compatible with $y$, and $y$ with $z$, $x$ is not necessarily compatible with $z$ to the same degree.










\subsection{Composition of Fuzzy Sets}
\label{sec:compos}

The concept of projection allows us to combine fuzzy relations sharing a common domain. Intuitively, composing two fuzzy relations involves intersecting their membership values (using a t-norm) and then projecting the result onto the domain where the relations do not overlap.

\begin{definition}[Composition of Two Fuzzy Relations]\label{def:compos}
    Let \( R \in \fuzzy{X \times Y} \) and \( G \in \fuzzy{Y \times Z} \) be fuzzy relations sharing the set \(Y\). Their composition \( R \circ G \) is the fuzzy relation in \(\fuzzy{X \times Z}\) defined by
    \[
    (R \circ G)(x,z) = \Pi_{X\times Z}\Bigl[\, T\bigl(R(x,y), G(y,z)\bigr) \Bigr] = \sup_{y\in Y}\, T\bigl(R(x,y), G(y,z)\bigr),
    \]
    where \(T\) is a t-norm acting as the fuzzy intersection.
\end{definition}

% This means that given three properties $X$, $Y$ and $Z$ and two fuzzy relation R from X to Y and another fuzzy relation from Y to Z, we have a way to induce a relation $R\circ G$ between X and Z.

% (add a diagram here)

Given three sets \(X\), \(Y\), and \(Z\), suppose we have a fuzzy relation \(R\) from \(X\) to \(Y\) and another fuzzy relation \(G\) from \(Y\) to \(Z\). Using the composition operation, we can derive a new fuzzy relation \(R \circ G\) that directly connects \(X\) to \(Z\).

\noindent
\begin{minipage}{0.7\textwidth}
In this diagram, the arrows represent fuzzy relations, with \(R\) mapping elements from \(X\) to \(Y\), \(G\) mapping from \(Y\) to \(Z\), and \(R \circ G\) representing the induced fuzzy relation between \(X\) and \(Z\) through composition.\\
\end{minipage}%
\begin{minipage}{0.3\textwidth}
  \begin{center}
    \begin{tikzcd}
      X \arrow[r, ``R", leftrightarrow] & Y \arrow[r, ``G", leftrightarrow] & Z \arrow[bend right=30, from=1-1, ``R \circ G"', leftrightarrow]
      \end{tikzcd}
  \end{center}

\end{minipage}


We can define a the composition of a fuzzy set with a fuzzy relation in a completely analogous way. 

\begin{definition}[Composition of a Fuzzy Set and a Fuzzy Relation]
    Let \( A \in \fuzzy{X} \) be a fuzzy set on \(X\) and \( R \in \fuzzy{X \times Y} \) be a fuzzy relation between \(X\) and \(Y\). The composition \( A \circ R \in \fuzzy{Y} \) is defined by
    \[
    (A \circ R)(y) = \Pi_{Y}\Bigl[\, T\bigl( A(x), R(x,y) \bigr) \Bigr] = \sup_{x \in X}\, T\bigl( A(x), R(x,y) \bigr),
    \]
    where \(T\) is a t-norm.
\end{definition}

\subsection{Extension Principle}

Following a similar idea as in the definition of the composition of fuzzy sets, we can derive a way to generalize crisp functions to fuzzy sets.\\

Let's consider the crisp function $f:\,X \longrightarrow Y$ where $X$ and $Y$ are classical sets. And consider as well the fuzzy sets $A \in \fuzzy{X}$. Then $f$ induces the classical relation $R=\{(x,y)\in X\times Y \mid f(x)=y\}$. And we can also make this relation fuzzy by copying the membership function of $A$:
$$ \mu_R (x,y) = \mu_A (x) \forall (x,y)\in R$$
Now we can define $B\in\fuzzy{Y}$, the fuzzy image of $A$ under $f$, as the projection of this fuzzy relation:
$$\mu_B (y) = \Pi_Y (R) = \sup\{R(x,y)\mid x\in X\} = \sup\{\mu_A (x)\mid x\in X, \, f(x)= y\} = \sup_{x\in f^{-1}(y)}\{\mu_A(x)\}$$

This is called the (Zadeh's) extension principle and is the basis for building arithmetic for fuzzy numbers (see section \ref{sec:fuzzy_numbers}) and generalizing any crisp function to fuzzy sets: 

\begin{definition}[Zadeh's extension principle]
  Let $f: X \longrightarrow Y$ be a crisp function and $A\in \fuzzy{X}$ a fuzzy set on $X$. Then we can define $f(A)\in \fuzzy{Y}$ as:
  \[
  \mu_{f(A)}(y)\equiv f(A)(y) = 
  \begin{cases}
    \sup_{x\in f^{-1}(y)}A(x) & \textnormal{if } f^{-1}(y)\neq \emptyset\\
    0 & \textnormal{otherwise}
  \end{cases}
  \quad\quad\quad \textnormal{where } f^{-1}(y)=\{x\in X \mid f(x)=y\}
  \]
\end{definition}


\begin{remark}
  If $f$ is \textbf{injective} then for any $y \in \textnormal{Im}(f)$ there exists a unique $x \in X$ such that $f(x)=y$, and therefore $f^{-1}(y)=\{x\}$. Then the first case can be rewritten as, $f(A)(y) = A(f^{-1}(y))$ if $y \in \textnormal{Im}(f)$.
\end{remark}

It is important to highlight as well that this definition is a straight forward generalization of set-valued functions where $f(A)= \{f(x)\mid x\in A\}$. In terms of boolean membership function:
$$\chi _{f(A)}(y)=\sup_{x\in f^{-1}(y)}\chi_A(x)$$

The definition above can be generalized to vector functions using the definition of fuzzy cartesian product, which requires us to take the intersection:

\begin{definition}[Sup-T extension principle]
  Let $f: X_1 \times \cdots \times X_n \longrightarrow Y$ be a crisp function and $A_1 \in \fuzzy{X_1}, \ldots, A_n \in \fuzzy{X_n}$ be fuzzy sets. Then we can define $f(A_1,\ldots,A_n)\in \fuzzy{Y}$ as:
  \[
  \mu_{f(A_1,\ldots,A_n)}(y)\equiv f(A_1,\ldots,A_n)(y) = 
  \begin{cases}
    \sup_{(x_1,\ldots,x_n)\in f^{-1}(y)} T(A_1(x_1),\ldots,A_n(x_n)) & \textnormal{if } f^{-1}(y)\neq \emptyset\\
    0 & \textnormal{otherwise}
  \end{cases}
  \]
  where $f^{-1}(y)=\{(x_1,\ldots,x_n)\in X_1\times\cdots\times X_n \mid f(x_1,\ldots,x_n)=y\}$ and $T$ is a t-norm.
\end{definition}

Which is again a generalization of vector set-valued functions where: $$f(A_1,\ldots,A_n)= \{f(x_1,\ldots,x_n)\mid x_i\in A_i\}$$
In terms of boolean membership function:
$$\chi _{f(A_1,\ldots,A_n)}(y)=\sup_{(x_1,\ldots,x_n)\in f^{-1}(y)}T(\chi_{A_1}(x_1),\ldots,\chi_{A_n}(x_n)) = \sup_{(x_1,\ldots,x_n)\in f^{-1}(y)}min\{\chi_{A_1}(x_1),\ldots,\chi_{A_n}(x_n)\}$$

Where we have used that every t-norm operates the same on boolean memberships.\\

\signal{Hay otros principios de extensión como el de Ramik con extensiones canónicas y order preserving operators, etc. No sé hasta qué punto eso podrá serme útil. Igual lo menciono en una frase y ya.}
\section{Fuzzy Numbers}\label{sec:fuzzy_numbers}
%See Nguyen paper page 7-8 of the pdf and 375-376 of the book.
According to Nguyen \cite{NGUYEN1978}:
 
\say{Interval analysis deals with closed bounded intervals (complex convex sets of $\R$) as an extension of numbers. Fuzzy numbers can be regarded as 
an extension of closed bounded intervals, [...]} 
\\

\signal{Another way to think about fuzzy numbers is as a special case of fuzzy intervals.}

\signal{Explain that we want to define fuzzy numbers to work with fuzzy attributes later on}


\begin{definition}[Normal Fuzzy Set]
    A fuzzy set $A\in \fuzzy{X}$ is called normal if there exists $x\in X$ such that $A(x)=1$. Otherwise it is called subnormal.
\end{definition}

\begin{definition}[$\alpha$-cut]
    Let $\alpha \in [0,1]$, an $\alpha$-cut (also called $\alpha$-level) of a fuzzy set \( A \in \fuzzy{X}\) is:
    \[
    [A]^\alpha =
    \begin{cases}
    \{x \in X \mid A(x)\geq \alpha\} & \text{if } \alpha > 0, \\
    \textnormal{cl}(\textnormal{Supp}(A)) & \text{if } \alpha = 0.
    \end{cases}
    \]
    where \textit{cl} denotes the closure.
\end{definition}

\begin{remark}
    From the definition of $\alpha$-cut, the nested property states that for
    $\alpha_1, \alpha_2 \in [0,1]$ if $\alpha_1\leq \alpha_2$ then $[A]^{\alpha_2}\subseteq [A]^{\alpha_1}$
\end{remark}

\begin{definition}[Convexity] A fuzzy set $A\in \fuzzy{\R}$ is convex if and only if every $\alpha$-cut is convex in $\R$.
    
\end{definition}

\begin{definition}[Fuzzy Number]
    A fuzzy number is a fuzzy set in the real line, i.e., $A\in \fuzzy{\R}$ such that:\vspace{-0.9em}
    \begin{romanenum}
        \item Normal\vspace{-0.5em}
        \item Convex\vspace{-0.5em}
        \item $\mu_A$ is continuous.\vspace{-0.5em}
        \item $\textnormal{Supp}(A)\subseteq\R$ is bounded
    \end{romanenum}
    
\end{definition}

\begin{proposition}[$\alpha$-cuts are closed intervals]
    Let $A\in \fuzzy{\R}$ be a fuzzy number. Then for every $\alpha \in [0,1]$, the $\alpha$-cut $[A]^\alpha$ is a closed interval in $\R$.
\end{proposition}

\begin{proof}
%1
The fact that $[A]^\alpha$ is an interval follows from the definition of convex subset in $\R$ with the usual topology, which can only be an interval (or a single point).\\
%2
Now we prove that $[A]^\alpha$ is closed. For $\alpha \in (0,1]$, since $\mu_A$ is continuous and $[\alpha, 1]$ is closed in $\R$, the set
\[
[A]^\alpha = \mu_A^{-1}([\alpha, 1])
\]
is closed in $\R$. %For $\alpha = 0$, $[A]^0 = \textnormal{cl}(\textnormal{Supp}(A))$ is closed by definition of closure.
\end{proof}

\begin{notation}{Notation}
    We will denote the $\alpha$-cuts of a fuzzy number $A$ as 
    \[[A]^\alpha=[a_1(\alpha),a_2(\alpha)]\textnormal{ where }\begin{cases}
        a_1(\alpha)=min[A]^\alpha&\\
        a_2(\alpha)=max[A]^\alpha&\\
    \end{cases}\]
\end{notation}

\begin{note}
The condition of bounded support can be relaxed to define \textit{quasi-fuzzy numbers} \signal{(Which properties still hold and which are lost?)}:
$$\textnormal{(iv}_{\textnormal{bis}}\textnormal{) } \lim{t}{\infty}A(t) = 0 \quad \land \quad \lim{t}{-\infty}A(t) = 0$$
\end{note}

The following proposition (mentioned in \cite[p.~3]{FULLER2} as a comment without proof) establishes that the membership function of any fuzzy number can be partitioned into three contiguous intervals: one where it monotonically increases, one where it equals 1, and one where it monotonically decreases. This characterization shows that every fuzzy number can be represented as an LR-fuzzy number (see example \ref{ex:fuzzy_num} for the definition).

\begin{proposition}[Membership function of fuzzy numbers]
    Let $A\in \fuzzy{\R}$ be a fuzzy number, then it satisfies:
    \begin{romanenum}
        \item $\mu_A(t)=0$ outside an interval (denoted by $[a,d]$)\vspace{-0.5em}
        \item $\exists b,c \in \R \mid a\leq b \leq c \leq d$ where $\begin{cases}
            \mu_A\textnormal{ is monotone increasing in }[a,b]\\
            \mu_A\textnormal{ is monotone decreasing in }[b,d]\\
        \end{cases}$\vspace{-0.5em}
        \item $\mu_A(t)=1 \forall t\in [b,c]$
    \end{romanenum}
\end{proposition}


\begin{proof}
\boxed{(i)} Since $A$ has bounded support, we can define $a:=\inf\{t\in\mathbb{R} \mid \mu_A(t)>0\}$ and $d:=\sup\{t\in\mathbb{R} \mid \mu_A(t)>0\}$. Therefore $\mu_A(t)=0$ for all $t\notin[a,d]$. \\

\boxed{(iii)} Since $A$ is normal, we define $b:=\inf\{t\in\mathbb{R} \mid \mu_A(t)=1\}$ and $c:=\sup\{t\in\mathbb{R} \mid \mu_A(t)=1\}$. By continuity and convexity if there $\exists t\in [b,c]$ where $\mu_A(t)<1$, then $\exists \epsilon >0 \mid t\notin [A]^{t+\epsilon}$ is not a closed interval. Therefore we have $\mu_A(t)=1$ for all $t\in[b,c]$. \\

\boxed{(ii)} Since every $\alpha$-cut $[A]^\alpha=[a(\alpha),d(\alpha)]$ is a closed interval. The nested property of $\alpha$-cuts implies $a(\alpha)$ is non-decreasing and $d(\alpha)$ is non-increasing. For any $s,t\in[a,b]$ with $s<t$ and $\mu_A(s)=\alpha$, we have $t\in[A]^\alpha$, so $\mu_A(t)\geq\alpha=\mu_A(s)$. Similarly for $s,t\in[c,d]$ with $s<t$ and $\mu_A(t)=\alpha$, we have $s\in[A]^\alpha$, so $\mu_A(s)\geq\alpha=\mu_A(t)$. Therefore $\mu_A$ is monotone increasing on $[a,b]$ and monotone decreasing on $[c,d]$.
\end{proof}


% Write me a python function to represent the following fuzzy numbers in theree plots in the same figure. I want the letters to be the same as the ones in the definition and I want them all to be in the positive quadrant. Also write the 1 of the membership in the y axis saying that axis is the membership function of A \mu_A. The area of the fuzzy number must be gray and plot also thin lines for the reference values.
\begin{example}\label{ex:fuzzy_num}
    Here are some examples of fuzzy numbers:
    \begin{itemize}
        \item \textbf{Triangular Fuzzy Number:} Defined by a triplet $A\equiv(a, \alpha, \beta)$ where $a$ is the peak and $\alpha$ and $\beta$ the right and left widths respectively. The membership function $\mu_A(x)$ is given by:
        \[
        \mu_A(x) = 
        \begin{cases} 
        1-\frac{a-x}{\alpha} & \text{if } a \leq x < a-\alpha, \\
        1-\frac{x-a}{\beta} & \text{if } a+\beta < x \leq a, \\
        0, & \text{otherwise.}
        \end{cases}
        \]
        
        \item \textbf{Trapezoidal Fuzzy Number:} Defined by a quadruplet $A\equiv(a, b, \alpha, \beta)$ where $[a,b]$ is the tolerance interval and $\alpha$ and $\beta$ the right and left widths respectively. The membership function $\mu_A(x)$ is given by:
        \[
        \mu_A(x) = 
        \begin{cases} 
        1-\frac{a-x}{\alpha} & \text{if } a \leq x < a-\alpha, \\
        1, & \text{if } b \leq x \leq a, \\
        1-\frac{x-b}{\beta} & \text{if } b+\beta < x \leq b, \\
        0, & \text{otherwise.}
        \end{cases}
        \]
        
        \item \textbf{LR-Fuzzy Number:} Defined by a quadruplet $A\equiv(a, b, \alpha, \beta)$ where $[a,b]$ is the core (or peak) interval and $\alpha$ and $\beta$ the left and right widths respectively. The membership function $\mu_A(x)$ is given by:
        \[
        \mu_A(x) = 
        \begin{cases} 
        L\left(\frac{a-x}{\alpha}\right) & \text{if } a-\alpha \leq x < a, \\
        1, & \text{if } a \leq x \leq b, \\
        R\left(\frac{x-b}{\beta}\right) & \text{if } b < x \leq b+\beta, \\
        0, & \text{otherwise,}
        \end{cases}
        \]
        where $L$ and $R$ are continuous monotone non-increasing functions from $[0,1]$ to $[0,1]$ with $L(0)=R(0)=1$.
    \end{itemize}
\end{example}
    
\begin{figure}[H]
    \centering
    \includegraphics[width=\textwidth]{ch1/figures/fuzzy_numbers.png}
    \caption{Plots of Triangular, Trapezoidal, and LR Fuzzy Numbers}
    \label{fig:fuzzy_numbers}
\end{figure}



\subsection{Nguyen's Theorems}
\signal{We use continuous functions because that way, we get the image of an interval is an interval as well. So then we get another fuzzy number because it maintains the convexity property?

That is because the image under $f:\R \longrightarrow \R$ continuous of a compact is compact and of a connected set is a connected set. Therefore continuous functions move closed intervals to closed intervals.}

%https://sci-hub.se/10.1016/0165-0114(91)90139-H
\begin{theorem}[First Nguyen Theorem]
    Let $f:\, \R \longrightarrow \R$ a continuous function and $A\in \R$ any fuzzy number \signal{(creo que vale para LR fuzzy num)}. Then,
    \[
    [f(A)]^{\alpha} = f([A]^{\alpha})=\{f(x)\mid x\in [A]^\alpha\}
    \]
    Moreover, if $f$ is monotonically increasing (if $f$ were decreasing, the order of the interval would be reversed), then:
    \[
    [f(A)]^{\alpha} = f([a_1(\alpha), a_2(\alpha)])=
    [f(a_1(\alpha)), f(a_2(\alpha))]
    \]
    where $[\cdot]^\alpha$ denotes the $\alpha$-cut of a fuzzy set and $a_1(\alpha), a_2(\alpha)$ the extremes of the $\alpha$-cut.
\end{theorem}

\signal{Sup- t-norm convolution para la generalización lo menciono?? Y eso de la convolución es útil para algo más?}

\begin{theorem}[Second Nguyen Theorem]
    Let $f:\, \R \times \R\longrightarrow \R$ a continuous function and $A,B$ \signal{any} fuzzy numbers. Then,
    \[
    [f(A,B)]^{\alpha} = f([A]^{\alpha},[B]^{\alpha})=\{f(x_1,x_2)\mid x_1\in [A]^\alpha, \, x_2\in [B]^\alpha\}
    \signal{=[A]^\alpha [B]^\alpha}
    \]
    where $[\cdot]^\alpha$ denotes the $\alpha$-cut of a fuzzy set.
\end{theorem}


\signal{Añado generalization of Nguyen Theorems by Fuller in section 1.9 of \cite{FULLER2}?}








\subsection{Nguyen's Theorems}
Nguyen's theorems provide fundamental results for computing the $\alpha$-cuts of fuzzy numbers that result from applying functions, based on Zadeh's extension principle. These theorems are crucial because they allow us to perform operations on fuzzy numbers by working with their $\alpha$-cuts (which are crisp intervals) directly. The following theorem \cite[Thm. 1.3.1, p. 17]{FULLER2}, \cite{NGUYEN1978}:

\begin{theorem}[First Nguyen Theorem]
    Let $f: \mathbb{R} \to \mathbb{R}$ be a continuous function and $A$ be a fuzzy number. Then, the $\alpha$-cut of the fuzzy number $f(A)$ (obtained via the extension principle) is given by:
    \[
    [f(A)]^{\alpha} = f([A]^{\alpha}) = \{f(x) \mid x \in [A]^\alpha\}
    \]

    Moreover, if $f$ is monotonically increasing, and $[A]^\alpha = [a_1(\alpha), a_2(\alpha)]$, then:
    \[
    [f(A)]^{\alpha} = [f(a_1(\alpha)), f(a_2(\alpha))]
    \]
    If $f$ were monotonically decreasing, the resulting interval would be $[f(a_2(\alpha)), f(a_1(\alpha))]$.
\end{theorem}
% \begin{proof}
%     (Intuition) The extension principle defines $(f(A))(y) = \sup_{x: f(x)=y} A(x)$.
%     For $y \in [f(A)]^\alpha$, we have $(f(A))(y) \ge \alpha$. This means there exists an $x_0$ such that $f(x_0)=y$ and $A(x_0) \ge \alpha$. Thus $x_0 \in [A]^\alpha$, and $y = f(x_0) \in f([A]^\alpha)$.
%     Conversely, if $y \in f([A]^\alpha)$, then $y=f(x_0)$ for some $x_0 \in [A]^\alpha$ (so $A(x_0) \ge \alpha$). Then $(f(A))(y) = \sup_{x: f(x)=y} A(x) \ge A(x_0) \ge \alpha$, so $y \in [f(A)]^\alpha$.
%     The continuity of $f$ is essential for ensuring that $f([A]^\alpha)$ is a closed interval when $[A]^\alpha$ is a closed interval, which is required for $f(A)$ to be a fuzzy number.
% \end{proof}

\begin{remark}
Since $A$ is a fuzzy number, its $\alpha$-cuts $[A]^\alpha$ are closed and bounded intervals (compact and connected sets in $\mathbb{R}$). A continuous function $f: \mathbb{R} \to \mathbb{R}$ maps compact sets to compact sets and connected sets (intervals) to connected sets (intervals). Therefore, $f([A]^\alpha)$ is also a closed and bounded interval. This property ensures that $f(A)$ (whose $\alpha$-cuts are these $f([A]^\alpha)$) is indeed a fuzzy number, as its $\alpha$-cuts remain convex (i.e., are intervals) and satisfy other necessary properties. \signal{Y qué pasa con la propiedad de convexity?}
\end{remark}
The following theorem \cite[Thm. 1.3.2, p. 18]{FULLER2}, \cite{NGUYEN1978}:
\begin{theorem}[Second Nguyen Theorem]
    Let $f: \mathbb{R} \times \mathbb{R} \to \mathbb{R}$ be a continuous function (in both arguments) and $A, B$ be fuzzy numbers. Then, the $\alpha$-cut of the fuzzy number $f(A,B)$ (obtained via the extension principle) is given by:
    \[
    [f(A,B)]^{\alpha} = f([A]^{\alpha}, [B]^{\alpha}) = \{f(x_1, x_2) \mid x_1 \in [A]^\alpha, x_2 \in [B]^\alpha\}
    \]
    \signal{Regarding the notation $=[A]^\alpha [B]^\alpha$: This specific notation is typically used when $f$ represents multiplication, i.e., $f(x_1, x_2) = x_1 \cdot x_2$. In that case, $f([A]^\alpha, [B]^\alpha)$ becomes the interval product $[A]^\alpha \cdot [B]^\alpha$. The theorem statement is more general: $f([A]^\alpha, [B]^\alpha)$ means applying the function $f$ to elements from the respective intervals. For example, if $f(x_1, x_2) = x_1 + x_2$, then $f([A]^\alpha, [B]^\alpha) = [A]^\alpha + [B]^\alpha$ (interval addition).}
\end{theorem}
% \begin{proof}
%     (Intuition) Similar to the first theorem, the extension principle for a two-place function is $(f(A,B))(z) = \sup_{x_1,x_2: f(x_1,x_2)=z} \min(A(x_1), B(x_2))$.
%     If $z \in [f(A,B)]^\alpha$, then $(f(A,B))(z) \ge \alpha$. This implies there exist $x_{1,0}, x_{2,0}$ such that $f(x_{1,0}, x_{2,0}) = z$ and $\min(A(x_{1,0}), B(x_{2,0})) \ge \alpha$. Thus, $A(x_{1,0}) \ge \alpha$ (so $x_{1,0} \in [A]^\alpha$) and $B(x_{2,0}) \ge \alpha$ (so $x_{2,0} \in [B]^\alpha$). Therefore, $z = f(x_{1,0}, x_{2,0}) \in f([A]^\alpha, [B]^\alpha)$.
%     Conversely, if $z \in f([A]^\alpha, [B]^\alpha)$, then $z = f(x_{1,0}, x_{2,0})$ for some $x_{1,0} \in [A]^\alpha$ and $x_{2,0} \in [B]^\alpha$. This means $A(x_{1,0}) \ge \alpha$ and $B(x_{2,0}) \ge \alpha$, so $\min(A(x_{1,0}), B(x_{2,0})) \ge \alpha$. Then $(f(A,B))(z) \ge \min(A(x_{1,0}), B(x_{2,0})) \ge \alpha$, so $z \in [f(A,B)]^\alpha$.
%     Continuity of $f$ ensures that the image of the compact set $[A]^\alpha \times [B]^\alpha$ is a compact interval, preserving the fuzzy number structure.
% \end{proof}

\subsubsection{Generalization of Nguyen's Theorem by Fullér and Keresztfalvi}
The original Nguyen's theorem for two-place functions implicitly relies on Zadeh's extension principle, which uses a sup-min convolution to aggregate membership degrees. Fullér and Keresztfalvi generalized this result to cases where the extension principle is defined via a sup-T-norm convolution, where $T$ is an arbitrary t-norm \cite[Sec. 1.9]{FULLER2}.

The generalized extension principle defines the membership of $z$ in $f(A,B)$ as:
\[
(f(A,B))(z) = \sup_{f(x,y)=z} T(A(x), B(y))
\]
\signal{Comment: Sup-t-norm convolution para la generalización lo menciono?? Y eso de la convolución es útil para algo más?
Response: Yes, the sup-T-norm convolution is central to this generalization. It allows for more flexibility in modeling how the membership degrees of $A$ and $B$ combine. Different t-norms represent different logical interpretations of "conjunction" or "aggregation." For example, $T(a,b) = \min(a,b)$ (Zadeh's original), $T(a,b) = ab$ (product t-norm), or $T(a,b) = \max(0, a+b-1)$ (Lukasiewicz t-norm) each lead to different fuzzy arithmetic operations.}

The generalized theorem states the condition for the equality:
\begin{equation} \label{eq:nguyen_generalized}
[f(A, B)]_\alpha = \bigcup_{T(\xi, \eta) \ge \alpha} f([A]^\xi, [B]^\eta), \quad \alpha \in (0, 1]
\end{equation}
(Note: $[A]^\xi$ and $A_\xi$ are often used interchangeably for $\alpha$-level sets.)

It's important to see that if $T(x,y) = \min(x,y)$, then $T(\xi, \eta) \ge \alpha$ implies $\xi \ge \alpha$ and $\eta \ge \alpha$. In this case, the union $\bigcup_{\min(\xi, \eta) \ge \alpha} f([A]^\xi, [B]^\eta)$ simplifies. Since $f([A]^\xi, [B]^\eta) \subseteq f([A]^\alpha, [B]^\alpha)$ for $\xi, \eta \ge \alpha$ (due to the nesting property of $\alpha$-cuts and assuming $f$ behaves well with set inclusion, e.g., for interval operations), the largest set in the union is $f([A]^\alpha, [B]^\alpha)$. Thus, Equation \eqref{eq:nguyen_generalized} reduces to:
\[
[f(A, B)]_\alpha = f([A]^\alpha, [B]^\alpha)
\]
which is precisely Nguyen's original second theorem.

Fullér and Keresztfalvi provide the following key results \cite[Thms. 1.9.1, 1.9.2]{FULLER2}:
\begin{theorem}
    Let $f: X \times Y \to Z$ be a two-place function, $A \in \mathcal{F}(X)$, $B \in \mathcal{F}(Y)$, and $T$ be a t-norm. A necessary and sufficient condition for the equality \eqref{eq:nguyen_generalized} to hold is that for each $z \in Z$, the supremum $\sup_{f(x,y)=z} T(A(x), B(y))$ is attained.
\end{theorem}

\begin{theorem}
    If $f: X \times Y \to Z$ is continuous, the t-norm $T$ is upper semicontinuous (u.s.c.), and $A, B$ are fuzzy subsets with u.s.c., compactly-supported membership functions (denoted $A \in \mathcal{F}(X, \mathcal{K}), B \in \mathcal{F}(Y, \mathcal{K})$), then the equality \eqref{eq:nguyen_generalized} holds.
    (Here $X, Y, Z$ are locally compact topological spaces).
\end{theorem}
% \begin{proof}
%     (Intuition for Theorem 2) The conditions (continuous $f$, u.s.c. $T$, and $A, B$ being u.s.c. with compact support) ensure that the function $\varphi(x,y) = T(A(x), B(y))$ is u.s.c. and the domain over which the supremum is taken, $f^{-1}(z) \cap (\text{supp} A \times \text{supp} B)$, is compact. An u.s.c. function on a compact set attains its maximum. This guarantees that the condition of the first theorem (supremum is attained) is met.
% \end{proof}
This generalization is significant as it extends the computational convenience of Nguyen's theorem to a broader class of fuzzy operations defined by various t-norms, which are common in fuzzy logic and systems.
The examples provided in \cite[p. 35]{FULLER2} illustrate how the $\alpha$-cuts of $f(A,B)$ can be generated for different t-norms like the weak t-norm $T_W$, product t-norm $T_P(x,y)=xy$, and Lukasiewicz t-norm $T_L(x,y) = \max(0, x+y-1)$.
For $T_P(x,y) = xy$, the formula becomes:
\[ [f(A, B)]_\alpha = \bigcup_{\xi \cdot \eta \ge \alpha} f([A]^\xi, [B]^\eta) = \bigcup_{\xi \in [\alpha,1]} f([A]^\xi, [B]^{\alpha/\xi}) \]
And for $T_L(x,y) = \max(0, x+y-1)$, it becomes:
\[ [f(A, B)]_\alpha = \bigcup_{\max(0,\xi+\eta-1) \ge \alpha} f([A]^\xi, [B]^\eta) = \bigcup_{\xi \in [\alpha,1]} f([A]^\xi, [B]^{\alpha+1-\xi}) \]
(assuming $\alpha+1-\xi \le 1$). These specific forms are useful for practical computations with different t-norm based arithmetic.












\subsection{Fuzzy Arithmetic}
\label{sec:fuzzy_arithmetic}

Building upon the extension principle, arithmetic operations for fuzzy numbers can be defined. These are very useful for defining and solving fuzzy linear programming problems where crisp constraint may be relaxed with fuzzy ones or for aggregating different criteria represented by fuzzy numbers. Their most common implementation is using the minimum t-norm and triangular numbers.

\paragraph{Sup-Min Based Arithmetic for LR-Fuzzy Numbers}
Let $\tilde{A} = (a_1, a_2, \alpha_L, \alpha_R)_{LR}$ and $\tilde{B} = (b_1, b_2, \beta_L, \beta_R)_{LR}$ be two fuzzy numbers of LR-type. Using the sup-min extension principle, the following operational rules can be derived \cite[p.16]{FULLER2}.

\begin{proposition}[Addition and Subtraction of LR-Fuzzy Numbers]
\label{prop:lr_add_sub}
The sum and difference of $\tilde{A}$ and $\tilde{B}$ are given by:
\begin{align}
\tilde{A} \oplus \tilde{B} &= (a_1+b_1, a_2+b_2, \alpha_L+\beta_L, \alpha_R+\beta_R)_{LR} \\
\tilde{A} \ominus \tilde{B} &= (a_1-b_2, a_2-b_1, \alpha_L+\beta_R, \alpha_R+\beta_L)_{LR}
\end{align}
\end{proposition}

\begin{proposition}[Scalar Multiplication of LR-Fuzzy Numbers]
\label{prop:lr_scalar_mult}
For a real number $\lambda \in \mathbb{R}$, the scalar multiplication $\lambda \odot \tilde{A}$ is:
\begin{equation}
\lambda \odot \tilde{A} =
\begin{cases}
(\lambda a_1, \lambda a_2, \lambda\alpha_L, \lambda\alpha_R)_{LR} & \text{if } \lambda \ge 0 \\
(\lambda a_2, \lambda a_1, |\lambda|\alpha_R, |\lambda|\alpha_L)_{LR} & \text{if } \lambda < 0
\end{cases}
\end{equation}
If $\tilde{A}$ is a quasi-triangular fuzzy number, i.e., $a_1=a_2=a$, denoted $\tilde{A}=(a, \alpha_L, \alpha_R)_{LR}$, then for $\lambda < 0$:
\begin{equation}
\lambda \odot \tilde{A} = (\lambda a, \lambda a, |\lambda|\alpha_R, |\lambda|\alpha_L)_{LR} = (\lambda a, |\lambda|\alpha_R, |\lambda|\alpha_L)_{LR}
\end{equation}
\end{proposition}

\begin{remark}
For triangular fuzzy numbers, which are a special case of LR-fuzzy numbers where $L(x)=R(x)=\max(0, 1-x)$:
\begin{itemize}
    \item If $\tilde{A} = (a, \alpha_L, \alpha_R)$ (peak $a$, left spread $\alpha_L$, right spread $\alpha_R$) and $\tilde{B} = (b, \beta_L, \beta_R)$, then:
    \begin{align*}
    \tilde{A} \oplus \tilde{B} &= (a+b, \alpha_L+\beta_L, \alpha_R+\beta_R) \\
    \tilde{A} \ominus \tilde{B} &= (a-b, \alpha_L+\beta_R, \alpha_R+\beta_L)
    \end{align*}
    \item If $\tilde{A} = (a, \alpha)$ (symmetric triangular, peak $a$, spread $\alpha$) and $\tilde{B} = (b, \beta)$, then:
    \begin{align*}
    \tilde{A} \oplus \tilde{B} &= (a+b, \alpha+\beta) \\
    \tilde{A} \ominus \tilde{B} &= (a-b, \alpha+\beta) \\
    \lambda \odot \tilde{A} &= (\lambda a, |\lambda|\alpha)
    \end{align*}
\end{itemize}
These simplified forms are widely used in practical applications \cite[p.17]{FULLER2}.
\end{remark}



The product of two fuzzy numbers $\tilde{A}$ and $\tilde{B}$ using the sup-min extension principle is defined as:
\begin{equation}
\mu_{\tilde{A} \otimes \tilde{B}}(z) = \sup_{x \cdot y = z} \min(\mu_{\tilde{A}}(x), \mu_{\tilde{B}}(y))
\end{equation}
Unlike addition and subtraction, the product of two LR-fuzzy numbers is not, in general, an LR-fuzzy number itself, and simple formulas for its parameters are usually not available.
However, computations can often be simplified by working with $\alpha$-cuts and usign the Nguyen's theorem \cite[Thm. 1.3.2]{FULLER2}:

\begin{theorem}
\label{thm:alpha_cut_product}
Let $f: \mathbb{R} \times \mathbb{R} \to \mathbb{R}$ be a continuous function, and let $\tilde{A}$ and $\tilde{B}$ be fuzzy numbers. Then the $\alpha$-cut of $f(\tilde{A}, \tilde{B})$ is given by:
\begin{equation}
[f(\tilde{A}, \tilde{B})]^\alpha = f([\tilde{A}]^\alpha, [\tilde{B}]^\alpha) = \{f(x,y) \mid x \in [\tilde{A}]^\alpha, y \in [\tilde{B}]^\alpha \}
\end{equation}
\end{theorem}

For the product operation, $f(x,y) = xy$. Thus, $[\tilde{A} \otimes \tilde{B}]^\alpha = [\tilde{A}]^\alpha \cdot [\tilde{B}]^\alpha$.
If $\tilde{A}$ and $\tilde{B}$ are non-negative fuzzy numbers (i.e., their supports are in $\mathbb{R}^+_0$), and their $\alpha$-cuts are $[\tilde{A}]^\alpha = [a_1(\alpha), a_2(\alpha)]$ and $[\tilde{B}]^\alpha = [b_1(\alpha), b_2(\alpha)]$, then the $\alpha$-cut of their product is:
\begin{equation}
[\tilde{A} \otimes \tilde{B}]^\alpha = [a_1(\alpha)b_1(\alpha), a_2(\alpha)b_2(\alpha)]
\end{equation}
This holds if and only if $\tilde{A}$ and $\tilde{B}$ are both non-negative \cite[p.18]{FULLER2}.
If one or both fuzzy numbers can take negative values, the interval multiplication for the $\alpha$-cuts becomes more complex:
\begin{equation}
[\tilde{A}]^\alpha \cdot [\tilde{B}]^\alpha = [\min(P), \max(P)]
\end{equation}
where $P = \{a_1(\alpha)b_1(\alpha), a_1(\alpha)b_2(\alpha), a_2(\alpha)b_1(\alpha), a_2(\alpha)b_2(\alpha)\}$.



\paragraph{T-norm Based Addition of Fuzzy Numbers}
When employing the sup-T extension principle with an Archimedean t-norm $T$ (having an additive generator $f$), the sum of fuzzy numbers takes a specific form.
Let $\tilde{A}_i = (a_i, b_i, \alpha, \beta)_{LR}$, $i=1, \dots, n$, be $n$ fuzzy numbers of LR-type with common spreads $\alpha, \beta$.
Let $f$ be the additive generator of $T$, and $L,R$ be the shape functions. The following theorem \cite[Thm. 1.7.1]{FULLER2}:

\begin{theorem}
\label{thm:tnorm_addition_lr}
Let $T$ be an Archimedean t-norm with additive generator $f$. Let $\tilde{a}_i = (a_i, b_i, \alpha, \beta)_{LR}$, $i=1, \dots, n$, be fuzzy numbers of LR-type. If $L$ and $R$ are twice differentiable, concave functions, and $f$ is a twice differentiable, strictly convex function, then the membership function of the T-sum $\tilde{A}_n = \tilde{a}_1 \oplus_T \dots \oplus_T \tilde{a}_n$ is given by:
\begin{equation}
\mu_{\tilde{A}_n}(z) =
\begin{cases}
1 & \text{if } A_n \le z \le B_n \\
f^{[-1]}\left( n \cdot f\left(L\left(\frac{A_n-z}{n\alpha}\right)\right) \right) & \text{if } A_n - n\alpha \le z < A_n \\
f^{[-1]}\left( n \cdot f\left(R\left(\frac{z-B_n}{n\beta}\right)\right) \right) & \text{if } B_n < z \le B_n + n\beta \\
0 & \text{otherwise}
\end{cases}
\end{equation}
where $A_n = \sum_{i=1}^n a_i$, $B_n = \sum_{i=1}^n b_i$, and $f^{[-1]}$ is the pseudo-inverse of $f$.
\end{theorem}
\begin{remark}
This theorem provides a computationally efficient way to calculate the t-norm based sum of LR-fuzzy numbers under specific conditions. It has been generalized for varying spreads and different conditions on $f, L, R$ by several authors \cite[p.29]{FULLER2}. For example, it remains valid if $f$ is convex and $L,R$ are concave.
\end{remark}
The product of two fuzzy numbers $\tilde{A}$ and $\tilde{B}$ using the sup-T extension principle, where $T$ is a t-norm, is defined as \cite[p.20]{FULLER2}:
\begin{equation}
\mu_{\tilde{A} \otimes_T \tilde{B}}(z) = \sup_{x \cdot y = z} T(\mu_{\tilde{A}}(x), \mu_{\tilde{B}}(y))
\end{equation}
Similar to the sup-min product, closed-form expressions for the resulting membership function are not always straightforward for general LR-fuzzy numbers. However, the functional relationship described next provides significant insight for specific cases involving identical positive LR-fuzzy numbers.


\paragraph{Functional Relationship Between T-norm Based Addition and Multiplication}
A remarkable functional relationship exists between t-norm based addition and t-norm based multiplication for positive LR-fuzzy numbers of the same type. The following theorem \cite[Thm. 1.8.1]{FULLER2}:

\begin{theorem}
\label{thm:functional_relationship_add_mult}
Let $T$ be an Archimedean t-norm with an additive generator $f$. Let $\tilde{M}_i = \tilde{M} = (a,b,\alpha,\beta)_{LR}$ for $i=1,\dots,n$ be $n$ identical positive fuzzy numbers of LR-type (i.e., their support is in $\mathbb{R}^+$). If $L$ and $R$ are twice differentiable, concave functions, and $f$ is a twice differentiable, strictly convex function, then for $z \ge 0$:
\begin{equation}
\mu_{(\tilde{M}_1 \oplus_T \dots \oplus_T \tilde{M}_n)}(nz) = \mu_{(\tilde{M}_1 \otimes_T \dots \otimes_T \tilde{M}_n)}(z^n) = f^{[-1]}(n \cdot f(\mu_{\tilde{M}}(z)))
\end{equation}
where $\oplus_T$ denotes t-norm based addition and $\otimes_T$ denotes t-norm based multiplication.
\end{theorem}

\begin{remark}
This theorem establishes that the membership function of the sum of $n$ identical fuzzy numbers, when evaluated at $nz$, is equal to the membership function of the product of these $n$ fuzzy numbers evaluated at $z^n$. Both are determined by scaling the transformed membership function of the original fuzzy number $\tilde{M}$ using its generator $f$. This relationship is particularly useful in analyzing the behavior of aggregated fuzzy quantities. An immediate consequence is that if $e^*_n(\tilde{M}) = (\tilde{M} \oplus_T \dots \oplus_T \tilde{M})/n$ (where division by $n$ is scalar multiplication by $1/n$), then $\lim_{n\to\infty} e^*_n(\tilde{M})$ converges to the core $[a,b]$ of $\tilde{M}$ \cite{FULLER2}{p.33}.
\end{remark}

\signal{
\subsection{Metrics for fuzzy numbers}

\subsection{Defuzzification and Fuzzy Linear Programming Problems}}
%Most methods used for solving such problems are based on ranking functions, alfa-cuts, using duality results or penalty functions. In these methods, authors deal with crisp formulations of the fuzzy problems. Recently, some heuristic algorithms have also been proposed. In these methods, some authors solve the fuzzy problem directly, while others solve the crisp problems approximately.

\section{Fuzzy Measures}
\signal{Possibility is normalized so that 1 membership is attained at some point. And it models the compatibility of 2 states. That is of being 1.80 height and being tall states.

Probability Models aleatoric uncertainty y Possibility models epistemic uncertainty?\\

Probability-Possibility Transformations:
Under certain conditions, one can associate a family of probability distributions with a given possibility distribution. For instance, the possibility distribution can be seen as an upper envelope for a set of probability measures that are consistent with the available imprecise information. Such transformations (e.g., the Dubois-Prade method) allow one to move between the two representations, albeit at the cost of introducing conservatism or ambiguity.\\


Bayesian Models for modeling both uncertainties:\\

Epistemic Uncertainty:

Bayesian models explicitly account for uncertainty about the model parameters by placing a prior over them and computing a posterior after observing data. This kind of uncertainty reflects our lack of knowledge due to limited data and can be reduced by gathering more information.\\

Aleatoric Uncertainty:

This type of uncertainty represents the inherent noise in the data itself (for example, measurement error). In Bayesian modeling, aleatoric uncertainty is often incorporated into the likelihood function. It is considered irreducible since it stems from the randomness in the data generation process.\\

Osea que a qué nos referimos con epistemologic uncertainty? A que nos falta info (data) o a que tenemos toda la info que se puede tener y no podemos reducir ese gap de incetidumbre porque es un problema de definición imprecisa. Tengo que diferenciar entre vagueza y epistemologico entonces?

\subsection{Probability-Possibility Transformations}
%https://www.researchgate.net/publication/2743789_On_PossibilityProbability_Transformations
From the paper: \cite{Dubois1993}: 
it is recalled in this paper that the probabilistic representations and the possibilisticones are not just two equivalent representations of uncertainty

The possibilisticrepresentation is weaker because it explicitly handles imprecision (e.g. incompleteknowledge) and because possibility measures are based on an \textbf{ordering structure} rather than an \textbf{additive one}.

the principle of insufficient reason from possibility toprobability, and the principle of maximum specificity from probability topossibility.

Mencionar: consistency principle, principle of insufficient reason and principle of maximum specificity
}


\subsection{Other uncertainty models}
\signal{Non-additive probabilities

Dempster-shafter evidence theory is equivalent to necessity and possibility
}



\signal{Y HABLAR TAMBIÉN DE PLAUSIBILITY MEASURE Y CÓMO ESO SE RELACIONA CON EL LAS OTRAS MEASURES. MIRA EN Joseph Y. Halpern - Reasoning about Uncertainty-The MIT Press (2003)}




Classical measure theory, particularly probability theory, relies heavily on the principle of additivity. This principle dictates how the measure of a union of disjoint sets relates to the measures of the individual sets. While powerful, this additivity can be restrictive in scenarios where interactions, redundancies, or synergies between elements are significant.

\begin{definition}[Fuzzy Measure (Capacity)]
Let $\Omega$ be a universal set. A \textbf{fuzzy measure} (or capacity) $\mu$ is a set function
\[ \mu: \powerset{\Omega} \to [0, 1] \]
that assigns a value to each subset of $\Omega$ (often called an event or criterion) such that:
\begin{enumerate}
    \item \textbf{Boundary Conditions:}
    \begin{itemize}
        \item $\mu(\emptyset) = 0$ (the measure of the empty set is zero).
        \item $\mu(\Omega) = 1$ (the measure of the universal set is one).
    \end{itemize}
    \item \textbf{Monotonicity:} For any $A, B \subseteq \Omega$, if $A \subseteq B$, then $\mu(A) \le \mu(B)$.
\end{enumerate}
\end{definition}

The crucial departure from classical probability (in its general form) is the absence of a strict additivity requirement for disjoint sets. This allows fuzzy measures to model situations where, for example, the combined importance of two criteria $A$ and $B$ might be greater than the sum of their individual importances (synergy) or less (redundancy). The value $\mu(A)$ quantifies the weight, importance, degree of belief, or capacity associated with the subset $A$.

\subsection{Representations of Fuzzy Measures}
A fuzzy measure $\mu$ is fundamentally defined by its $2^{|\Omega|}$ values assigned to all subsets of $\Omega$. This is its standard representation. However, alternative representations exist that can be more convenient for certain analyses or computations.
\begin{definition}[Möbius Transform]
For a fuzzy measure $\mu$ on a finite set $\Omega$, its \textbf{Möbius transform} $m: \powerset{\Omega} \to \mathbb{R}$ is another set function defined as:
\[ m(A) = \sum_{B \subseteq A} (-1)^{|A \setminus B|} \mu(B), \quad \forall A \subseteq \Omega \]
Conversely, the fuzzy measure can be recovered from its Möbius transform:
\[ \mu(A) = \sum_{B \subseteq A} m(B), \quad \forall A \subseteq \Omega \]
\end{definition}
The Möbius transform provides an alternative way to characterize the fuzzy measure and is particularly useful in analyzing interactions between elements, a topic explored when considering how these measures are used in aggregation processes. For very small universal sets, a Hasse diagram can visually represent the fuzzy measure values for all subsets.

\subsection{Types of Fuzzy Measures}
The general definition of a fuzzy measure is quite broad. Within this framework, numerous specific types of fuzzy measures have been defined to capture different kinds of uncertainty, non-additivity, or interaction patterns. Two of the most prominent are probability measures and possibility measures, which will be the focus of the next section. Other types, such as belief measures or $\lambda$-measures, also exist and offer different ways to model non-additive information. The choice of fuzzy measure type is crucial when these measures are employed in more complex frameworks, such as for aggregation or integration, which are important applications discussed in subsequent chapters.

Fuzzy measures are classified based on several criteria. These include restrictions on their output values (e.g., 0-1 or Boolean measures), their behavior concerning additivity (additive, subadditive, superadditive) and modularity (submodular, supermodular, modular), which describe interactions between set contributions. Symmetry, where the measure depends only on set cardinality, is another criterion. Decomposability defines how the measure of a union of disjoint sets is derived from the measures of its components, with $\lambda$-fuzzy measures and possibility measures being key examples. Measures can also be classified by their duality properties, their relationship to $k$-monotonicity or $k$-alternativity (leading to belief and plausibility measures), or by their construction method, such as distorted probability measures.

\subsection*{Special Mention: Probability and Possibility}

Within this classification framework:

\textbf{Probability measures} are a cornerstone. They are defined as \textit{additive fuzzy measures} (Definition 2.4), meaning $\mu(A \cup B) = \mu(A) + \mu(B)$ for disjoint sets $A$ and $B$, and $\mu(N)=1$. Due to their additivity, they are also \textit{modular} (Note 2.6), \textit{self-dual} (Note 2.1 implies this as additive measures are self-dual), and simultaneously \textit{belief measures} and \textit{plausibility measures} (Note 2.17). They are also decomposable with respect to the Łukasiewicz t-conorm (Note 2.25).

\textbf{Possibility measures} (Pos) are characterized by the property $\text{Pos}(A \cup B) = \max\{\text{Pos}(A), \text{Pos}(B)\}$ (Definition 2.14). This makes them a specific type of \textit{decomposable fuzzy measure} (Note 2.26, using the maximum t-conorm). They are inherently \textit{subadditive} (page 7, bottom paragraph) and are a type of \textit{plausibility measure} (Note 2.18). Their duals are \textit{necessity measures}, which are a type of \textit{belief measure}.

\begin{figure}[h!]
\centering
\resizebox{\textwidth}{!}{%
\begin{tikzpicture}[
    % Styles
    basic/.style={draw, text width=3cm, text centered, rounded corners, minimum height=1cm, drop shadow},
    maincat/.style={basic, fill=LightSkyBlue, font=\bfseries},
    subcat/.style={basic, fill=PaleGreen, text width=2.8cm},
    highlight/.style={basic, fill=Gold, font=\bfseries, text width=3.2cm, minimum height=1.2cm},
    detail/.style={font=\tiny, text width=3cm, text centered},
    link/.style={-{Stealth[length=2.5mm, width=2mm]}, thick},
    dashedlink/.style={link, dashed}
]

% Nodes
\node[maincat] (fm) {Fuzzy Measures};

% Branch 1: Additivity/Modularity
\node[subcat, below left=1.5cm and 1.5cm of fm] (additive) {Additive Measures};
\node[highlight, below=1cm of additive] (prob) {Probability Measures};
\node[detail, below=0.2cm of prob] (probdetail) {(Modular, Self-Dual, Belief \& Plausibility, $S_L$-Decomposable)};
\path[link] (additive) edge (prob);

% Branch 2: Decomposability
\node[subcat, below right=0.8cm and 2cm of fm] (decomposable) {Decomposable Measures};
\node[highlight, below left=1cm and -0.2cm of decomposable] (poss) {Possibility Measures};
\node[detail, below=0.2cm of poss] (possdetail) {(max-decomposable, Plausibility, Subadditive)};
\node[subcat, fill=LightPink, below right=1cm and -0.2cm of decomposable] (lambda) {$\lambda$-Fuzzy Measures};
\node[detail, below=0.2cm of lambda] (lambdadetail) {($\lambda=0 \implies$ Additive, \\ $\lambda<0 \implies$ Plausibility, \\ $\lambda>0 \implies$ Belief)};
\path[link] (decomposable) edge (poss);
\path[link] (decomposable) edge (lambda);

% Branch 3: Belief/Plausibility (k-order properties)
\node[subcat, below left=3.5cm and -2.5cm of fm] (belief) {Belief Measures};
\node[subcat, below right=3.5cm and -2.5cm of fm] (plaus) {Plausibility Measures};
\node[subcat, fill=Lavender, below=1cm of belief] (nec) {Necessity Measures};
\node[detail, below=0.2cm of nec] (necdetail) {(Dual of Possibility)};
\path[link] (belief) edge (nec);

% Cross-links (dashed)
\path[dashedlink, blue] (prob.east) edge[bend left=15] ($(belief.west)+(0,0.2)$);
\path[dashedlink, blue] (prob.east) edge[bend right=15] ($(plaus.west)-(0,0.2)$);
\path[dashedlink, red] (poss.east) edge[bend left=10] (plaus.west);
\path[dashedlink, green] (nec.east) edge[bend right=10] (belief.west);

% Connect main categories to root
\path[link] (fm) edge (additive);
\path[link] (fm) edge (decomposable);
\path (fm) edge[link] node[midway, above, sloped, font=\scriptsize] {k-order props.} (belief);
\path (fm) edge[link] node[midway, above, sloped, font=\scriptsize] {k-order props.} (plaus);

% Other less central types (can be added for completeness if space allows)
\node[subcat, fill=Wheat, above right=0.5cm and 1.5cm of fm, text width=2.5cm] (other) {Other types:};
\node[detail, below=0.2cm of other, text width=2.5cm] (otherdetail) {0-1 (Boolean)\\Symmetric\\Sub/Supermodular\\Distorted Prob.};
\path[link] (fm) edge (other);


\end{tikzpicture}%
}
\caption{Simplified classification map of Fuzzy Measures, highlighting Probability and Possibility measures and their key properties/relationships.}
\label{fig:fuzzy_classification}
\end{figure}

\section{Probability and Possibility Measures}
Probability and possibility measures are two distinct yet related formalisms for quantifying uncertainty, both falling under the umbrella of fuzzy measures.

\begin{definition}[Probability Measure]
A \textbf{probability measure} $P$ is a fuzzy measure that additionally satisfies the axiom of \textbf{finite additivity} for disjoint sets:
For any two disjoint sets $A, B \subseteq \Omega$ (i.e., $A \cap B = \emptyset$):
\[ P(A \cup B) = P(A) + P(B) \]
\end{definition}
A probability measure quantifies the likelihood or frequency of an event occurring, or the degree of belief that a proposition is true. The sum of probabilities for a complete set of mutually exclusive elementary events is 1.

\begin{definition}[Possibility Measure]
A \textbf{possibility measure} $\Pi$, pioneered by Zadeh \cite{Zadeh1978}, is a fuzzy measure characterized by the axiom of \textbf{maxitivity} (or sup-additivity):
For any two sets $A, B \subseteq \Omega$:
\[ \Pi(A \cup B) = \max(\Pi(A), \Pi(B)) \]
\end{definition}
A possibility measure quantifies the degree to which an event is considered possible, plausible, or compatible with available knowledge. It is often derived from a \textbf{possibility distribution} $\pi: \Omega \to [0, 1]$, where $\pi(x)$ is the possibility that a variable takes the value $x$. Then, for any $A \subseteq \Omega$, $\Pi(A) = \sup_{x \in A} \pi(x)$. Normalization typically requires $\sup_{x \in \Omega} \pi(x) = 1$, which implies $\Pi(\Omega) = 1$.

\subsection{Conceptual Differences: An Example}
Consider a standard six-sided die.
\begin{itemize}
    \item \textbf{Probability:} If the die is fair, the probability of rolling any specific number $x \in \{1, ..., 6\}$ is $P(\{x\}) = 1/6$.
    The probability of rolling an even number is $P(\text{even}) = P(\{2\}) + P(\{4\}) + P(\{6\}) = 1/6 + 1/6 + 1/6 = 3/6 = 1/2$.
    Similarly, $P(\text{odd}) = 1/2$. Note that $P(\text{even}) + P(\text{odd}) = 1$.
    Also, $P(\text{even} \cup \text{odd}) = P(\Omega) = 1$.

    \item \textbf{Possibility:} If our knowledge only states that "it is possible to roll any number from 1 to 6," we might assign a possibility distribution $\pi(x) = 1$ for all $x \in \{1, ..., 6\}$.
    Then, the possibility of rolling an even number is $\Pi(\text{even}) = \max(\pi(2), \pi(4), \pi(6)) = \max(1,1,1) = 1$.
    Similarly, $\Pi(\text{odd}) = \max(\pi(1), \pi(3), \pi(5)) = 1$.
    Here, $\Pi(\text{even}) = 1$ means "it is fully possible to roll an even number." It does not mean it is certain.
    Crucially, it is not required that $\Pi(A) + \Pi(\overline{A}) = 1$. In this example, $\Pi(\text{even}) = 1$ and $\Pi(\overline{\text{even}}) = \Pi(\text{odd}) = 1$. This reflects a state of incomplete information: it's fully possible to be even, and fully possible to be odd.
    Also, $\Pi(\text{even} \cup \text{odd}) = \max(\Pi(\text{even}), \Pi(\text{odd})) = \max(1,1) = 1$.
\end{itemize}
Associated with every possibility measure is a dual \textbf{necessity measure} $N$, defined as $N(A) = 1 - \Pi(\overline{A})$. $N(A)$ quantifies the degree to which event $A$ is certainly true or necessarily implied. For our dice example, $N(\text{even}) = 1 - \Pi(\text{odd}) = 1 - 1 = 0$. It is not necessary to roll an even number.

\subsection{Probability-Possibility Transformations}
Despite their distinct conceptual foundations, probability and possibility measures are not entirely disconnected. The relationship $N(A) \le P(A) \le \Pi(A)$ suggests an ordering and provides a basis for transformations.

\subsubsection{Rationale and Core Assumption}
Transformations between these measures are often sought for practical reasons:
\begin{itemize}
    \item We might have probabilistic information but need a possibilistic model for certain types of reasoning (e.g., reasoning about feasibility under uncertainty).
    \item We might have possibilistic information (e.g., from expert elicitation of what's feasible) but need a probability distribution for decision-making processes that require expected utility calculations.
\end{itemize}
The core assumption underlying many transformations is that \textbf{possibility acts as an upper bound for probability}: $P(A) \le \Pi(A)$ for all $A \subseteq \Omega$. This implies that what is probable must first be possible. An event cannot be likely if it's not even deemed possible to some corresponding degree. The set of all probability measures $P$ consistent with a given possibility measure $\Pi$ is called a credal set.

\subsubsection{Transformations from Probability to Possibility (P$\to\Pi$)}
The goal here is to summarize or bound a precise probability distribution $p$ (defined on elementary events) with a possibility distribution $\pi$ (which induces $\Pi$) such that $p(x) \le \pi(x)$ for all $x \in \Omega$, and often, to preserve the ranking of probabilities.
One common method, attributed to Dubois and Prade \cite{DuboisPrade1983}, is the \textbf{least-specific dominating possibility distribution}.
Given a probability distribution $p=(p_1, \dots, p_n)$ over a finite universe $\Omega = \{\omega_1, \dots, \omega_n\}$, with elements ordered such that $p_1 \ge p_2 \ge \dots \ge p_n$ (where $p_k$ is the probability of $\omega_k$). The possibility degree for $\omega_k$ is defined based on cumulative probabilities:
\[ \pi(\omega_k) = \sum_{i=k}^{n} p_i \]
This is typically normalized so that $\max_k \pi(\omega_k) = \pi(\omega_1) = 1$.
\textbf{Intuition:} The most probable element is fully possible. The possibility of less probable elements reflects the cumulative probability of them and all elements less probable than them. This ensures the order-preserving property and consistency with $P(A) \le \Pi(A)$. Such transformations inherently involve a loss of information, as the additivity of probability is replaced by the weaker maxitivity of possibility.

\subsubsection{Transformations from Possibility to Probability ($\Pi\to$P)}
The goal is to select a single probability distribution $p$ from the credal set $\mathcal{P}(\Pi)$ defined by a possibility distribution $\pi$. This often involves an "information filling-in" step, guided by principles like maximum entropy or insufficient reason.
A well-known example is the \textbf{pignistic transform} (context of belief functions, of which possibility measures are a special case). For a consonant belief structure (like possibility), given distinct possibility levels $1 = \alpha_0 > \alpha_1 > \dots > \alpha_m = 0$, we define $\alpha$-cuts $E_j = \{\omega : \pi(\omega) \ge \alpha_j\}$. The probability mass $(\alpha_{j-1} - \alpha_j)$ that "drops" between two successive possibility levels is distributed uniformly among the elements in the narrower (more possible) set $E_{j-1}$:
\[ p(\omega) = \sum_{j=1}^{m} \frac{\alpha_{j-1} - \alpha_j}{|E_{j-1}|} \mathbf{1}_{\{\omega \in E_{j-1}\}} \]
\textbf{Intuition:} Elements that are equally highly possible share the available probability mass assigned to that level of possibility. This transform selects the probability distribution that is maximally non-committal (maximizes entropy) within the constraints imposed by the possibility measure. These transformations add assumptions to select one specific probability distribution from many consistent ones.

\subsubsection{Limitations and Assumptions of Transformations}
It is crucial to understand that transformations are not universally applicable without careful consideration of their underlying assumptions:
\begin{enumerate}
    \item \textbf{The $P(A) \le \Pi(A)$ Postulate:} This is a fundamental modeling choice, not an inherent law. It is justified if the possibility measure truly represents available information about the upper bounds of potential probabilities (e.g., $\Pi$ derived from imprecise observations constraining $P$). It can be problematic if $P$ and $\Pi$ are derived from entirely independent sources or model fundamentally different aspects of a problem.
    \item \textbf{Information Loss/Gain:} Transforming $P \to \Pi$ loses the precision of additivity. Transforming $\Pi \to P$ involves making assumptions (like maximum entropy or insufficient reason) to pick one $P$ out of a set of possibilities. The choice of transformation method itself is an assumption.
    \item \textbf{Semantic Alignment:} Transformations are most meaningful when probability and possibility are indeed modeling different facets of the \textit{same underlying uncertainty}. If they model completely unrelated concepts, applying formal transformations can be a category mistake, as the numbers, despite being in $[0, 1]$, have entirely different semantics.
\end{enumerate}
\section{Fuzzy Logic}\label{sec:fuzzy_logic}
\subsection{Fuzzy implications}
S-implications
R-implications
t-norm implications
\signal{$(\fuzzy{X},\cup , \cap , \lnot)$ is a complete, completely distributive, lattice with an involution. This extends the boolean algebra.}

\signal{The notion of equality is replaced by a graded relation (often measured via the biresiduum $\leftrightarrow$)}

\signal{Fuzzy description logics explained %https://www.umbertostraccia.it/cs/download/papers/KES09/KES09.pdf

The Lukasiewicz t-conorm is closely related to the basic binary operation of multi-valued
algebras. Additionally, t-norms and t-conorms form examples of aggregation operators. They
play a significant role in the axiomatic definition of the concept of triangular norm-based measure
and, in particular, of the concept of probability of fuzzy events; the Frank family of t-norms and
t-conorms plays a particular role [6]. 

%https://cake.fiu.edu/Publications/Ngan+al-18-LC.Logic_Connectives_of_Complex_Fuzzy_Sets_ROMJIST_downloaded.pdf 

It should be mentioned that t-norms overlap with copulas [3, 24]: commutative associative
copulas are t-norms; t-norms which satisfy the 1-Lipschitz condition are copulas. Some families
of t-norms are known as families of copulas under different names

A new t-norm was introduced and applied in Active Learning Method (ALM) by Kiaei et
al. [20]. The original operators of ALM were presented, along with the Ink Drop Spread and
the Center of Gravity operators, and two basic morphological operators. The obtained results
show that new operators have overcome several of the disadvantages of the original operators.
The new operators were applied well in ALM. In that paper, aggregations of fuzzy relations
using aggregation functions have been considered. This has been performed by determining
certain conditions expressed in computational formulas and deploying t-norms and t-conorms
properties. Recently, T-operators have been used to combine criteria in MCDM.}

\signal{
    Lukasiewicz logic algebraic structure and formal implications thanks to the residual property.\\

    Gödel Logic uses 
    $a \otimes b = \min(a, b)$, which tends to produce conservative scores dominated by the weakest criterion. While useful for risk-averse scenarios, it fails to differentiate alternatives when all criteria are partially satisfied.

    Product Logic employs 
    $a \otimes b = a \cdot b$, amplifying the impact of low-scoring criteria. This can lead to premature elimination of alternatives with one poor attribute.

    Łukasiewicz Logic's additive t-norm balances compensation between criteria, allowing alternatives to offset weaknesses in one dimension with strengths in others—a critical feature for complex trade-off analysis.
}

\section{Possibilistic Correlation}
\section{Limitations of Fuzzy Sets and other alternatives}
\signal{Aquí mencionar por encima alternativas sin entrar en mucho detalle: type-1 (pertenencia puntual), type-2 (pertenencia intervalar), complex memberships, intuitionistic, hesitant, multiplicative versions of hesitant and intuitionistic, bipolar and multipolar, fuzzy-rough sets, Neutrosophic and Plithogenic sets, picture fuzzy sets...}