%See Nguyen paper page 7-8 of the pdf and 375-376 of the book.
According to Nguyen \cite{NGUYEN1978}:
 
\say{Interval analysis deals with closed bounded intervals (complex convex sets of $\R$) as an extension of numbers. Fuzzy numbers can be regarded as 
an extension of closed bounded intervals, [...]} 
\\

\signal{Another way to think about fuzzy numbers is as a special case of fuzzy intervals.}

\signal{Explain that we want to define fuzzy numbers to work with fuzzy attributes later on}


\begin{definition}[Normal Fuzzy Set]
    A fuzzy set $A\in \fuzzy{X}$ is called normal if there exists $x\in X$ such that $A(x)=1$. Otherwise it is called subnormal.
\end{definition}

\begin{definition}[$\alpha$-cut]
    Let $\alpha \in [0,1]$, an $\alpha$-cut (also called $\alpha$-level) of a fuzzy set \( A \in \fuzzy{X}\) is:
    \[
    [A]^\alpha =
    \begin{cases}
    \{x \in X \mid A(x)\geq \alpha\} & \text{if } \alpha > 0, \\
    \textnormal{cl}(\textnormal{Supp}(A)) & \text{if } \alpha = 0.
    \end{cases}
    \]
    where \textit{cl} denotes the closure.
\end{definition}

\begin{remark}
    From the definition of $\alpha$-cut, the nested property states that for
    $\alpha_1, \alpha_2 \in [0,1]$ if $\alpha_1\leq \alpha_2$ then $[A]^{\alpha_2}\subseteq [A]^{\alpha_1}$
\end{remark}

\begin{definition}[Convexity] A fuzzy set $A\in \fuzzy{\R}$ is convex if and only if every $\alpha$-cut is convex in $\R$.
    
\end{definition}

\begin{definition}[Fuzzy Number]
    A fuzzy number is a fuzzy set in the real line, i.e., $A\in \fuzzy{\R}$ such that:\vspace{-0.9em}
    \begin{romanenum}
        \item Normal\vspace{-0.5em}
        \item Convex\vspace{-0.5em}
        \item $\mu_A$ is continuous.\vspace{-0.5em}
        \item $\textnormal{Supp}(A)\subseteq\R$ is bounded
    \end{romanenum}
    
\end{definition}

\begin{proposition}[$\alpha$-cuts are closed intervals]
    Let $A\in \fuzzy{\R}$ be a fuzzy number. Then for every $\alpha \in [0,1]$, the $\alpha$-cut $[A]^\alpha$ is a closed interval in $\R$.
\end{proposition}

\begin{proof}
%1
The fact that $[A]^\alpha$ is an interval follows from the definition of convex subset in $\R$ with the usual topology, which can only be an interval (or a single point).\\
%2
Now we prove that $[A]^\alpha$ is closed. For $\alpha \in (0,1]$, since $\mu_A$ is continuous and $[\alpha, 1]$ is closed in $\R$, the set
\[
[A]^\alpha = \mu_A^{-1}([\alpha, 1])
\]
is closed in $\R$. %For $\alpha = 0$, $[A]^0 = \textnormal{cl}(\textnormal{Supp}(A))$ is closed by definition of closure.
\end{proof}

\begin{notation}{Notation}
    We will denote the $\alpha$-cuts of a fuzzy number $A$ as 
    \[[A]^\alpha=[a_1(\alpha),a_2(\alpha)]\textnormal{ where }\begin{cases}
        a_1(\alpha)=min[A]^\alpha&\\
        a_2(\alpha)=max[A]^\alpha&\\
    \end{cases}\]
\end{notation}

\begin{note}
The condition of bounded support can be relaxed to define \textit{quasi-fuzzy numbers} \signal{(Which properties still hold and which are lost?)}:
$$\textnormal{(iv}_{\textnormal{bis}}\textnormal{) } \lim{t}{\infty}A(t) = 0 \quad \land \quad \lim{t}{-\infty}A(t) = 0$$
\end{note}

The following proposition (mentioned in \cite[p.~3]{FULLER2} as a comment without proof) establishes that the membership function of any fuzzy number can be partitioned into three contiguous intervals: one where it monotonically increases, one where it equals 1, and one where it monotonically decreases. This characterization shows that every fuzzy number can be represented as an LR-fuzzy number (see example \ref{ex:fuzzy_num} for the definition).

\begin{proposition}[Membership function of fuzzy numbers]
    Let $A\in \fuzzy{\R}$ be a fuzzy number, then it satisfies:
    \begin{romanenum}
        \item $\mu_A(t)=0$ outside an interval (denoted by $[a,d]$)\vspace{-0.5em}
        \item $\exists b,c \in \R \mid a\leq b \leq c \leq d$ where $\begin{cases}
            \mu_A\textnormal{ is monotone increasing in }[a,b]\\
            \mu_A\textnormal{ is monotone decreasing in }[b,d]\\
        \end{cases}$\vspace{-0.5em}
        \item $\mu_A(t)=1 \forall t\in [b,c]$
    \end{romanenum}
\end{proposition}


\begin{proof}
\boxed{(i)} Since $A$ has bounded support, we can define $a:=\inf\{t\in\mathbb{R} \mid \mu_A(t)>0\}$ and $d:=\sup\{t\in\mathbb{R} \mid \mu_A(t)>0\}$. Therefore $\mu_A(t)=0$ for all $t\notin[a,d]$. \\

\boxed{(iii)} Since $A$ is normal, we define $b:=\inf\{t\in\mathbb{R} \mid \mu_A(t)=1\}$ and $c:=\sup\{t\in\mathbb{R} \mid \mu_A(t)=1\}$. By continuity and convexity if there $\exists t\in [b,c]$ where $\mu_A(t)<1$, then $\exists \epsilon >0 \mid t\notin [A]^{t+\epsilon}$ is not a closed interval. Therefore we have $\mu_A(t)=1$ for all $t\in[b,c]$. \\

\boxed{(ii)} Since every $\alpha$-cut $[A]^\alpha=[a(\alpha),d(\alpha)]$ is a closed interval. The nested property of $\alpha$-cuts implies $a(\alpha)$ is non-decreasing and $d(\alpha)$ is non-increasing. For any $s,t\in[a,b]$ with $s<t$ and $\mu_A(s)=\alpha$, we have $t\in[A]^\alpha$, so $\mu_A(t)\geq\alpha=\mu_A(s)$. Similarly for $s,t\in[c,d]$ with $s<t$ and $\mu_A(t)=\alpha$, we have $s\in[A]^\alpha$, so $\mu_A(s)\geq\alpha=\mu_A(t)$. Therefore $\mu_A$ is monotone increasing on $[a,b]$ and monotone decreasing on $[c,d]$.
\end{proof}


% Write me a python function to represent the following fuzzy numbers in theree plots in the same figure. I want the letters to be the same as the ones in the definition and I want them all to be in the positive quadrant. Also write the 1 of the membership in the y axis saying that axis is the membership function of A \mu_A. The area of the fuzzy number must be gray and plot also thin lines for the reference values.
\begin{example}\label{ex:fuzzy_num}
    Here are some examples of fuzzy numbers:
    \begin{itemize}
        \item \textbf{Triangular Fuzzy Number:} Defined by a triplet $A\equiv(a, \alpha, \beta)$ where $a$ is the peak and $\alpha$ and $\beta$ the right and left widths respectively. The membership function $\mu_A(x)$ is given by:
        \[
        \mu_A(x) = 
        \begin{cases} 
        1-\frac{a-x}{\alpha} & \text{if } a \leq x < a-\alpha, \\
        1-\frac{x-a}{\beta} & \text{if } a+\beta < x \leq a, \\
        0, & \text{otherwise.}
        \end{cases}
        \]
        
        \item \textbf{Trapezoidal Fuzzy Number:} Defined by a quadruplet $A\equiv(a, b, \alpha, \beta)$ where $[a,b]$ is the tolerance interval and $\alpha$ and $\beta$ the right and left widths respectively. The membership function $\mu_A(x)$ is given by:
        \[
        \mu_A(x) = 
        \begin{cases} 
        1-\frac{a-x}{\alpha} & \text{if } a \leq x < a-\alpha, \\
        1, & \text{if } b \leq x \leq a, \\
        1-\frac{x-b}{\beta} & \text{if } b+\beta < x \leq b, \\
        0, & \text{otherwise.}
        \end{cases}
        \]
        
        \item \textbf{LR-Fuzzy Number:} Defined by a quadruplet $A\equiv(a, b, \alpha, \beta)$ where $[a,b]$ is the core (or peak) interval and $\alpha$ and $\beta$ the left and right widths respectively. The membership function $\mu_A(x)$ is given by:
        \[
        \mu_A(x) = 
        \begin{cases} 
        L\left(\frac{a-x}{\alpha}\right) & \text{if } a-\alpha \leq x < a, \\
        1, & \text{if } a \leq x \leq b, \\
        R\left(\frac{x-b}{\beta}\right) & \text{if } b < x \leq b+\beta, \\
        0, & \text{otherwise,}
        \end{cases}
        \]
        where $L$ and $R$ are continuous monotone non-increasing functions from $[0,1]$ to $[0,1]$ with $L(0)=R(0)=1$.
    \end{itemize}
\end{example}
    
\begin{figure}[H]
    \centering
    \includegraphics[width=\textwidth]{ch1/figures/fuzzy_numbers.png}
    \caption{Plots of Triangular, Trapezoidal, and LR Fuzzy Numbers}
    \label{fig:fuzzy_numbers}
\end{figure}



\subsection{Nguyen's Theorems}
\signal{We use continuous functions because that way, we get the image of an interval is an interval as well. So then we get another fuzzy number because it maintains the convexity property?

That is because the image under $f:\R \longrightarrow \R$ continuous of a compact is compact and of a connected set is a connected set. Therefore continuous functions move closed intervals to closed intervals.}

%https://sci-hub.se/10.1016/0165-0114(91)90139-H
\begin{theorem}[First Nguyen Theorem]
    Let $f:\, \R \longrightarrow \R$ a continuous function and $A\in \R$ any fuzzy number \signal{(creo que vale para LR fuzzy num)}. Then,
    \[
    [f(A)]^{\alpha} = f([A]^{\alpha})=\{f(x)\mid x\in [A]^\alpha\}
    \]
    Moreover, if $f$ is monotonically increasing (if $f$ were decreasing, the order of the interval would be reversed), then:
    \[
    [f(A)]^{\alpha} = f([a_1(\alpha), a_2(\alpha)])=
    [f(a_1(\alpha)), f(a_2(\alpha))]
    \]
    where $[\cdot]^\alpha$ denotes the $\alpha$-cut of a fuzzy set and $a_1(\alpha), a_2(\alpha)$ the extremes of the $\alpha$-cut.
\end{theorem}

\signal{Sup- t-norm convolution para la generalización lo menciono?? Y eso de la convolución es útil para algo más?}

\begin{theorem}[Second Nguyen Theorem]
    Let $f:\, \R \times \R\longrightarrow \R$ a continuous function and $A,B$ \signal{any} fuzzy numbers. Then,
    \[
    [f(A,B)]^{\alpha} = f([A]^{\alpha},[B]^{\alpha})=\{f(x_1,x_2)\mid x_1\in [A]^\alpha, \, x_2\in [B]^\alpha\}
    \signal{=[A]^\alpha [B]^\alpha}
    \]
    where $[\cdot]^\alpha$ denotes the $\alpha$-cut of a fuzzy set.
\end{theorem}


\signal{Añado generalization of Nguyen Theorems by Fuller in section 1.9 of \cite{FULLER2}?}








\subsection{Nguyen's Theorems}
Nguyen's theorems provide fundamental results for computing the $\alpha$-cuts of fuzzy numbers that result from applying functions, based on Zadeh's extension principle. These theorems are crucial because they allow us to perform operations on fuzzy numbers by working with their $\alpha$-cuts (which are crisp intervals) directly. The following theorem \cite[Thm. 1.3.1, p. 17]{FULLER2}, \cite{NGUYEN1978}:

\begin{theorem}[First Nguyen Theorem]
    Let $f: \mathbb{R} \to \mathbb{R}$ be a continuous function and $A$ be a fuzzy number. Then, the $\alpha$-cut of the fuzzy number $f(A)$ (obtained via the extension principle) is given by:
    \[
    [f(A)]^{\alpha} = f([A]^{\alpha}) = \{f(x) \mid x \in [A]^\alpha\}
    \]

    Moreover, if $f$ is monotonically increasing, and $[A]^\alpha = [a_1(\alpha), a_2(\alpha)]$, then:
    \[
    [f(A)]^{\alpha} = [f(a_1(\alpha)), f(a_2(\alpha))]
    \]
    If $f$ were monotonically decreasing, the resulting interval would be $[f(a_2(\alpha)), f(a_1(\alpha))]$.
\end{theorem}
% \begin{proof}
%     (Intuition) The extension principle defines $(f(A))(y) = \sup_{x: f(x)=y} A(x)$.
%     For $y \in [f(A)]^\alpha$, we have $(f(A))(y) \ge \alpha$. This means there exists an $x_0$ such that $f(x_0)=y$ and $A(x_0) \ge \alpha$. Thus $x_0 \in [A]^\alpha$, and $y = f(x_0) \in f([A]^\alpha)$.
%     Conversely, if $y \in f([A]^\alpha)$, then $y=f(x_0)$ for some $x_0 \in [A]^\alpha$ (so $A(x_0) \ge \alpha$). Then $(f(A))(y) = \sup_{x: f(x)=y} A(x) \ge A(x_0) \ge \alpha$, so $y \in [f(A)]^\alpha$.
%     The continuity of $f$ is essential for ensuring that $f([A]^\alpha)$ is a closed interval when $[A]^\alpha$ is a closed interval, which is required for $f(A)$ to be a fuzzy number.
% \end{proof}

\begin{remark}
Since $A$ is a fuzzy number, its $\alpha$-cuts $[A]^\alpha$ are closed and bounded intervals (compact and connected sets in $\mathbb{R}$). A continuous function $f: \mathbb{R} \to \mathbb{R}$ maps compact sets to compact sets and connected sets (intervals) to connected sets (intervals). Therefore, $f([A]^\alpha)$ is also a closed and bounded interval. This property ensures that $f(A)$ (whose $\alpha$-cuts are these $f([A]^\alpha)$) is indeed a fuzzy number, as its $\alpha$-cuts remain convex (i.e., are intervals) and satisfy other necessary properties. \signal{Y qué pasa con la propiedad de convexity?}
\end{remark}
The following theorem \cite[Thm. 1.3.2, p. 18]{FULLER2}, \cite{NGUYEN1978}:
\begin{theorem}[Second Nguyen Theorem]
    Let $f: \mathbb{R} \times \mathbb{R} \to \mathbb{R}$ be a continuous function (in both arguments) and $A, B$ be fuzzy numbers. Then, the $\alpha$-cut of the fuzzy number $f(A,B)$ (obtained via the extension principle) is given by:
    \[
    [f(A,B)]^{\alpha} = f([A]^{\alpha}, [B]^{\alpha}) = \{f(x_1, x_2) \mid x_1 \in [A]^\alpha, x_2 \in [B]^\alpha\}
    \]
    \signal{Regarding the notation $=[A]^\alpha [B]^\alpha$: This specific notation is typically used when $f$ represents multiplication, i.e., $f(x_1, x_2) = x_1 \cdot x_2$. In that case, $f([A]^\alpha, [B]^\alpha)$ becomes the interval product $[A]^\alpha \cdot [B]^\alpha$. The theorem statement is more general: $f([A]^\alpha, [B]^\alpha)$ means applying the function $f$ to elements from the respective intervals. For example, if $f(x_1, x_2) = x_1 + x_2$, then $f([A]^\alpha, [B]^\alpha) = [A]^\alpha + [B]^\alpha$ (interval addition).}
\end{theorem}
% \begin{proof}
%     (Intuition) Similar to the first theorem, the extension principle for a two-place function is $(f(A,B))(z) = \sup_{x_1,x_2: f(x_1,x_2)=z} \min(A(x_1), B(x_2))$.
%     If $z \in [f(A,B)]^\alpha$, then $(f(A,B))(z) \ge \alpha$. This implies there exist $x_{1,0}, x_{2,0}$ such that $f(x_{1,0}, x_{2,0}) = z$ and $\min(A(x_{1,0}), B(x_{2,0})) \ge \alpha$. Thus, $A(x_{1,0}) \ge \alpha$ (so $x_{1,0} \in [A]^\alpha$) and $B(x_{2,0}) \ge \alpha$ (so $x_{2,0} \in [B]^\alpha$). Therefore, $z = f(x_{1,0}, x_{2,0}) \in f([A]^\alpha, [B]^\alpha)$.
%     Conversely, if $z \in f([A]^\alpha, [B]^\alpha)$, then $z = f(x_{1,0}, x_{2,0})$ for some $x_{1,0} \in [A]^\alpha$ and $x_{2,0} \in [B]^\alpha$. This means $A(x_{1,0}) \ge \alpha$ and $B(x_{2,0}) \ge \alpha$, so $\min(A(x_{1,0}), B(x_{2,0})) \ge \alpha$. Then $(f(A,B))(z) \ge \min(A(x_{1,0}), B(x_{2,0})) \ge \alpha$, so $z \in [f(A,B)]^\alpha$.
%     Continuity of $f$ ensures that the image of the compact set $[A]^\alpha \times [B]^\alpha$ is a compact interval, preserving the fuzzy number structure.
% \end{proof}

\subsubsection{Generalization of Nguyen's Theorem by Fullér and Keresztfalvi}
The original Nguyen's theorem for two-place functions implicitly relies on Zadeh's extension principle, which uses a sup-min convolution to aggregate membership degrees. Fullér and Keresztfalvi generalized this result to cases where the extension principle is defined via a sup-T-norm convolution, where $T$ is an arbitrary t-norm \cite[Sec. 1.9]{FULLER2}.

The generalized extension principle defines the membership of $z$ in $f(A,B)$ as:
\[
(f(A,B))(z) = \sup_{f(x,y)=z} T(A(x), B(y))
\]
\signal{Comment: Sup-t-norm convolution para la generalización lo menciono?? Y eso de la convolución es útil para algo más?
Response: Yes, the sup-T-norm convolution is central to this generalization. It allows for more flexibility in modeling how the membership degrees of $A$ and $B$ combine. Different t-norms represent different logical interpretations of "conjunction" or "aggregation." For example, $T(a,b) = \min(a,b)$ (Zadeh's original), $T(a,b) = ab$ (product t-norm), or $T(a,b) = \max(0, a+b-1)$ (Lukasiewicz t-norm) each lead to different fuzzy arithmetic operations.}

The generalized theorem states the condition for the equality:
\begin{equation} \label{eq:nguyen_generalized}
[f(A, B)]_\alpha = \bigcup_{T(\xi, \eta) \ge \alpha} f([A]^\xi, [B]^\eta), \quad \alpha \in (0, 1]
\end{equation}
(Note: $[A]^\xi$ and $A_\xi$ are often used interchangeably for $\alpha$-level sets.)

It's important to see that if $T(x,y) = \min(x,y)$, then $T(\xi, \eta) \ge \alpha$ implies $\xi \ge \alpha$ and $\eta \ge \alpha$. In this case, the union $\bigcup_{\min(\xi, \eta) \ge \alpha} f([A]^\xi, [B]^\eta)$ simplifies. Since $f([A]^\xi, [B]^\eta) \subseteq f([A]^\alpha, [B]^\alpha)$ for $\xi, \eta \ge \alpha$ (due to the nesting property of $\alpha$-cuts and assuming $f$ behaves well with set inclusion, e.g., for interval operations), the largest set in the union is $f([A]^\alpha, [B]^\alpha)$. Thus, Equation \eqref{eq:nguyen_generalized} reduces to:
\[
[f(A, B)]_\alpha = f([A]^\alpha, [B]^\alpha)
\]
which is precisely Nguyen's original second theorem.

Fullér and Keresztfalvi provide the following key results \cite[Thms. 1.9.1, 1.9.2]{FULLER2}:
\begin{theorem}
    Let $f: X \times Y \to Z$ be a two-place function, $A \in \mathcal{F}(X)$, $B \in \mathcal{F}(Y)$, and $T$ be a t-norm. A necessary and sufficient condition for the equality \eqref{eq:nguyen_generalized} to hold is that for each $z \in Z$, the supremum $\sup_{f(x,y)=z} T(A(x), B(y))$ is attained.
\end{theorem}

\begin{theorem}
    If $f: X \times Y \to Z$ is continuous, the t-norm $T$ is upper semicontinuous (u.s.c.), and $A, B$ are fuzzy subsets with u.s.c., compactly-supported membership functions (denoted $A \in \mathcal{F}(X, \mathcal{K}), B \in \mathcal{F}(Y, \mathcal{K})$), then the equality \eqref{eq:nguyen_generalized} holds.
    (Here $X, Y, Z$ are locally compact topological spaces).
\end{theorem}
% \begin{proof}
%     (Intuition for Theorem 2) The conditions (continuous $f$, u.s.c. $T$, and $A, B$ being u.s.c. with compact support) ensure that the function $\varphi(x,y) = T(A(x), B(y))$ is u.s.c. and the domain over which the supremum is taken, $f^{-1}(z) \cap (\text{supp} A \times \text{supp} B)$, is compact. An u.s.c. function on a compact set attains its maximum. This guarantees that the condition of the first theorem (supremum is attained) is met.
% \end{proof}
This generalization is significant as it extends the computational convenience of Nguyen's theorem to a broader class of fuzzy operations defined by various t-norms, which are common in fuzzy logic and systems.
The examples provided in \cite[p. 35]{FULLER2} illustrate how the $\alpha$-cuts of $f(A,B)$ can be generated for different t-norms like the weak t-norm $T_W$, product t-norm $T_P(x,y)=xy$, and Lukasiewicz t-norm $T_L(x,y) = \max(0, x+y-1)$.
For $T_P(x,y) = xy$, the formula becomes:
\[ [f(A, B)]_\alpha = \bigcup_{\xi \cdot \eta \ge \alpha} f([A]^\xi, [B]^\eta) = \bigcup_{\xi \in [\alpha,1]} f([A]^\xi, [B]^{\alpha/\xi}) \]
And for $T_L(x,y) = \max(0, x+y-1)$, it becomes:
\[ [f(A, B)]_\alpha = \bigcup_{\max(0,\xi+\eta-1) \ge \alpha} f([A]^\xi, [B]^\eta) = \bigcup_{\xi \in [\alpha,1]} f([A]^\xi, [B]^{\alpha+1-\xi}) \]
(assuming $\alpha+1-\xi \le 1$). These specific forms are useful for practical computations with different t-norm based arithmetic.












\subsection{Fuzzy Arithmetic}
\label{sec:fuzzy_arithmetic}

Building upon the extension principle, arithmetic operations for fuzzy numbers can be defined. These are very useful for defining and solving fuzzy linear programming problems where crisp constraint may be relaxed with fuzzy ones or for aggregating different criteria represented by fuzzy numbers. Their most common implementation is using the minimum t-norm and triangular numbers.

\paragraph{Sup-Min Based Arithmetic for LR-Fuzzy Numbers}
Let $\tilde{A} = (a_1, a_2, \alpha_L, \alpha_R)_{LR}$ and $\tilde{B} = (b_1, b_2, \beta_L, \beta_R)_{LR}$ be two fuzzy numbers of LR-type. Using the sup-min extension principle, the following operational rules can be derived \cite[p.16]{FULLER2}.

\begin{proposition}[Addition and Subtraction of LR-Fuzzy Numbers]
\label{prop:lr_add_sub}
The sum and difference of $\tilde{A}$ and $\tilde{B}$ are given by:
\begin{align}
\tilde{A} \oplus \tilde{B} &= (a_1+b_1, a_2+b_2, \alpha_L+\beta_L, \alpha_R+\beta_R)_{LR} \\
\tilde{A} \ominus \tilde{B} &= (a_1-b_2, a_2-b_1, \alpha_L+\beta_R, \alpha_R+\beta_L)_{LR}
\end{align}
\end{proposition}

\begin{proposition}[Scalar Multiplication of LR-Fuzzy Numbers]
\label{prop:lr_scalar_mult}
For a real number $\lambda \in \mathbb{R}$, the scalar multiplication $\lambda \odot \tilde{A}$ is:
\begin{equation}
\lambda \odot \tilde{A} =
\begin{cases}
(\lambda a_1, \lambda a_2, \lambda\alpha_L, \lambda\alpha_R)_{LR} & \text{if } \lambda \ge 0 \\
(\lambda a_2, \lambda a_1, |\lambda|\alpha_R, |\lambda|\alpha_L)_{LR} & \text{if } \lambda < 0
\end{cases}
\end{equation}
If $\tilde{A}$ is a quasi-triangular fuzzy number, i.e., $a_1=a_2=a$, denoted $\tilde{A}=(a, \alpha_L, \alpha_R)_{LR}$, then for $\lambda < 0$:
\begin{equation}
\lambda \odot \tilde{A} = (\lambda a, \lambda a, |\lambda|\alpha_R, |\lambda|\alpha_L)_{LR} = (\lambda a, |\lambda|\alpha_R, |\lambda|\alpha_L)_{LR}
\end{equation}
\end{proposition}

\begin{remark}
For triangular fuzzy numbers, which are a special case of LR-fuzzy numbers where $L(x)=R(x)=\max(0, 1-x)$:
\begin{itemize}
    \item If $\tilde{A} = (a, \alpha_L, \alpha_R)$ (peak $a$, left spread $\alpha_L$, right spread $\alpha_R$) and $\tilde{B} = (b, \beta_L, \beta_R)$, then:
    \begin{align*}
    \tilde{A} \oplus \tilde{B} &= (a+b, \alpha_L+\beta_L, \alpha_R+\beta_R) \\
    \tilde{A} \ominus \tilde{B} &= (a-b, \alpha_L+\beta_R, \alpha_R+\beta_L)
    \end{align*}
    \item If $\tilde{A} = (a, \alpha)$ (symmetric triangular, peak $a$, spread $\alpha$) and $\tilde{B} = (b, \beta)$, then:
    \begin{align*}
    \tilde{A} \oplus \tilde{B} &= (a+b, \alpha+\beta) \\
    \tilde{A} \ominus \tilde{B} &= (a-b, \alpha+\beta) \\
    \lambda \odot \tilde{A} &= (\lambda a, |\lambda|\alpha)
    \end{align*}
\end{itemize}
These simplified forms are widely used in practical applications \cite[p.17]{FULLER2}.
\end{remark}



The product of two fuzzy numbers $\tilde{A}$ and $\tilde{B}$ using the sup-min extension principle is defined as:
\begin{equation}
\mu_{\tilde{A} \otimes \tilde{B}}(z) = \sup_{x \cdot y = z} \min(\mu_{\tilde{A}}(x), \mu_{\tilde{B}}(y))
\end{equation}
Unlike addition and subtraction, the product of two LR-fuzzy numbers is not, in general, an LR-fuzzy number itself, and simple formulas for its parameters are usually not available.
However, computations can often be simplified by working with $\alpha$-cuts and usign the Nguyen's theorem \cite[Thm. 1.3.2]{FULLER2}:

\begin{theorem}
\label{thm:alpha_cut_product}
Let $f: \mathbb{R} \times \mathbb{R} \to \mathbb{R}$ be a continuous function, and let $\tilde{A}$ and $\tilde{B}$ be fuzzy numbers. Then the $\alpha$-cut of $f(\tilde{A}, \tilde{B})$ is given by:
\begin{equation}
[f(\tilde{A}, \tilde{B})]^\alpha = f([\tilde{A}]^\alpha, [\tilde{B}]^\alpha) = \{f(x,y) \mid x \in [\tilde{A}]^\alpha, y \in [\tilde{B}]^\alpha \}
\end{equation}
\end{theorem}

For the product operation, $f(x,y) = xy$. Thus, $[\tilde{A} \otimes \tilde{B}]^\alpha = [\tilde{A}]^\alpha \cdot [\tilde{B}]^\alpha$.
If $\tilde{A}$ and $\tilde{B}$ are non-negative fuzzy numbers (i.e., their supports are in $\mathbb{R}^+_0$), and their $\alpha$-cuts are $[\tilde{A}]^\alpha = [a_1(\alpha), a_2(\alpha)]$ and $[\tilde{B}]^\alpha = [b_1(\alpha), b_2(\alpha)]$, then the $\alpha$-cut of their product is:
\begin{equation}
[\tilde{A} \otimes \tilde{B}]^\alpha = [a_1(\alpha)b_1(\alpha), a_2(\alpha)b_2(\alpha)]
\end{equation}
This holds if and only if $\tilde{A}$ and $\tilde{B}$ are both non-negative \cite[p.18]{FULLER2}.
If one or both fuzzy numbers can take negative values, the interval multiplication for the $\alpha$-cuts becomes more complex:
\begin{equation}
[\tilde{A}]^\alpha \cdot [\tilde{B}]^\alpha = [\min(P), \max(P)]
\end{equation}
where $P = \{a_1(\alpha)b_1(\alpha), a_1(\alpha)b_2(\alpha), a_2(\alpha)b_1(\alpha), a_2(\alpha)b_2(\alpha)\}$.



\paragraph{T-norm Based Addition of Fuzzy Numbers}
When employing the sup-T extension principle with an Archimedean t-norm $T$ (having an additive generator $f$), the sum of fuzzy numbers takes a specific form.
Let $\tilde{A}_i = (a_i, b_i, \alpha, \beta)_{LR}$, $i=1, \dots, n$, be $n$ fuzzy numbers of LR-type with common spreads $\alpha, \beta$.
Let $f$ be the additive generator of $T$, and $L,R$ be the shape functions. The following theorem \cite[Thm. 1.7.1]{FULLER2}:

\begin{theorem}
\label{thm:tnorm_addition_lr}
Let $T$ be an Archimedean t-norm with additive generator $f$. Let $\tilde{a}_i = (a_i, b_i, \alpha, \beta)_{LR}$, $i=1, \dots, n$, be fuzzy numbers of LR-type. If $L$ and $R$ are twice differentiable, concave functions, and $f$ is a twice differentiable, strictly convex function, then the membership function of the T-sum $\tilde{A}_n = \tilde{a}_1 \oplus_T \dots \oplus_T \tilde{a}_n$ is given by:
\begin{equation}
\mu_{\tilde{A}_n}(z) =
\begin{cases}
1 & \text{if } A_n \le z \le B_n \\
f^{[-1]}\left( n \cdot f\left(L\left(\frac{A_n-z}{n\alpha}\right)\right) \right) & \text{if } A_n - n\alpha \le z < A_n \\
f^{[-1]}\left( n \cdot f\left(R\left(\frac{z-B_n}{n\beta}\right)\right) \right) & \text{if } B_n < z \le B_n + n\beta \\
0 & \text{otherwise}
\end{cases}
\end{equation}
where $A_n = \sum_{i=1}^n a_i$, $B_n = \sum_{i=1}^n b_i$, and $f^{[-1]}$ is the pseudo-inverse of $f$.
\end{theorem}
\begin{remark}
This theorem provides a computationally efficient way to calculate the t-norm based sum of LR-fuzzy numbers under specific conditions. It has been generalized for varying spreads and different conditions on $f, L, R$ by several authors \cite[p.29]{FULLER2}. For example, it remains valid if $f$ is convex and $L,R$ are concave.
\end{remark}
The product of two fuzzy numbers $\tilde{A}$ and $\tilde{B}$ using the sup-T extension principle, where $T$ is a t-norm, is defined as \cite[p.20]{FULLER2}:
\begin{equation}
\mu_{\tilde{A} \otimes_T \tilde{B}}(z) = \sup_{x \cdot y = z} T(\mu_{\tilde{A}}(x), \mu_{\tilde{B}}(y))
\end{equation}
Similar to the sup-min product, closed-form expressions for the resulting membership function are not always straightforward for general LR-fuzzy numbers. However, the functional relationship described next provides significant insight for specific cases involving identical positive LR-fuzzy numbers.


\paragraph{Functional Relationship Between T-norm Based Addition and Multiplication}
A remarkable functional relationship exists between t-norm based addition and t-norm based multiplication for positive LR-fuzzy numbers of the same type. The following theorem \cite[Thm. 1.8.1]{FULLER2}:

\begin{theorem}
\label{thm:functional_relationship_add_mult}
Let $T$ be an Archimedean t-norm with an additive generator $f$. Let $\tilde{M}_i = \tilde{M} = (a,b,\alpha,\beta)_{LR}$ for $i=1,\dots,n$ be $n$ identical positive fuzzy numbers of LR-type (i.e., their support is in $\mathbb{R}^+$). If $L$ and $R$ are twice differentiable, concave functions, and $f$ is a twice differentiable, strictly convex function, then for $z \ge 0$:
\begin{equation}
\mu_{(\tilde{M}_1 \oplus_T \dots \oplus_T \tilde{M}_n)}(nz) = \mu_{(\tilde{M}_1 \otimes_T \dots \otimes_T \tilde{M}_n)}(z^n) = f^{[-1]}(n \cdot f(\mu_{\tilde{M}}(z)))
\end{equation}
where $\oplus_T$ denotes t-norm based addition and $\otimes_T$ denotes t-norm based multiplication.
\end{theorem}

\begin{remark}
This theorem establishes that the membership function of the sum of $n$ identical fuzzy numbers, when evaluated at $nz$, is equal to the membership function of the product of these $n$ fuzzy numbers evaluated at $z^n$. Both are determined by scaling the transformed membership function of the original fuzzy number $\tilde{M}$ using its generator $f$. This relationship is particularly useful in analyzing the behavior of aggregated fuzzy quantities. An immediate consequence is that if $e^*_n(\tilde{M}) = (\tilde{M} \oplus_T \dots \oplus_T \tilde{M})/n$ (where division by $n$ is scalar multiplication by $1/n$), then $\lim_{n\to\infty} e^*_n(\tilde{M})$ converges to the core $[a,b]$ of $\tilde{M}$ \cite{FULLER2}{p.33}.
\end{remark}

\signal{
\subsection{Metrics for fuzzy numbers}

\subsection{Defuzzification and Fuzzy Linear Programming Problems}}
%Most methods used for solving such problems are based on ranking functions, alfa-cuts, using duality results or penalty functions. In these methods, authors deal with crisp formulations of the fuzzy problems. Recently, some heuristic algorithms have also been proposed. In these methods, some authors solve the fuzzy problem directly, while others solve the crisp problems approximately.
