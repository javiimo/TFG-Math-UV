\signal{Aquí mencionar por encima alternativas sin entrar en mucho detalle: type-1 (pertenencia puntual), type-2 (pertenencia intervalar), complex memberships, intuitionistic, hesitant, multiplicative versions of hesitant and intuitionistic, bipolar and multipolar, fuzzy-rough sets, Neutrosophic and Plithogenic sets, picture fuzzy sets...}

The fuzzy sets that have been studied in this chapter are their original definition from Zadeh in \cite{Zadeh1965}, also known as type-1. However, different problems have motivated the definition of variations or extensions of these fuzzy sets leading to a long list of fuzzy sets, most of them briefly presented in this section:

\begin{itemize}
    \item \textbf{Type-2 Fuzzy Sets} \cite{Zadeh1975}: Membership grades are themselves fuzzy sets, modelling uncertainty \emph{about} the membership function itself. \textbf{Interval Type-2 Fuzzy Sets} is a common simplification where the secondary membership is an interval.
 
    \item \textbf{Type-n Fuzzy Sets} \cite{Turksen1986_TypeNRelated}: Generalize Type-2 sets by defining membership as a fuzzy set of Type-($n-1$), primarily of theoretical interest.

    \item \textbf{Intuitionistic Fuzzy Sets (IFS)} \cite{Atanassov1986}: Characterized by independent membership $\mu(x)$ and non-membership $\nu(x)$ degrees, constrained by $\mu(x) + \nu(x) \leq 1$. The remainder $1 - (\mu(x) + \nu(x))$ represents hesitation.

    \item \textbf{Interval-Valued Intuitionistic Fuzzy Sets (IVIFS)} \cite{AtanassovGargov1989}: Extend IFS by representing both membership and non-membership degrees as intervals within $[0,1]$.

    \item \textbf{Pythagorean Fuzzy Sets (PFS)} \cite{Yager2013_Pythagorean}: Similar to IFS, but use a relaxed constraint $\mu(x)^2 + \nu(x)^2 \leq 1$. 

    \item \textbf{q-Rung Orthopair Fuzzy Sets (q-ROFS)} \cite{Yager2017_qRung}: Generalize IFS and PFS with the constraint $\mu(x)^q + \nu(x)^q \leq 1$ for $q \ge 1$.

    \item \textbf{Spherical Fuzzy Sets (SFS)} \cite{GundogduKahraman2019_Spherical}: Extend orthopairs to three components: membership $\mu(x)$, non-membership $\nu(x)$, and indeterminacy $\pi(x)$, constrained by $\mu(x)^2 + \nu(x)^2 + \pi(x)^2 \leq 1$. They handle these aspects on a unit sphere.

    \item \textbf{Hesitant Fuzzy Sets (HFS)} \cite{Torra2010}: Assign a set of several possible membership values in $[0,1]$ to an element. This directly models hesitation among different membership degrees.

    \item \textbf{Bipolar Fuzzy Sets} \cite{Zhang1994_Bipolar}: Represent independent positive membership (satisfaction of a property) and negative membership (satisfaction of a counter-property). This models situations with bipolar or dichotomous information. \textbf{Bipolar Complex / Bipolar Spherical Fuzzy Sets} extend the bipolar concept using complex numbers or spherical orthopairs. This allows for richer representations in polarity-aware scenarios.

    \item \textbf{Neutrosophic Sets (NS)} \cite{Smarandache1998_Neutrosophic, Wang2010_SVNS}: Defined by three independent components: Truth (T), Indeterminacy (I), and Falsity (F) memberships. Single-Valued Neutrosophic Sets (SVNS) constrain T, I, F to $[0,1]$.

    \item \textbf{Picture Fuzzy Sets (PiFS)} \cite{Cuong2013_Picture}: Characterized by positive, neutral, and negative membership degrees, with their sum being $\leq 1$. These are suited for scenarios like voting with explicit "yes, abstain, no" options.

    \item \textbf{Rough Fuzzy Sets \& Fuzzy Rough Sets} \cite{DuboisPrade1990_FuzzyRough}: Hybrid models combining fuzzy sets and rough sets. They blend graded membership with boundary uncertainty from indiscernibility.

    \item \textbf{Soft Sets} \cite{Molodtsov1999_Soft}: A parameterized family of crisp subsets, mapping parameters to subsets of a universe. An alternative tool for uncertainty dealing with parameterization.

    \item \textbf{Fuzzy Soft Sets} \cite{MajiBiswasRoy2001_FuzzySoft}: Extend soft sets by allowing the parameterized subsets to be fuzzy sets rather than crisp ones. This combines parameterization with graded membership.

    \item \textbf{Fuzzy Multisets}: Assign a multiset of membership counts or grades instead of a single grade. Useful when the frequency or repetition of membership observations is significant.

    \item \textbf{L-Fuzzy Sets} \cite{Goguen1967}: Generalize Type-1 sets by taking membership grades from an arbitrary lattice L, not restricted to the interval $[0,1]$. This provides an order-theoretic abstraction.

    \item \textbf{Complex Fuzzy Sets}: Membership grades are complex numbers, typically $r \cdot e^{i\theta}$. This encodes both magnitude (degree of membership) and phase (additional information).

    \item \textbf{Quantum Fuzzy Sets} \cite{Pykacz2015_Quantum}: Interpret a quantum register's state-vector as the characteristic function of a fuzzy set. This applies principles from quantum mechanics to model membership.

    \item \textbf{Plithogenic Sets} \cite{Smarandache2018_Plithogenic}: Generalize crisp, fuzzy, IFS, and NS by characterizing elements with multiple attributes. Each attribute value has an appurtenance degree and a contradiction degree.

    \item \textbf{Hypersoft Sets} \cite{Smarandache2018_Hypersoft}: Extend soft sets to handle tuples or hierarchies of parameters (multi-argument approximate functions). This addresses multi-attribute decision-making.
    \begin{itemize}
        \item \textbf{Fuzzy Hypersoft Sets}: Combine fuzzy membership with the multi-parameter structures of hypersoft sets.
        \item \textbf{Plithogenic Soft Sets \& Plithogenic Hypersoft Sets}: Merge Plithogenic fuzziness into soft set or hypersoft set parameterizations.
    \end{itemize}

    \item \textbf{Multipolar Fuzzy Sets (m-Polar Fuzzy Sets)} \cite{Chen2014_mPolar}: Generalize bipolar fuzzy sets to $m$ poles, assigning a vector of $m$ membership degrees from $[0,1]^m$. Models multi-polar evidence or criteria.

    \item \textbf{k-Polar Intuitionistic Fuzzy Sets (IKFS)}: Assign $k$ distinct pairs of intuitionistic membership/non-membership degrees to each element. Captures multi-source or multi-criteria hesitation.

    \item \textbf{Multi-Neutrosophic Sets}: Evaluate truth, indeterminacy, and falsity through multiple sources, yielding n-tuples of (T,I,F) components.
    \begin{itemize}
        \item \textbf{Tripolar Neutrosophic Sets}: A specific type organizing truth, indeterminacy, and falsity along three distinct polar axes.
    \end{itemize}

    \item \textbf{Multi-Grade Fuzzy Sets}: Assign several membership grades (e.g., linguistic terms like ``low,'' ``medium,'' ``high'') from a structured, ordered scale. Captures more nuanced membership.
\end{itemize}















\begin{landscape}


\begin{figure}[h!]
    \centering
    \begin{tikzpicture}[
        event/.style={
            draw=nodecolor, 
            thick,
            fill=nodecolor!20,
            rounded corners=3pt,
            minimum height=1.5em,
            text width=3.5cm, % Adjust as needed
            align=center,
            font=\scriptsize
        },
        connector/.style={
            draw=connectorcolor,
            thick,
            -{Stealth[length=2mm, width=1.5mm]}
        }
    ]
        % Define timeline start and end years
        \def\ystart{1965}
        \def\yend{2020} % Extended a bit for clarity at the end
        \def\timelinewidth{32} % Width of the timeline in cm
        \def\unit{\timelinewidth/(\yend-\ystart)} % cm per year

        % Draw the main timeline axis
        \draw[timelinecolor, -{Stealth[length=3mm, width=2.5mm]}, very thick] (0,0) -- (\timelinewidth+0.5,0) node[right, black, font=\small] {Year};

        % Draw year ticks and labels
        \foreach \year in {1965, 1970, 1975, 1980, 1985, 1990, 1995, 2000, 2005, 2010, 2015, 2020} {
            \pgfmathsetmacro{\xpos}{(\year-\ystart)*\unit}
            \draw[timelinecolor, thick] (\xpos, -0.2) -- (\xpos, 0.2) node[below=3pt, black, font=\tiny] {\year};
        }

        % --- Fuzzy Set Type Nodes ---
        % Arguments: year, y-offset, anchor (for text), text content, bibkey, node name (unique)
        \newcommand{\timelineevent}[6]{
            \pgfmathsetmacro{\eventx}{(\unit*#1)-(\unit*\ystart)}
            \node[event, anchor=#3] (#5_node) at (\eventx, #2) {#4 \cite{#5}};
            \draw[connector] (#5_node.#3) -- (\eventx, 0);
        }
        
        \newcommand{\timelineeventshift}[7]{ % With x-shift for close events
            \pgfmathsetmacro{\eventx}{(\unit*#1)-(\unit*\ystart) + #6}
            \node[event, anchor=#3] (#5_node) at (\eventx, #2) {#4 \cite{#5}};
            \draw[connector] (#5_node.#3) -- (\eventx-#6, 0); % Connector to actual year
        }


        % 1960s
        \timelineevent{1965}{1.5}{south}{Classical Fuzzy Sets}{Zadeh1965}{clfs}
        \timelineevent{1967}{-1.5}{north}{L-Fuzzy Sets}{Goguen1967}{lfs}

        % 1970s
        \timelineevent{1975}{1.5}{south}{Type-2 Fuzzy Sets}{Zadeh1975}{t2fs}

        % 1980s
        \timelineevent{1982}{-1.5}{north}{Rough Sets}{Pawlak1982_Rough}{rs}
        \timelineeventshift{1986}{2.0}{south}{Intuitionistic FS}{Atanassov1986}{-0.1}{ifs} % Shifted slightly
        \timelineeventshift{1986}{-2.0}{north}{Higher-Type FS (related)}{Turksen1986_TypeNRelated}{0.1}{htfs} % Shifted slightly
        \timelineevent{1989}{1.5}{south}{Interval-Valued IFS}{AtanassovGargov1989}{ivifs}
        
        % 1990s
        \timelineevent{1990}{-1.5}{north}{Fuzzy-Rough Sets}{DuboisPrade1990_FuzzyRough}{frs}
        \timelineevent{1994}{1.5}{south}{Bipolar Fuzzy Sets}{Zhang1994_Bipolar}{bfs}
        \timelineevent{1998}{-1.5}{north}{Neutrosophic Sets}{Smarandache1998_Neutrosophic}{ns}
        \timelineevent{1999}{1.5}{south}{Soft Sets}{Molodtsov1999_Soft}{ss}

        % 2000s
        \timelineevent{2001}{-1.5}{north}{Fuzzy Soft Sets}{MajiBiswasRoy2001_FuzzySoft}{fss}
        
        % 2010s
        \timelineeventshift{2010}{2.0}{south}{Hesitant Fuzzy Sets}{Torra2010}{-0.15}{hfs}
        \timelineeventshift{2010}{-2.0}{north}{Single-Valued NS}{Wang2010_SVNS}{0.15}{svns}
        \timelineeventshift{2013}{1.5}{south}{Pythagorean FS}{Yager2013_Pythagorean}{-0.15}{pyfs}
        \timelineeventshift{2013}{-1.5}{north}{Picture Fuzzy Sets}{Cuong2013_Picture}{0.15}{pfs}
        \timelineevent{2014}{1.5}{south}{m-Polar Fuzzy Sets}{Chen2014_mPolar}{mpfs}
        \timelineevent{2015}{-1.5}{north}{Quantum Fuzzy Sets}{Pykacz2015_Quantum}{qfs}
        \timelineevent{2017}{1.5}{south}{q-Rung Orthopair FS}{Yager2017_qRung}{qrofs}
        \timelineeventshift{2018}{-2.0}{north}{Hypersoft Sets}{Smarandache2018_Hypersoft}{-0.15}{hs}
        \timelineeventshift{2018}{2.0}{south}{Plithogenic Sets}{Smarandache2018_Plithogenic}{0.15}{pls}
        \timelineevent{2019}{-1.5}{north}{Spherical Fuzzy Sets}{GundogduKahraman2019_Spherical}{sphs}

    \end{tikzpicture}
    \caption{Timeline of Fuzzy Set Type Introductions. References: Classical Fuzzy Sets \cite{Zadeh1965}, L-Fuzzy Sets \cite{Goguen1967}, Type-2 Fuzzy Sets \cite{Zadeh1975}, Rough Sets \cite{Pawlak1982_Rough}, Intuitionistic Fuzzy Sets \cite{Atanassov1986}, Higher-Type FS (related) \cite{Turksen1986_TypeNRelated}, Interval-Valued IFS \cite{AtanassovGargov1989}, Fuzzy-Rough Sets \cite{DuboisPrade1990_FuzzyRough}, Bipolar Fuzzy Sets \cite{Zhang1994_Bipolar}, Neutrosophic Sets \cite{Smarandache1998_Neutrosophic}, Soft Sets \cite{Molodtsov1999_Soft}, Fuzzy Soft Sets \cite{MajiBiswasRoy2001_FuzzySoft}, Hesitant Fuzzy Sets \cite{Torra2010}, Single-Valued Neutrosophic Sets \cite{Wang2010_SVNS}, Pythagorean Fuzzy Sets \cite{Yager2013_Pythagorean}, Picture Fuzzy Sets \cite{Cuong2013_Picture}, m-Polar Fuzzy Sets \cite{Chen2014_mPolar}, Quantum Fuzzy Sets \cite{Pykacz2015_Quantum}, q-Rung Orthopair Fuzzy Sets \cite{Yager2017_qRung}, Hypersoft Sets \cite{Smarandache2018_Hypersoft}, Plithogenic Sets \cite{Smarandache2018_Plithogenic}, Spherical Fuzzy Sets \cite{GundogduKahraman2019_Spherical}.}
    \label{fig:fuzzy_timeline}
\end{figure}

\end{landscape}