% \signal{Aquí mencionar por encima alternativas sin entrar en mucho detalle: type-1 (pertenencia puntual), type-2 (pertenencia intervalar), complex memberships, intuitionistic, hesitant, multiplicative versions of hesitant and intuitionistic, bipolar and multipolar, fuzzy-rough sets, Neutrosophic and Plithogenic sets, picture fuzzy sets...}

The fuzzy sets that have been studied in this chapter are the original definition from Zadeh in \cite{Zadeh1965}, also known as type-1. However, different problems have motivated the definition of variations of these fuzzy sets leading to a long list of fuzzy sets. The main approaches include the following:

% \paragraph{Increasing membership dimensionality:} Instead of considering only a membership function with one component $\mu(x)$, it can be expanded multiple components. Each of them may have a different meaning and extra constraints might be imposed that relate different components. A popular example is intuitionistic,... But also (all the other that are less popular)...


% \paragraph{Modifying the structure of the membership value:} Changing the $[0,1]$ domain for something else. L-fuzzy sets from Goguen or hesitant fuzzy sets or interval valued

% \paragraph{Higher orders of uncertainty:} Rather than having a precise membership value, we might want to address the uncertainty regarding that value as well. For example type-2,... or using intervals... also type-n can be defined...

% \paragraph{Integrating with other uncertainty frameworks: } Integration with rough sets or soft sets

\paragraph{Increasing membership dimensionality:} Instead of considering only a membership function with one component $\mu(x)$, it can be expanded multiple components. Each of them may have a different meaning and extra constraints might be imposed that relate different components. A popular example is intuitionistic fuzzy sets (IFS) \cite{Atanassov1986}, which define membership through two components: a degree of membership $\mu(x)$ and a degree of non-membership $\nu(x)$, satisfying $\mu(x) + \nu(x) \leq 1$, with the remainder representing hesitation. Pythagorean fuzzy sets (PFS) \cite{Yager2013_Pythagorean} offer a similar two-component structure but with a relaxed constraint $\mu(x)^2 + \nu(x)^2 \leq 1$, allowing for a broader representation of uncertainty. But also q-Rung Orthopair Fuzzy Sets (q-ROFS) \cite{Yager2017_qRung} that further generalize this constraint, Spherical Fuzzy Sets (SFS) \cite{GundogduKahraman2019_Spherical} which add an indeterminacy component under a spherical constraint, Picture Fuzzy Sets (PiFS) \cite{Cuong2013_Picture} with positive, neutral, and negative memberships, Neutrosophic Sets (NS) \cite{Smarandache1998_Neutrosophic, Wang2010_SVNS} characterized by truth, indeterminacy, and falsity, Bipolar Fuzzy Sets \cite{Zhang1994_Bipolar} handling positive and negative aspects (with extensions like Bipolar Complex or Bipolar Spherical Fuzzy Sets), Plithogenic Sets \cite{Smarandache2018_Plithogenic} dealing with multiple attributes and their appurtenance/contradiction degrees,  and Multipolar Fuzzy Sets (m-Polar Fuzzy Sets) \cite{Chen2014_mPolar} extend these multi-dimensional ideas.

\paragraph{Modifying the structure of the membership value:} Changing the $[0,1]$ domain for something else. L-fuzzy sets from Goguen \cite{Goguen1967}, which generalize type-1 fuzzy sets by allowing membership grades to be drawn from an arbitrary lattice L instead of just $[0,1]$, provide an order-theoretic abstraction. Another example is hesitant fuzzy sets (HFS) \cite{Torra2010}, where the membership of an element is a set of several possible values in $[0,1]$, directly modeling hesitation among different membership degrees. The concept of interval-valued fuzzy sets, where the membership degree itself is an interval within $[0,1]$, such as, in the definition of components for Interval-Valued Intuitionistic Fuzzy Sets \cite{AtanassovGargov1989}.

\paragraph{Higher orders of uncertainty:} Rather than having a precise membership value, we might want to address the uncertainty regarding that value as well. For example type-2 fuzzy sets \cite{Zadeh1975}, where the membership grade of an element is itself a fuzzy set (often a fuzzy number) over $[0,1]$, capture this higher-order uncertainty. A widely used simplification is Interval Type-2 Fuzzy Sets \cite{Zadeh1975}, where this secondary membership footprint is an interval, directly representing the uncertainty about the primary membership with a range of possible values. Also type-n fuzzy sets \cite{Turksen1986_TypeNRelated} can be defined, which generalize this by defining membership as a fuzzy set of Type-($n-1$), though these are primarily of theoretical interest.

\paragraph{Integration with other uncertainty frameworks:} The combination with rough sets, leading to Rough Fuzzy Sets and Fuzzy Rough Sets \cite{DuboisPrade1990_FuzzyRough}, which are hybrid models combining graded membership with the boundary region uncertainty from indiscernibility inherent in rough set theory. Another approach is using soft sets \cite{Molodtsov1999_Soft}, giving rise to Fuzzy Soft Sets \cite{MajiBiswasRoy2001_FuzzySoft} where parameterized collections of objects are fuzzy sets rather than crisp ones. 


% \begin{itemize}
%     \item \textbf{Type-2 Fuzzy Sets} \cite{Zadeh1975}: Membership grades are themselves fuzzy sets, modelling uncertainty \emph{about} the membership function itself. \textbf{Interval Type-2 Fuzzy Sets} is a common simplification where the secondary membership is an interval.
 
%     \item \textbf{Type-n Fuzzy Sets} \cite{Turksen1986_TypeNRelated}: Generalize Type-2 sets by defining membership as a fuzzy set of Type-($n-1$), primarily of theoretical interest.

%     \item \textbf{Intuitionistic Fuzzy Sets (IFS)} \cite{Atanassov1986}: Characterized by independent membership $\mu(x)$ and non-membership $\nu(x)$ degrees, constrained by $\mu(x) + \nu(x) \leq 1$. The remainder $1 - (\mu(x) + \nu(x))$ represents hesitation.

%     \item \textbf{Interval-Valued Intuitionistic Fuzzy Sets (IVIFS)} \cite{AtanassovGargov1989}: Extend IFS by representing both membership and non-membership degrees as intervals within $[0,1]$.

%     \item \textbf{Pythagorean Fuzzy Sets (PFS)} \cite{Yager2013_Pythagorean}: Similar to IFS, but use a relaxed constraint $\mu(x)^2 + \nu(x)^2 \leq 1$. 

%     \item \textbf{q-Rung Orthopair Fuzzy Sets (q-ROFS)} \cite{Yager2017_qRung}: Generalize IFS and PFS with the constraint $\mu(x)^q + \nu(x)^q \leq 1$ for $q \ge 1$.

%     \item \textbf{Spherical Fuzzy Sets (SFS)} \cite{GundogduKahraman2019_Spherical}: Extend orthopairs to three components: membership $\mu(x)$, non-membership $\nu(x)$, and indeterminacy $\pi(x)$, constrained by $\mu(x)^2 + \nu(x)^2 + \pi(x)^2 \leq 1$. They handle these aspects on a unit sphere.

%     \item \textbf{Hesitant Fuzzy Sets (HFS)} \cite{Torra2010}: Assign a set of several possible membership values in $[0,1]$ to an element. This directly models hesitation among different membership degrees.

%     \item \textbf{Bipolar Fuzzy Sets} \cite{Zhang1994_Bipolar}: Represent independent positive membership (satisfaction of a property) and negative membership (satisfaction of a counter-property). This models situations with bipolar or dichotomous information. \textbf{Bipolar Complex / Bipolar Spherical Fuzzy Sets} extend the bipolar concept using complex numbers or spherical orthopairs. This allows for richer representations in polarity-aware scenarios.

%     \item \textbf{Neutrosophic Sets (NS)} \cite{Smarandache1998_Neutrosophic, Wang2010_SVNS}: Defined by three independent components: Truth (T), Indeterminacy (I), and Falsity (F) memberships. Single-Valued Neutrosophic Sets (SVNS) constrain T, I, F to $[0,1]$.

%     \item \textbf{Picture Fuzzy Sets (PiFS)} \cite{Cuong2013_Picture}: Characterized by positive, neutral, and negative membership degrees, with their sum being $\leq 1$. These are suited for scenarios like voting with explicit ``yes, abstain, no" options.

%     \item \textbf{Rough Fuzzy Sets \& Fuzzy Rough Sets} \cite{DuboisPrade1990_FuzzyRough}: Hybrid models combining fuzzy sets and rough sets. They blend graded membership with boundary uncertainty from indiscernibility.

%     \item \textbf{Soft Sets} \cite{Molodtsov1999_Soft}: A parameterized family of crisp subsets, mapping parameters to subsets of a universe. An alternative tool for uncertainty dealing with parameterization.

%     \item \textbf{Fuzzy Soft Sets} \cite{MajiBiswasRoy2001_FuzzySoft}: Extend soft sets by allowing the parameterized subsets to be fuzzy sets rather than crisp ones. This combines parameterization with graded membership.

%     \item \textbf{Fuzzy Multisets}: Assign a multiset of membership counts or grades instead of a single grade. Useful when the frequency or repetition of membership observations is significant.

%     \item \textbf{L-Fuzzy Sets} \cite{Goguen1967}: Generalize Type-1 sets by taking membership grades from an arbitrary lattice L, not restricted to the interval $[0,1]$. This provides an order-theoretic abstraction.

%     \item \textbf{Complex Fuzzy Sets}: Membership grades are complex numbers, typically $r \cdot e^{i\theta}$. This encodes both magnitude (degree of membership) and phase (additional information).

%     \item \textbf{Quantum Fuzzy Sets} \cite{Pykacz2015_Quantum}: Interpret a quantum register's state-vector as the characteristic function of a fuzzy set. This applies principles from quantum mechanics to model membership.

%     \item \textbf{Plithogenic Sets} \cite{Smarandache2018_Plithogenic}: Generalize crisp, fuzzy, IFS, and NS by characterizing elements with multiple attributes. Each attribute value has an appurtenance degree and a contradiction degree.

%     \item \textbf{Hypersoft Sets} \cite{Smarandache2018_Hypersoft}: Extend soft sets to handle tuples or hierarchies of parameters (multi-argument approximate functions). This addresses multi-attribute decision-making.
%     \begin{itemize}
%         \item \textbf{Fuzzy Hypersoft Sets}: Combine fuzzy membership with the multi-parameter structures of hypersoft sets.
%         \item \textbf{Plithogenic Soft Sets \& Plithogenic Hypersoft Sets}: Merge Plithogenic fuzziness into soft set or hypersoft set parameterizations.
%     \end{itemize}

%     \item \textbf{Multipolar Fuzzy Sets (m-Polar Fuzzy Sets)} \cite{Chen2014_mPolar}: Generalize bipolar fuzzy sets to $m$ poles, assigning a vector of $m$ membership degrees from $[0,1]^m$. Models multi-polar evidence or criteria.

%     \item \textbf{k-Polar Intuitionistic Fuzzy Sets (IKFS)}: Assign $k$ distinct pairs of intuitionistic membership/non-membership degrees to each element. Captures multi-source or multi-criteria hesitation.

%     \item \textbf{Multi-Neutrosophic Sets}: Evaluate truth, indeterminacy, and falsity through multiple sources, yielding n-tuples of (T,I,F) components.
%     \begin{itemize}
%         \item \textbf{Tripolar Neutrosophic Sets}: A specific type organizing truth, indeterminacy, and falsity along three distinct polar axes.
%     \end{itemize}

%     \item \textbf{Multi-Grade Fuzzy Sets}: Assign several membership grades (e.g., linguistic terms like ``low,'' ``medium,'' ``high'') from a structured, ordered scale. Captures more nuanced membership.
% \end{itemize}
















\begin{figure}[!ht]
    \centering % Center the TikZ picture on the page
    \begin{tikzpicture}[
        % --- Style for Event Nodes (Vertical) ---
        event_v/.style={
            draw=nodecolor,
            thick,
            fill=nodecolor!20,
            rounded corners=3pt,
            minimum height=1.8em, % Retained from original
            text width=4.2cm,    % Reduced width for narrower boxes, encourages wrapping
            align=center,
            font=\small,         % Retained from original
            inner xsep=4pt,      % Fine-tuned padding
            inner ysep=3pt       % Fine-tuned padding
        },
        % --- Style for Connectors ---
        connector/.style={
            draw=connectorcolor,
            thick,
            -{Stealth[length=2mm, width=1.5mm]} % Retained from original
        }
    ]
        % --- Timeline Parameters ---
        \def\ystart{1965}
        \def\yend{2020}
        \def\timelineheight{22} % Desired height of the timeline in cm (Retained)
        \pgfmathsetmacro{\unit}{\timelineheight/(\yend-\ystart)} % cm per year
        
        % Horizontal gaps for node placement
        \def\hgapdefault{2cm} % Default horizontal gap from axis to node's attachment edge
        \def\hgapfar{2.0cm}    % Farther horizontal gap for certain nodes to improve layout
    
        % --- Draw the main vertical timeline axis ---
        \draw[timelinecolor, -{Stealth[length=3mm, width=2.5mm]}, very thick]
            (0,0) -- (0, \timelineheight + 0.8*\unit) node[above, black, font=\small] {Year};
    
        % --- Draw year ticks and labels ---
        \foreach \year in {1965, 1970, 1975, 1980, 1985, 1990, 1995, 2000, 2005, 2010, 2015, 2020} {
            \pgfmathsetmacro{\ypos}{(\year-\ystart)*\unit}
            \draw[timelinecolor, thick] (-0.2, \ypos) -- (0.2, \ypos);
            \node[black, font=\scriptsize, anchor=east] at (-0.3, \ypos) {\year};
        }
    
        % --- Macros for Vertical Timeline Events ---
        % Args: 1:year, 2:side ('L' or 'R'), 3:hgap (horizontal gap), 4:text content, 5:bibkey, 6:node name
        \newcommand{\timelineeventvertical}[6]{
            \pgfmathsetmacro{\eventy}{(\unit*(#1-\ystart))}
            \coordinate (event_axis_point) at (0, \eventy); % Point on the main axis
            \ifstrequal{#2}{R}{% Event on the Right side
                \node[event_v, anchor=west] (#6_node) at (#3, \eventy) {#4 \cite{#5}};
                \draw[connector] (#6_node.west) -- (event_axis_point);
            }{% Event on the Left side
                \node[event_v, anchor=east] (#6_node) at (-#3, \eventy) {#4 \cite{#5}};
                \draw[connector] (#6_node.east) -- (event_axis_point);
            }
        }
    
        % Args: 1:year, 2:side ('L'/'R'), 3:hgap, 4:text, 5:bibkey, 6:node name, 7:y-shift (cm)
        \newcommand{\timelineeventverticalshift}[7]{
            \pgfmathsetmacro{\eventyraw}{(\unit*(#1-\ystart))}
            \pgfmathsetmacro{\eventy}{\eventyraw + #7} % Apply y-shift
            \coordinate (event_axis_point_raw) at (0, \eventyraw); % Connector to actual year
            \ifstrequal{#2}{R}{% Right side
                \node[event_v, anchor=west] (#6_node) at (#3, \eventy) {#4 \cite{#5}};
                \draw[connector] (#6_node.west) -- (event_axis_point_raw);
            }{% Left side
                \node[event_v, anchor=east] (#6_node) at (-#3, \eventy) {#4 \cite{#5}};
                \draw[connector] (#6_node.east) -- (event_axis_point_raw);
            }
        }
    
        % --- Populate the Timeline ---
        % The 3rd argument to event macros is now the horizontal gap (\hgapdefault or \hgapfar)
        % Y-shifts are applied using \timelineeventverticalshift for crowded regions
    
        % 1960s
        \timelineeventvertical{1965}{R}{\hgapdefault}{Classical Fuzzy Sets}{Zadeh1965}{clfs}
        \timelineeventvertical{1967}{L}{\hgapdefault}{L-Fuzzy Sets}{Goguen1967}{lfs}
    
        % 1970s
        \timelineeventvertical{1975}{R}{\hgapdefault}{Type-2 Fuzzy Sets}{Zadeh1975}{t2fs}
    
        % 1980s
        \timelineeventvertical{1982}{L}{\hgapdefault}{Rough Sets}{Pawlak1982_Rough}{rs}
        \timelineeventvertical{1986}{R}{\hgapdefault}{Intuitionistic FS}{Atanassov1986}{ifs}
        \timelineeventvertical{1986}{L}{\hgapdefault}{Type-n FS}{Turksen1986_TypeNRelated}{htfs}
        \timelineeventvertical{1989}{R}{\hgapdefault}{Interval-Valued IFS}{AtanassovGargov1989}{ivifs}
    
        % 1990s
        \timelineeventvertical{1990}{L}{\hgapdefault}{Fuzzy-Rough Sets}{DuboisPrade1990_FuzzyRough}{frs}
        \timelineeventvertical{1994}{R}{\hgapdefault}{Bipolar Fuzzy Sets}{Zhang1994_Bipolar}{bfs}
        \timelineeventvertical{1998}{L}{\hgapdefault}{Neutrosophic Sets}{Smarandache1998_Neutrosophic}{ns}
        \timelineeventvertical{1999}{R}{\hgapdefault}{Soft Sets}{Molodtsov1999_Soft}{ss}
    
        % 2000s
        \timelineeventvertical{2001}{L}{\hgapdefault}{Fuzzy Soft Sets}{MajiBiswasRoy2001_FuzzySoft}{fss}
    
        % 2010s
        \timelineeventvertical{2010}{R}{\hgapdefault}{Hesitant Fuzzy Sets}{Torra2010}{hfs}
        \timelineeventvertical{2010}{L}{\hgapdefault}{Single-Valued NS}{Wang2010_SVNS}{svns}
        
        % Pair: Pythagorean FS (2013 R, Far, Shift Down) and m-Polar Fuzzy Sets (2014 R, Default, Shift Up)
        % Original: Pythagorean FS [30] is further; m-Polar FS [21] is closer.
        \timelineeventverticalshift{2013}{R}{\hgapfar}{Pythagorean FS}{Yager2013_Pythagorean}{pyfs}{-0.2} % Shift down, further
        \timelineeventverticalshift{2014}{R}{\hgapdefault}{m-Polar Fuzzy Sets}{Chen2014_mPolar}{mpfs}{0.5}    % Shift up, closer
    
        % Pair: Picture Fuzzy Sets (2013 L, Far, Shift Down) and Quantum Fuzzy Sets (2015 L, Default, Shift Up)
        % Original: Picture FS [19] is further; Quantum FS [37] is closer.
        % These are 2 years apart, so smaller shifts are sufficient.
        \timelineeventverticalshift{2013}{L}{\hgapfar}{Picture Fuzzy Sets}{Cuong2013_Picture}{pfs}{0}    % Shift down, further
        \timelineeventverticalshift{2015}{L}{\hgapdefault}{Quantum Fuzzy Sets}{Pykacz2015_Quantum}{qfs}{0.3} % Shift up, closer
        
        % Pair: q-Rung Orthopair FS (2017 R, Far, Shift Down) and Plithogenic Sets (2018 R, Default, Shift Up)
        % Original: q-Rung [33] is further; Plithogenic Sets [23] is closer.
        \timelineeventverticalshift{2017}{R}{\hgapfar}{q-Rung Orthopair FS}{Yager2017_qRung}{qrofs}{0.5} % Shift down, further
        \timelineeventverticalshift{2018}{R}{\hgapdefault}{Plithogenic Sets}{Smarandache2018_Plithogenic}{pls}{1} % Shift up, closer
    
        % Pair: Hypersoft Sets (2018 L, Far, Shift Down) and Spherical Fuzzy Sets (2019 L, Default, Shift Up)
        % Original: Hypersoft Sets [36] is further; Spherical FS [18] is closer.
        \timelineeventverticalshift{2018}{L}{\hgapfar}{Hypersoft Sets}{Smarandache2018_Hypersoft}{hs}{0} % Shift down, further
        \timelineeventverticalshift{2019}{L}{\hgapdefault-0.5}{Spherical Fuzzy Sets}{GundogduKahraman2019_Spherical}{sphs}{0.5} % Shift up, closer
    
    \end{tikzpicture}
    \caption{Timeline of fuzzy set types.}
    \label{fig:fuzzy_timeline_vertical} % Consider a new label if keeping both versions
\end{figure}
