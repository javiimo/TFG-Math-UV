\chapter{Results from Analysis for T-norms}

\section{Continuity of T-norms}
\label{app:cont_tnorms}
T-norms are real-valued functions of two variables defined over the compact domain $[0,1]^2$ with values in $[0,1]$, and they satisfy the monotonicity property (that is, $T(x,y) \le T(x',y')$ if $x \le x'$ and $y \le y'$). These properties greatly simplify the study of their continuity. This appendix presents key results from analysis concerning the continuity and semi-continuity of t-norms\footnote{These results also apply to t-conorms, as they do not depend on one or zero identity properties.}.

\begin{notation}{Equivalence of Norms on $\R^n$}
    On a finite-dimensional vector space (like $\mathbb{R}^n$), all norms are equivalent. This is, given any two norms $\|\cdot\|_a$ and $\|\cdot\|_b$ on $\mathbb{R}^n$ satisfy

 $$
 \exists c, C > 0 \text{ such that } \forall x \in \mathbb{R}^n, \quad c\|x\|_a \leq \|x\|_b \leq C\|x\|_a.
 $$
 Since we are working on $\R^2$, the statements are equal for any given norm and we will denote it by \say{$\|\cdot\|$}.
\end{notation}

\begin{definition}[Continuity]
A t-norm $T$ is \emph{continuous at a point} $(x_0, y_0) \in [0,1]^2$ if:
\[
\forall \epsilon > 0, \exists \delta > 0 : \forall (x,y) \in [0,1]^2, \|(x,y) - (x_0,y_0)\| < \delta \implies |T(x,y) - T(x_0,y_0)| < \epsilon
\]
\end{definition}

Intuitively, this means that the value $T(x,y)$ can be made arbitrarily close to $T(x_0,y_0)$ by choosing $(x,y)$ sufficiently close to $(x_0,y_0)$. A t-norm $T$ is \emph{continuous} if it is continuous at every point in its domain $[0,1]^2$.

\begin{definition}[Uniform Continuity]
A t-norm $T$ is \emph{uniformly continuous} if:
\[
\forall \epsilon > 0, \exists \delta > 0 : \forall (x,y),(x',y') \in [0,1]^2, \|(x,y) - (x',y')\| < \delta \implies |T(x,y) - T(x',y')| < \epsilon
\]
\end{definition}
The key difference from pointwise continuity is that $\delta$ depends only on $\epsilon$, not on the specific location in the domain.
\begin{remark}
    Since the domain $[0,1]^2$ of a t-norm is a compact set, a t-norm $T$ is continuous if and only if it is uniformly continuous \cite[p.~30]{Klement2000}.
\end{remark}

Due to monotonicity, a t-norm's continuity over its 2D domain can be verified by checking continuity along its one-dimensional sections:
\begin{proposition}[Continuity via Components for Monotone Functions {\cite[Prop.~1.19]{Klement2000}}]
A t-norm $T$ (or any non-decreasing function on $[0,1]^2$) is continuous if and only if it is continuous in each variable separately. That is, for all fixed $x_0, y_0 \in [0,1]$, the functions $T(x_0, \cdot)$ and $T(\cdot, y_0)$ are continuous.
\end{proposition}

\begin{remark}
Because of the commutativity, for a t-norm
its continuity is equivalent to its continuity in the first component.\cite[p.~16]{Klement2000}
\end{remark}

\section{Semi-continuity of T-norms}
\label{app:semicont-tnorms}

Many theoretical contexts such as fuzzy logic require only less restrictive properties than continuity, such as left-continuity or upper semi-continuity. The following sections present these definitions and derive their relationships using the properties of t-norms.


\begin{definition}[Semi-continuity for T-norms {\cite[Def.~1.20]{Klement2000}}]
    A t-norm $T$ is:
    \begin{itemize}
        \item \emph{Lower semi-continuous (LSC) at $(x_0,y_0)$} if:
        \[
        \forall \epsilon > 0, \exists \delta > 0 :\, T(x,y) > T(x_0,y_0) - \epsilon \quad \forall (x,y) \in \left(]x_0 - \delta, x_0] \times ]y_0 - \delta, y_0]\right) \cap [0,1]^2
        \]
        \item \emph{Upper semi-continuous (USC) at $(x_0,y_0)$} if:
        \[
        \forall \epsilon > 0, \exists \delta > 0 :\, T(x,y) < T(x_0,y_0) + \epsilon \quad \forall (x,y) \in \left([x_0, x_0 + \delta[ \times [y_0, y_0 + \delta[\right) \cap [0,1]^2
        \]
    \end{itemize}
    A t-norm is LSC (USC) if it is LSC (USC) at every point in $[0,1]^2$.
    \end{definition}
    \begin{notation}{Topological semi-continuity}
        The definition above is only valid in metric spaces such as $\R^2$. The topological definitions are: $$T(x_0, y_0) \le \liminf_{(x,y) \to (x_0,y_0)} T(x,y) \text{ for LSC}$$   $$\text{and } T(x_0, y_0) \ge \limsup_{(x,y) \to (x_0,y_0)} T(x,y)\text{ for USC}$$
    \end{notation}
  
    Intuitively, lower semi-continuity means that the function's value does not have an isolated peak relative to its surroundings. More precisely, values $T(x,y)$ approaching from the \emph{bottom-left quadrant} are not significantly lower than $T(x_0,y_0)$. If there's a discontinuity, $T(x_0,
    y_0)$ is at the \emph{bottom} of an upward jump. Conversely, upper 
    semi-continuity means $T(x_0,y_0)$ is at the \emph{top} of a downward 
    jump.

\begin{definition}[One-Sided Continuity]
A t-norm $T$ is:
\begin{itemize}
    \item \emph{Left-continuous in its first component at $(x_0,y_0)$} if:
    \[
    \forall \epsilon > 0, \exists \delta > 0 : \, |T(x,y_0) - T(x_0,y_0)| < \epsilon \quad \forall x \in ]x_0-\delta, x_0]
    \]
    \item \emph{Right-continuous in its first component at $(x_0,y_0)$} if:
    \[
    \forall \epsilon > 0, \exists \delta > 0 :\, |T(x,y_0) - T(x_0,y_0)| < \epsilon \quad \forall x \in [x_0, x_0+\delta[, |T(x,y_0) - T(x_0,y_0)| < \epsilon
    \]
\end{itemize}
For the second component the definition is completely analogous.
$T$ is \emph{left-continuous} (or \emph{right-continuous}) if it is left-continuous (or right-continuous) in both components.
\end{definition}

\begin{remark}
    Let $T$ be a t-norm:
      \begin{align*}
        T \text{ continuous } &\Leftrightarrow T \text{ lower semi-continuous and upper semi-continuous} \\
        T \text{ continuous } &\Leftrightarrow T \text{ left-continuous and right-continuous}
      \end{align*}
\end{remark}


For general functions, semi-continuity and one-sided continuity are unrelated concepts, as semi-continuity is related to the function's range while one-sided continuity deals with domain limits. However, for monotone functions like t-norms, these properties are equivalent. This is particularly useful because one-sided continuity is often easier to establish or analyze from the construction of a t-norm.\\
\begin{proposition}[{\cite[Prop.~1.22]{Klement2000}}]
For a t-norm $T$:
\begin{itemize}
    \item $T$ is lower semi-continuous if and only if $T$ is left-continuous.
    \item $T$ is upper semi-continuous if and only if $T$ is right-continuous.
\end{itemize}
\end{proposition}

For Archimedean t-norms (those for which $T(x,x) < x$ for all $x \in ]0,1[$), an even stronger relationship between one-sided continuity and full continuity exists. This class of t-norms is fundamental in many applications.
\begin{proposition}[{\cite[Prop.~2.16]{Klement2000}}]
  For an Archimedean t-norm $T$, the following are equivalent:
  \begin{enumerate}
      \item[(i)] $T$ is left-continuous.
      \item[(ii)] $T$ is continuous.
  \end{enumerate}
\end{proposition}
This result significantly simplifies the verification of continuity for Archimedean t-norms. If such a t-norm is known to be left-continuous (e.g., due to properties of its additive or multiplicative generator, which are often easier to analyze for one-sided continuity), it is automatically fully continuous. This is a non-trivial property that does not hold for general monotone functions.

\section{Generators}
\label{app:generators_tnorms}
Families of t-norms can be elegantly characterized and constructed through the concept of generators. 
These representations emerged from studying functional equations, particularly the associativity equation
 $T(x, T(y,z)) = T(T(x,y),z)$. The study of this equation began with N. H. Abel's work in the 19th century. J. Aczél later advanced the theory through his research on functional equations. Important insights also came from P. S. Mostert, A. L. Shields, and C. H. Ling's work on topological semigroups \cite[Sec.~5.1]{Klement2000}. Though the theory behind functional equations and topological semigroups is beyond the scope of this work, these fields were essential in developing the theory of t-norm generators.\\

The core idea of a generator is to transform the t-norm operation on $[0,1]$ into a simpler arithmetic operation on a different domain, via a strictly monotonic function. For continuous Archimedean t-norms, the additive generator provides such a framework.

An \textbf{additive generator} of a t-norm $T$ is a strictly decreasing function $t: [0,1] \to [0, \infty]$ such that:
\begin{enumerate}
    \item $t(1) = 0$.
    \item $t$ is right-continuous at $0$, i.e., $t(0) = \lim_{x \to 0^+} t(x)$.
\end{enumerate}
The t-norm $T$ is then constructed from its additive generator $t$ using the formula:
\begin{equation} \label{eq:additive_generator_t_norm}
    T(x,y) = t^{(-1)}( \min(t(0), t(x) + t(y)) )
\end{equation}
where $t^{(-1)}: [0, t(0)] \to [0,1]$ is the pseudo-inverse of $t$, defined as $t^{(-1)}(u) = \sup \{z \in [0,1] \mid t(z) \ge u \}$ (or, due to strict decrease and continuity, often simply the standard inverse on the range of $t$). The inclusion of $\min(t(0), \dots)$ in Equation~\eqref{eq:additive_generator_t_norm} ensures that the argument of $t^{(-1)}$ remains within its domain $[0, t(0)]$.

The value $t(0)$ is particularly significant as it determines the nature of the continuous Archimedean t-norm:
\begin{itemize}
    \item If $t(0) = \infty$, then $\min(t(0), t(x) + t(y)) = t(x) + t(y)$, and the resulting t-norm $T$ is \textbf{strict} (e.g., for $t(x) = -\log(x)$, $T(x,y) = xy$, the product t-norm).
    \item If $t(0)$ is finite (i.e., $t(0) < \infty$), the resulting t-norm $T$ is \textbf{nilpotent} (e.g., for $t(x) = 1-x$, $T(x,y) = \max(0, x+y-1)$, the Łukasiewicz t-norm).
\end{itemize}
This representation via additive generators, as detailed in \cite[Sec.~3.2, Def.~3.25]{Klement2000}, not only simplifies the study of continuous Archimedean t-norms but also provides a constructive mechanism, highlighting the deep connection between their algebraic properties and the analytical properties of their associated generator functions.