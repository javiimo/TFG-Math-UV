\paragraph{Highest Concurrency:} In figure \ref{} we can see the concurrency (i.e., the number of active interventions at a given time) of an example solution. The highest concurrency represents the maximum number of interventions being executed simultaneously at any time. 




\paragraph{Highest Risk:} This attribute measures the average worst-case risk across all interventions, where for each intervention we consider its maximum risk over all time periods and scenarios. Formally, we have:

\[\operatorname{worst}^{(i,\tau)} \coleq \max_{t=1,\dots,d_{i,\tau}} \max_{s\in S} \mathrm{risk}_{s,t}^{(i,\tau)} \quad \text{where} \quad \text{highest\_risk} \coleq \frac{1}{|\mathcal{I}|}\sum_{i\in\mathcal{I}} \operatorname{worst}^{(i,\tau_i)}\]



\paragraph{Winter/Summer/IS-like:} These attributes indicate the proportion of total interventions that occur in each season (Winter, Summer, or Interseason). Note that the proportions sum to more than 1 since interventions can span multiple seasons.