The problem that will be tackled in this chapter is the ROADEF/EURO Challenge 2020: Maintenance Planning Problem \cite{roadef2020}, which was jointly organized by the French Operational Research and Decision Support Society (ROADEF) and the European Operational Research Society (EURO) in collaboration with RTE (Réseau de Transport d'Électricité).\\

The ROADEF/EURO Challenge is a prestigious biennial competition that bridges the gap between academia and industry by inviting researchers to tackle complex, real-world optimization problems. The 2020 edition was motivated by operational necessity and strategic foresight in the energy sector: while preventive maintenance is essential for grid reliability, the act of taking a power line out of service temporarily weakens the network, increasing its vulnerability to unexpected contingencies like extreme weather or other equipment failures. This challenge has become significantly more complex due to the energy transition. The increasing integration of intermittent renewable energies creates new operational dynamics and constraints, making traditional planning methods insufficient. RTE had already developed a unique strategy to quantify the financial risks of every potential maintenance task across thousands of future scenarios. The core purpose of the challenge was to invite the world's top researchers to develop a powerful optimization engine for the second, most complex step of this process: using this massive, pre-calculated risk data to build a feasible and robust annual maintenance schedule.\\

The competition was organized in a systematic multi-phase structure to identify optimal solutions from an international pool of participants. A total of 74 teams from over 20 countries participated across junior (pre-PhD) and senior tracks. The competition proceeded through three main phases, with each phase utilizing different sets of 15 problem instances representing different maintenance planning scenarios. The qualification phase required teams to submit solutions for a public instance set (Set A), with the top 15 teams advancing based on a point-per-instance scoring system. In the semi-final phase, qualified teams tackled a larger instance set (Set B), resulting in 13 finalists (10 senior and 3 junior). The final phase challenged teams with both a known set (Set C) and a hidden set (Set X) of instances. Final rankings were determined through two distinct evaluation runs on the organizers' machines: a 15-minute "fast score" run and a decisive 90-minute "quality score" run. The winner was ultimately announced at the EURO 2021 conference. \\


The competition employed a progressive evaluation methodology utilizing increasingly complex instances to assess algorithm effectiveness and scalability. For our purposes, instance difficulty was quantified through objective function scores. The X dataset comprised the most challenging problems, characterized by significant risk penalties during peak demand periods. Therefore, instance \texttt{X12} was selected for detailed analysis despite having only the second-highest objective score after \texttt{X14} which achieved the highest score. The reason is that it demonstrated greater improvement between the 15-minute and 90-minute computational runs. This behavior indicates that extended computation times could generate more diverse scheduling alternatives for our multi-criteria optimization problem.


\section{ Original Problem Description}
The maintenance planning problem aims to schedule the starting times of a set of interventions (i.e., maintenance tasks) over a time horizon (typically one year, discretized into days or weeks). The goal is to minimize risk-related costs represented as a single objective function \ref{eq:objective}, while satisfying resource and exclusion constraints. 
This is formalized in table \ref{tab:variables} and as an integer programming optimization problem (following the description from \cite{ConsueloRoadef}):\\


\begin{table}[h]
  \centering
  \begin{tabular}{|c|p{13cm}|}
    \hline
    \textbf{Symbol} & \textbf{Description} \\ \hline
    $T$ & Total number of time periods in the planning horizon. \\ \hline
    $t$ & Time period index, with $t \in \{1,\ldots,T\}$. \\ \hline
    $\tau$ & Starting time period for interventions, with $\tau \in \{1,\ldots,T\}$. \\ \hline
    $I$ & Set of interventions. \\ \hline
    $x_{i,\tau}$ & Binary variable: 1 if intervention $i\in I$ starts at period $\tau$, 0 otherwise. \\ \hline
    $d_{i,\tau}$ & Duration of intervention $i$ if started at period $\tau$. \\ \hline
    $C$ & Set of resources (teams). \\ \hline
    $l_{c,t},\, u_{c,t}$ & Lower and upper bounds on the availability of resource $c\in C$ at period $t$. \\ \hline
    $r_{c,t}(i,\tau)$ & Consumption of resource $c$ in period $t$ by intervention $i$, if started at $\tau$. \\ \hline
    $\mathcal{E}$ & Set of exclusion triplets $(i_1,i_2,t)$ (interventions $i_1,i_2$ cannot overlap at $t$). \\ \hline
    $S_t$ & Set of scenarios for period $t$. \\ \hline
    $\mathrm{risk}_{s,t}^{(i,\tau)}$ & Risk cost in scenario $s\in S_t$ for intervention $i$ (if started at $\tau$) during period $t$. \\ \hline
    $\alpha$ & Weight in the objective, with $\alpha\in[0,1]$. \\ \hline
  \end{tabular}
  \caption{Main Variables and Parameters}
  \label{tab:variables}
\end{table}

\vspace{1em}


\noindent\textbf{Objective:}
\begin{equation}
\min \; \alpha\,\mathrm{obj}_1 + (1-\alpha)\,\mathrm{obj}_2(\beta) \qquad \text{ with }\alpha \in [0,1]
\label{eq:objective}
\end{equation}
where
\[
\begin{aligned}
&\mathrm{obj}_1 = \frac{1}{T}\sum_{t=1}^{T} \mathrm{risk}_t,\quad \mathrm{obj}_2(\beta) = \frac{1}{T}\sum_{t=1}^{T} \max\Big\{0,\;Q_t^\beta-\mathrm{risk}_t\Big\},\\[1ex]
&\quad\text{with: }\quad\mathrm{risk}_t = \frac{1}{|S_t|}\sum_{s\in S_t}\sum_{\substack{i\in I\\ \tau \le t < \tau+d_{i,\tau}}} \mathrm{risk}_{s,t}^{(i,\tau)}\,x_{i,\tau},\quad Q_t^\beta=\beta\text{-quantile of } \Big\{\mathrm{risk}_{s,t} : s\in S_t\Big\}.
\end{aligned}
\]
\noindent Here, $\mathrm{obj}_1$ is the mean cumulative planning risk and $\mathrm{obj}_2(\beta)$ quantifies the extremely high-risk scenarios and periods we want to avoid.

\vspace{1em}
\noindent\textbf{Subject to:}

\noindent Each intervention is scheduled exactly once:
\[
\sum_{\tau=1}^{T-d_{i,\tau}+1} x_{i,\tau} = 1,\quad \forall\, i\in I
\]

\noindent Intervention finishes within the horizon:
\[
t + d_{i,\tau} \le T+1,\quad \forall\, i\in I \text{ and } t \text{ with } x_{i,\tau}=1
\]

\noindent Resource capacity limits:
\[
l_{c,t} \le \sum_{i\in I}\sum_{\tau \le t < \tau+d_{i,\tau}} r_{c,t}(i,\tau)\,x_{i,\tau} \le u_{c,t},\quad \forall\, c\in C,\; \forall\, t
\]

\noindent Exclusion constraints:
\[
\sum_{\tau:\, t\in [\tau,\,\tau+d_{i_1,\tau}-1]} x_{i_1,\tau} + \sum_{\tau:\, t\in [\tau,\,\tau+d_{i_2,\tau}-1]} x_{i_2,\tau} \le 1,\quad \forall\,(i_1,i_2,t)\in \mathcal{E}
\]

\noindent Binary variables:
\[
x_{i,\tau} \in \{0,1\},\quad \forall\, i\in I,\; \tau=1,\ldots,T-d_{i,\tau}+1
\]


