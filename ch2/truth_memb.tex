\signal{Aquí explicaré varias formas de dar esos valores.}

Fuzzy sets, as discussed in chapter \ref{ch:intro}, provide a way to encode data. In the case of MCDM  problems, we are particularly interested in modeling the vagueness of fuzzy attributes. In the previous chapter, the starting point always assumed a fuzzy set was given (and therefore its membership function). However, there isn't a unique way to determine the membership value of an object to a fuzzy set. In general it depends on the definition and the kind of information the fuzzy set encodes but in most of the applications, there will be choices to be made that won't come from a universal objective truth everyone agrees with (whose existance is a topic of philosophy, not addressed in this work) but rather from a more practical perspective often involving subjectivity. Instead of seeing this as a flaw and a lack of rigor, it can be considered a major advantage for mathematical models that need to take into account the opinion of stakeholders. It is only problematic when its  \\

There isn't a standard 

\paragraph{}

\paragraph{Linguistic Variables}
1. Base Variable and Linguistic Values:
A linguistic variable (e.g., Age) is typically associated with a numerical base variable (e.g., the chronological age from 0 to 100).
The words or sentences that the linguistic variable can take (e.g., young, old, very young) are its linguistic values.
A linguistic value like young is interpreted as a label for a fuzzy restriction on the values of the base variable. This fuzzy restriction defines the meaning of the linguistic value.
2. Compatibility Function (The Meaning):
The meaning of a linguistic value is defined by a compatibility function. This function associates each value of the base variable with a number in the interval [0, 1].
This number represents the compatibility of the numerical value with the linguistic label. For example, the compatibility of the numerical age 28 with the linguistic value young might be 0.7, while the compatibility of age 35 might be 0.2 (p. 4, Fig. 1).
The paper explicitly states that compatibility is distinct from probability. It is a subjective measure of how well a numerical value fits a linguistic concept.