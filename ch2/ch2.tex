\chapter{Multi-Criteria Decision Making}

For convenience, we will use the abbreviation MCDM to refer to Multi-Criteria Decision Making throughout this chapter and the remainder of the book.



\section*{Motivation}
We will start explaining the difference between Multicriteria Decision-Making (MCDM) and Multiobjective Optimization with the following problem:\\

Imagine a team working on \textnormal{sustainable building design} wants to find the best design considering only 2 variables: the amount of natural lightning and the energy efficiency. Then they model it mathematically:\vspace{-0.7em}
\begin{itemize}
    \item Both properties of a building design will be quantified with continuous real variables ranging from 0 to 5.\vspace{-0.8em}
    \item Technical limitations give a well defined a feasible region.\vspace{-0.7em}
\end{itemize}
The key challenge is balancing natural lighting and energy efficiency. More natural light means we need larger windows, but this reduces the building's thermal insulation. On the other hand, smaller windows help save energy but don't let in as much daylight.\\

Formally, this is a Multiobjective Optimization Problem where we maximize two functions simultaneously: natural lighting and energy efficiency. Such problems generally do not have a unique optimal solution where both objectives reach their maximum values, but rather a set of solutions called the \textit{Pareto frontier}: a subset of feasible solutions where improving one objective necessarily requires worsening another\footnote{This condition is called Pareto Optimality.}.\\

Once the Pareto frontier has been identified
\footnote{In general, Multiobjective Optimization problems are not tractable, often involve many dimensions, and cannot be solved analytically like the simple example plotted in \ref{fig:pareto_frontier}. Common solution approaches include metaheuristic algorithms such as Goal Programming and Evolutionary Algorithms.}
(see figure \ref{fig:pareto_frontier}), it cannot be directly presented as a solution to the customer since it represents an infinite set of building designs. Therefore, the team selects 5 representative options to present to the customer. \\

This leads to a new decision problem: how should the customer choose among these 5 options? The decision requires evaluating both objective criteria (like number of rooms and floor area) and subjective criteria (like aesthetic appeal and practical layout). A decision maker must determine which criteria to consider, their relative importance, and how to assess subjective attributes. This type of selection problem is what we call a MCDM problem. The options (in this case, building designs) will be assumed to be given.

\begin{figure}[ht]
    \centering
    \includegraphics[width=0.6\textwidth]{ch2/figures/pareto.png}
    \caption{Pareto Frontier, feasible region and proposed designs for Sustainable Building Design}
    \label{fig:pareto_frontier}
\end{figure}


\section{Crisp MCDM Methods}
\signal{Aquí pues tengo varios tipos, ver cuales. En principio:

Utility function methods (e.g., Multi‐Attribute Utility Theory, additive or multiplicative value functions)

Outranking methods (e.g., ELECTRE, PROMETHEE)

Sorting/Assignment Methods
Some methods are designed not to rank alternatives but to assign them into predefined categories (for instance, ELECTRE TRI or other multiple criteria sorting techniques).

Reference point methods (e.g., methods where the DM sets aspiration levels: TOPSIS and VIKOR)

Pairwise comparison methods (e.g., AHP, ANP)

Interactive methods (where the decision maker iteratively refines preferences)}
\section{Fuzzy aggregation}
\input{ch2/fuzzy_avg}