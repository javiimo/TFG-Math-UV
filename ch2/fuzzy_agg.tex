% Para qué usamos en el MDCM el fuzzy? Para manejar Uncertainty y subjectivity.

% Information integration (aggregation)

% Distance measures

% Preference relations

% \signal{
%     Entropy desde el punto de vista de la posibilidad es el que más específico es. Es máxima si el conjunto tiene un unico punto, entonces sabes que es super especifico, ese concepto corresponde a ese unico punto y a nada más. Desde el punto de vista de la informacion es cuanto más informativo, mayor entropía. Desde el punto de vista de la probabilidad es cuanto más te distingue los sucesos: la delta de dirac tiene mínimo y la uniforme máximo o algo así era. 
%     Eso lo tengo que repasar.
%     }\\


% Cosas para mencionar sobre fuzzy en MCDM:

% \signal{OWA operators, y sus generalizaciones. Orness, andness, orlike y andlike.

% Entropía de un OWA, quantifiers

% Fuzzy implication operators para la importancia de los OWA

% Fuzzy ratings, que es como tener 2 fuzzy weights y así incorporas la linguistic variable.

% Fuzzy reasoning: tienes fuzzy rules y las agregas con un OWA por ejemplo. Puedes hacer la implicación y luego agregar o agregar y luego hacer la implicación.

% MICA operators es la clase más general de operators en fuzzy modeling.

% 3 mecanismos de MISO fuzzy system

% Compositional rule of inference

% Generalized method of case inference rule

% Interdependencia de los criterios!

% Lo de que la importancia es relativa, puedo tener unos criterios sí y otros no y tal y que eso me cambie el grado de importancia.}

\signal{Aquí pensaba definir agregación en general como una función. Luego presento los OWAs y sus variaciones. Hasta aquí sí que lo tengo claro.

Pero para seguir, quería simplemente poner los métodos que vaya a usar en la aplicación práctica del tema 3. No sé si un fuzzy TOPSIS o ELECTRE o alguno de esos que tengo en el diagrama de arriba. O si por el contrario puedo continuar con generalizaciones de los OWA pero eso ya sería meterme otra vez con fuzzy measures y acabaría en la integral de Choquet, de Sugeno o de Shilkret o alguna de esas. También otro método sería meterme en lo de reglas de inferencia, ya que he empezado con lo de Pavelka y el razonamiento aproximado, podría ser interesante igual.}