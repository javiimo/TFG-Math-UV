\section{Vagueness and Sorites Paradox}

Among the uncertainty types discussed above, vagueness is most relevant to the following chapters, as we will model it using fuzzy sets.\\

Fuzzy sets extend classical (or crisp) sets by introducing partial memberships, moving beyond the binary notion of "belongs" or "doesn't belong". This provides a more expressive framework for modeling concepts that cannot be adequately represented using traditional sets.\\

To demonstrate the necessity of fuzzy sets and following \cite{HájekSorites}, let us consider the Sorites Paradox. Attributed to the philosopher Eubulides, this paradox emerges from the following type of reasoning:

\begin{enumerate}
    \item A single grain of sand does not form a heap.
    \item Adding one grain to something that is not a heap does not make it a heap.
    \item Consequently, there are no heaps.
\end{enumerate}


There are many variations of this paradox with different vague concepts such as age, size, height, baldness, etc. All share the same issue: there is a gradual transition, and evaluating to only true or false does not allow us to represent it adequately, therefore reaching a contradiction.\\

This paradox can be resolved by considering partial membership to the set of heaps, where a single grain may have 0.001 membership which grows with the number of grains until reaching a group of a million grains, which could have 1 membership. While these numbers are arbitrary, this model better captures our intuitive concept of a heap than the conclusion that "heaps don't exist".\\

However, in the paradox we are dealing with truth values rather than membership. In this case, the concepts are completely analogous, as there is a direct relationship between the membership degree of an amount of grains to the set of heaps and the truth value of the proposition \textit{"This amount of grains is a heap"}. In the paper \cite{HájekSorites}, the authors argue that the proposition "\textit{Adding one grain to something that is not a heap does not make it a heap}" is not strictly true but rather almost true, thus by considering varying degrees of truth we avoid the contradiction.\\

This example illustrates how vagueness is the core issue with these concepts: the boundary of the set of heaps is not crisp but rather fuzzy. There is no precise point where something transitions from not being a heap to being one, it is a gradual process. \\

In the following chapters, we will demonstrate that fuzzy logic and the treatment of vagueness is not merely an academic curiosity, but rather a powerful framework with both practical applications and rigorous theoretical foundations.